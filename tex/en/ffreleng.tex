\begin{aosachapter}{Firefox Release Engineering}{s:ffreleng}{Chris AtLee, Lukas Blakk, John O'Duinn, and Armen Zambrano Gasparnian}

The Mozilla Release Engineering team has made a lot of advances
recently in how the release automation for Firefox works. We have
reduced the requirements for human involvement during signing, sending
notices to various stakeholders throughout the process, and automated
many other small manual steps. Each manual step in the process is an
opportunity for human error. While what we have now isn't perfect,
we're always striving to streamline and automate our release
process. Our final goal is being able to push a button and walk away,
with minimal human interventions, thereby eliminating many of the
headaches and do-overs we experienced with our older part-manual,
part-automated release processes. In this chapter, we will explore and
explain the infrastructure decisions as well as scripts that comprise
the complete Firefox rapid release system as of Firefox 10.

You'll follow the system from the perspective of a release-worthy
Mercurial changeset as it is turned into a release candidate---and
then a public release---available to over 400 million daily users
worldwide.  We'll start with builds and code signing, then customized
partner and localization repacks, the QA process, and how we generate
updates for every supported version, platform and localization. All of
these steps must be completed before any release can be pushed out to
Mozilla Community's network of mirrors which provide the downloads to
our users.

We'll look at some of the newer decisions that have been made to
improve this process like our sanity-checking script that helps
eliminate much of what used to be vulnerable to human error, our
automated signing script, our integration of mobile releases into the
desktop release stream, and the land of patcher/AUS where updates are
created and served to our users across multiple versions of the
software.

Follow a Mozilla Firefox release from the moment the Release
Coordinator gives the official ``Go'' to when it's available for
download (or update) to your computer or mobile device.

\begin{aosasect1}{Look N Ways Before You Start a Release}

This section describes the mechanics of how we generate release builds
for the Firefox product. Most of this chapter details the significant
steps in a release process that occurs once the builds start, but
there is also plenty of complex cross-group communications to deal
with before Release Engineering even starts to generate release
builds, so lets start there.

\aosafigure[100pt]{../images/ffreleng/gotobuild.png}{Getting From Code to ``Go to build''}{fig.ffreleng.getting}

When we started on the project to improve Mozilla's release process,
we began with the premise that the more popular Firefox became, the
more users we would have, the more attractive a target Firefox would
become to ``blackhat hackers'' looking for security vulnerabilities to
exploit. Also, the more popular Firefox became, the more users we
would have to protect from a newly discovered security vulnerability,
so the more important it would be to deliver a security fix as quickly
as possible. We even have a term for this---a ``chemspill''
release. Instead of being surprised by the occasional need for a
chemspill release in between our regularly scheduled releases, we
decided to plan as if every release could be a chemspill release, and
designed our release automation accordingly.

This mindset has three important side-effects:

\begin{aosaenumerate}

\item We do a postmortem after \emph{every} release, and look to see
  where things could be made smoother, easier, and faster next
  time. If at all possible, we find and fix any one thing immediately,
  no matter how small, before the next release. This constant
  polishing of our release automation means we're always looking for
  new ways to rely on less human involvement while also enhancing
  robustness and speeding up turnaround time. A lot of effort is spent
  making our tools and processes bulletproof so that ``rare'' events
  like network hiccups, disk space issues or typos made by real live
  humans are caught and handled as early as possible.  Even though
  we're already fast enough for regular, non-chemspill releases, we
  continue to want to reduce the risk of any human error in a future
  release, especially in a chemspill release.

\item When we do have a chemspill release, the humans in Release
  Engineering are not stressed by it. We're used to the idea of going
  as fast as possible with calm precision, and we've built tools to do
  this as safely and robustly as we know how. Less stress means more
  calm and precise work, on a well-rehearsed process, which in turn
  helps chemspill releases go smoothly.

\item We created a Mozilla-wide ``go to build'' process. When doing a
  a non-chemspill release, its possible to have everyone looking
  through the same bug triage queries, have everyone see clearly when
  the last fix was landed, and tested just fine, and have consensus on
  when to start builds. However, in a chemspill release---where
  minutes matter---keeping track of all the details of the issue,
  following up bug confirmations and fixes, gets very tricky very
  quickly.  To reduce complexity, and the risk of mistakes, Mozilla
  now has a full-time person---someone explicitly not in Release
  Engineering---track the readiness of the code for a ``go to build''
  decision.  Changing processes during a chemspill is risky, so in
  order to make sure everyone is familiar with the process when
  minutes matter, we use this same process for ``chemspill'' and
  ``regular'' releases.

\end{aosaenumerate}

\aosafigure{../images/ffreleng/timeline.png}{Complete Release Timeline}{fig.ffreleng.timeline}

\end{aosasect1}

\begin{aosasect1}{"Go to Build"}

\begin{aosasect2}{Who Can Send the ``Go to Build''?}

Before the start of the release, one person is designated to assume
the responsibility for the entire release. It is worth noting that
this person is not in Release Engineering. This person needs to be
someone that everyone trusts to attend triage meetings, have
background context on all the work being landed, referee bug severity
disputes fairly, approve landing of late breaking changes, and make
tough back-out decisions.  Additionally, for the actual release day,
this person is on point for all communications with the different
groups (developers, QA, Release Engineering, website developers, PR,
marketing, {\ldots}).

Different companies use different titles for this type of role.  Some
titles are: Release Manager, Release Engineer (!), Program Manager,
Project Manager, Product Manager, Product Czar, Release Driver. This
chapter will use the term ``Release Coordinator'' as it most clearly
defines the role in our process, but the important point is that the
role, and the final authority of the role, is clearly understood by
everyone before the release starts, regardless of their background, or
previous work experiences elsewhere. In the heat of the moment of a
release day, we all have to abide by, and trust, the final decision
that this person makes.

This Release Coordinator is also the only person outside of Release
Engineering who is authorized to send ``stop builds'' emails if a
show-stopper problem is discovered with the release. Any reports of
``suspected show-stopper problems'' are redirected to the Release
Coordinator, who will evaluate, make the final go/no-go decision and
communicate that decision to everyone in a timely manner.

\end{aosasect2}

\begin{aosasect2}{How to Send the ``Go to Build''?}

Early experiments with sending ``go to build'' in IRC channels, or
verbally over the phone, led to misunderstandings at some point,
occasionally causing problems for the release in progress. Therefore,
we now require that the ``go to build'' signal for every release is
done by email to a mailing list that includes everyone across all
groups involved in release processes. The subject of the email
includes ``go to build'' and the explicit product name and version
number, for example:

\begin{verbatim}
go to build Firefox 6.0.1
\end{verbatim}

Similarly, if a problem found in the release, then the Release
Coordinator will send a new "all stop" email to the same mailing list,
with a new subject line. We found that it was not ok to just hit reply
on the most-recent email about the release---email threading of some
email clients caused some people to not notice the "all stop" email if
it was way down a long and unrelated thread.

\end{aosasect2}

\begin{aosasect2}{What Is In the ``Go to Build'' Email?}

\begin{aosaenumerate}

\item The exact code to be built from. Ideally, give the URL to the
explicit change in your source code repo that the release builds are
to be created from.

  \begin{aosaenumerate2}

    \item Comments like ``use the latest code'' are never ok: In one
      release, after the ``go to build'' email was sent and before
      builds started, a well intentioned developer landed a change
      without approval in the wrong branch. The release included that
      unwanted change in the builds. Thankfully, this mistake was
      caught before we shipped, but we did have to do a full stop and
      rebuild everything.

    \item In a time-based Version Control System, like CVS, be fully
      explicit of the exact time to use. Give time down to seconds,
      and specify timezone. In one release, when Firefox was still
      based on CVS, the Release Coordinator specified the cutoff time
      to be used for the builds, but did not give the timezone. By the
      time Release Engineering noticed the missing timezone info, the
      Release Coordinator was asleep in his timezone. Release
      Engineering correctly guessed that the intent was local time (in
      California), but in a late-night mixup over PDT instead of PST,
      we ended up missing the last critical bug fix. This was caught
      by QA before we shipped, but we did have to stop builds and
      start build over using the correct cutoff time.

    \end{aosaenumerate2}

\item A clear sense of the urgency for this particular release.  This
  sounds so obvious, you might be tempted to not bother including
  it. However, it is important when handling some important edge
  cases, so here is a quick summary:

  \begin{aosaenumerate2}

    \item Some releases are ``routine'', and can be worked on in
      normal working hours. They are a pre-scheduled release, they are
      on schedule, and there is no emergency. Of course, all release
      builds need to be created in a timely manner, but there is no
      need for Release Engineers to pull all-nighters and burn out
      humans for a ``routine'' release.  Instead, we schedule this out
      properly in advance, so everything stays on schedule with people
      working normal hours. This keeps people fresh and better able to
      handle unscheduled urgent work if the need arises.

    \item Some releases are urgent, where minutes matter. These are
      typically to fix a published security exploit, or fix a
      newly-introduced-top-crash problem, impacting a large percentage
      of our userbase. These ``chemspills'' need to be created as
      quickly as possible and are typically not pre-scheduled
      releases.

    \item Some releases change from ``routine'' to ``chemspill'' or
      from ``chemspill'' to ``routine''. For example, if a security
      fix in a ``routine'' release was accidentally leaked, it is now
      a ``chemspill'' release. If a business requirement like a
      ``special sneak preview'' release needed for an upcoming
      conference announcement was suddenly delayed for business
      reasons, the release now changes from ``chemspill'' to
      ``routine''.

    \item Some releases have different people each holding different
      opinions on whether the release is ``normal'' or ``urgent'',
      depending on their perspective on the fixes being shipped in the
      release.

  \end{aosaenumerate2}

\end{aosaenumerate}

It is the role of the Release Coordinator to balance all the facts and
opinions, reach a decision, and then communicate that decision about
urgency consistently across all groups. If new information arrives,
the Release Coordinator reassesses, and then communicates the new
urgency to all the same groups. Having some groups believe a release
is ``urgent'', while other groups believe the release is ``normal''
can be very destructive to cross-group cohesion across an
organization.

Finally, these emails also became very useful to measure where time
was spent during a release.  While these are only accurate to
wall-clock time resolution, this accuracy is really helpful when
figuring out where next to focus our efforts on making things faster.
As the old adage goes, before you can improve something, first you
have to be able to measure it.

Throughout the beta cycle for Firefox, we also do weekly releases from
our \code{mozilla-beta}
repository\footnote{\url{http://hg.mozilla.org/releases/mozilla-beta/}}. Each
one of these beta releases goes through our usual full release
automation and is treated almost identically to our regular final
releases. To minimize surprises during a release, our intent is to
have no new ``untested'' changes to release automation or
infrastructure by the time we start the final release builds.

\end{aosasect2}

\end{aosasect1}

\begin{aosasect1}{Tagging, Building, and Source Tarballs}

\aosafigure{../images/ffreleng/tagging.png}{Automated tagging}{fig.ffreleng.tagging}
    
In preparation for starting automation, we recently started to use a
script that was originally written by a summer Build and Release
intern,
\code{release\_sanity.py}\footnote{\url{http://mxr.mozilla.org/build/source/tools/buildbot-helpers/release_sanity.py}}. This
Python script is relied upon to assist a Release Engineer double-check
all configurations for a release match what is checked into our tools
and configuration repos as well as what is in the specified release
code revisions for mozilla-release and all the languages for this
release (which will be what the builds and language repacks are
generated from).

The script accepts the buildbot config files for any release configs
that will be used (desktop, mobile), the branch to look at (eg:
mozilla-release), the build and version number, and the names of the
products that are to be built (fennec,firefox). It will error out if
all the release repos (mozilla-release and locales) are not properly
matched to what's in the configs, if locale repo changesets don't
match our shipping locales and l10n changeset files, and if the
release version and build number don't match what has been landed in
our build tools under the tag which is generated by using the product,
version, and build number. This script also checks the build
configuration files (mozconfigs) from nightly builds to release builds
where branding and update channels are set, among other things. If all
the tests in this script pass, it will run the step to reconfig the
buildbot master where the script is being run and where release
builders will be triggered and then it generates the sendchange that
starts the automated release process.

% FIXME: should it be 'release_sanity' in the line below?
After a Release Engineer passes release sanity and kicks off builders,
the first automated step in the Firefox release process is tagging all
of the related source code repositories to record which revision of
the source, language repos, and related tools are used for this
version and build number of a release candidate. A single Firefox
release uses code from about 85 version control repositories that host
things such as the product code, localization strings, release
automation code, and helper utilities. Tagging all these repositories
is, therefore, critical to ensure that future steps of the release
automation are all using a consistent set of revisions. It also has a
number of other benefits: Linux distributions and other contributors
can reproduce builds with exactly the same code that goes into the
official builds, it also records the revisions of source and tools
used on a per-release basis for future comparison of what changed
between releases. For Firefox releases, one example tag set is
\code{FIREFOX\_10\_0\_RELEASE FIREFOX\_10\_0\_BUILD1
  FENNEC\_10\_0\_RELEASE FENNEC\_10\_0\_BUILD1}.  These tags allow us
to keep a history of Firefox and Fennec (mobile Firefox) releases'
version and build numbers in our release repos.
    
Once all the repositories are tagged, a series of dependent builders
automatically start up: one builder for each release platform plus a
source bundle that includes all source used in the release.  Source
bundle and built installers are all uploaded to the release directory
as they become available.  This allows anyone to see exactly what code
is in a release and gives a snapshot that would allow us to re-create
the builds if we ever needed to (if our VCS failed somehow).
    
For the Firefox build's source, sometimes we need to do a cut over
from an earlier repo. For example with a beta release this means
pulling in the signed off revision from Mozilla-Aurora (our
more-stable-than-nightly repo) for Firefox 10.0b1. For a release it
means pulling in the approved changes from Mozilla-Beta (typically the
same code used for 10.0b6) to the Mozilla-Release repo.  This release
branch is then created in as a named branch whose parent changeset is
the signed-off revision from the 'go' to build provided by the Release
Coordinator. This release branch can be used to make release-specific
modifications to the source code, such as bumping version numbers, or
finalizing the set of locales that will be built. If a critical
security vulnerability is discovered in the future that requires an
immediate fix (a ``chemspill'' situation), a minimal set of changes to
address the vulnerability will be landed on this relbranch and a new
version of Firefox released. When we have to do another round of
builds for a particular release, buildN, we use these relbranches to
grab the same code that was signed off on for 'go to build' which is
where any changes to that release code will have been landed. The
automation starts again and bumps the tagging to the new changeset on
that relbranch.
    
Our tagging process does a \emph{lot} of operations with local and
remote Mercurial (hg) repositories. To factor out some of the most
common operations we've written a few tools to assist us:
\code{hgtool.py}\footnote{\url{http://hg.mozilla.org/build/mozharness/file/a0fce0162fd5/scripts/hgtool.py}}
and
\code{retry.py}\footnote{\url{http://hg.mozilla.org/build/tools/file/7adc08bd1386/lib/python/util/retry.py}}.
\code{retry.py} is a simple wrapper that can take a given command and
run it, retrying several times if it fails. It can also watch for
exceptional output conditions and retry or report failure in those
cases. We've found it useful to wrap retry.py around most of the
commands which can fail due to external dependencies.  For tagging,
the hg operations could fail due to temporary network outages, web
server issues, or the backend hg server being temporarily
overloaded. Being able to automatically retry these operations and
continue on saves a lot of our time, since we don't have to manually
recover and get the release automation running again.
    
\code{hgtool.py} is a utility that encapsulates several common hg
operations, like cloning/pulling/updating with a single invocation. It
also adds support for hg's share extension, which we use extensively
to avoid having to have several full clones of repositories in
different directories on the same machine.  Adding support for shared
local repositories significant sped up our tagging process since most
full clones of the product and locale repositories could be avoided.
    
An important consideration for factoring out tools like these is to
make our automation as testable as possible. Because tools like
\code{hgtool.py} are small, single purpose utilities built on top of
reusable libraries, they're much easier to test in isolation.

Today our tagging is done in two parallel jobs: one for desktop
Firefox which takes around 20 minutes to complete as it includes
tagging 80+ locale repos, and another for mobile Firefox which takes
around 10 minutes to complete since we have less locales currently
available for our mobile releases. In the future we would like to
streamline our release automation process so that we tag \emph{all}
the various repositories in parallel. The initial builds can be
started as soon as the product code and tools talos requirement
repository is tagged without having to wait for all the locale
repositories to be tagged. By the time these builds are finished, the
rest of the repositories will have been tagged so that localization
repackages and future steps can be completed.  We estimate this can
reduce the total time to have builds ready by 15 minutes.

\end{aosasect1}

\begin{aosasect1}{Localization Repacks and Partner Repacks}

\aosafigure{../images/ffreleng/repacksl10n.png}{Repacking Firefox for Each Localization}{fig.ffreleng.repack}

Once the desktop builds are generated and uploaded to
\code{ftp.mozilla.org}, our automation triggers the localization
repackaging jobs. This consists of a handful of jobs that take the
original build (using the en-US locale), unpacking it and stuffing in
the strings for each locale that we are shipping for this release
(this is why we call them repackages). Each job takes a handful of
locales. This approach allows us to parallelize the jobs across many
machines. More explicitly, instead of doing all 84 localizations in
one machine, we can split them across six different machines and take
approximately a sixth of the time it would have required to do so in
one machine. We could split it even further but we would take away too
many machines from the pool which would affect other unrelated jobs
triggered by our developers as part of our continuous integration
system.
    
The process for mobile (Android) is slightly different as we produce
only two installers: an English version and a multi-language version
with just a dozen languages built into the installer instead of a
build per language. In the future, other languages will be requested
on demand as add-ons from \code{addons.mozilla.org}.

% FIXME: check that this cross-reference is to the right figure
In \aosafigref{fig.ffreleng.repack}, you can see that we currently rely
on three different sources for our locale information:
\code{shipped\_locales}, \code{l10\_changesets} and
\code{l10n-changesets\_mobile-release.json}. There is a plan to move
all three into a unified JSON file.  These files contain information
about the different localizations we have and certain platform
exceptions.  Specifically, for a given localization we need to know
which revision of the repository to use for a given release and we
need to know if the localization can build on all of our supported
platforms (e.g.  Japanese for Mac comes from a different repo all
together).  Two of these files are used for the Desktop releases and
one for the Mobile release (this JSON file contains both the list of
platforms and the changesets).

Who decides which languages we ship? First of all, localizers
themselves nominate their specific changeset for a given release. The
nominated changeset gets reviewed by Mozilla's localization team and
shows up in a web dashboard that lists the changesets needed for each
language. The Release Coordinator reviews this before sending the ``go
to build'' email. On the day of a release, we retrieve this list of
changesets and we repackage them accordingly.

Besides localization repackages we also generate partner
repackages. These are customized builds for various partners we have
who want to customize the experience for their customers.  The main
type of changes are custom bookmarks, custom homepage and custom
search engines but many other things can be changed. These customized
builds are generated for the latest Firefox release and not for betas.

\end{aosasect1}

\begin{aosasect1}{Signing}

In order for users to be sure that the copy of Firefox they have
downloaded is indeed the one we just built, we apply a few different
types of digital signatures to the builds.

The first type of signing done is for our windows builds. We use a
Microsoft Authenticode (signcode) signing key to sign all our
\code{.exe} and \code{.dll} files. Windows can use these signatures to
verify that the application comes from a trusted source. We also sign
the Firefox installer executable with the Authenticode key.  We then
generate a set of MD5 and SHA1 checksums for all the builds, and
generate detached GPG signatures for the checksum files as well as all
the builds. These signatures are used by mirrors and other community
members to validate their downloads.

For security purposes, we sign on a dedicated signing machine that is
blocked off via firewall and VPN from outside connections. Our
keyphrases, passwords, and keystores are passed among Release
Engineers only in secure channels, often in person, to minimize the
risk of exposure as much as possible in a team.

\aosafigure[325pt]{../images/ffreleng/signing.png}{Signing Firefox Installers}{fig.ffreleng.signing}

Until recently this signing process involved a release engineer
working on a dedicated server (the ``signing master'') for almost an
hour manually doing the steps of downloading builds, signing them, and
uploading them back to ftp.mozilla.org before the automation could
continue.  Once signing on the master is completed and all files have
been uploaded, a log file of all the signing activities is uploaded to
the release candidates directory on ftp.mozilla.org.  The appearance
of this log file on ftp.mozilla.org signifies the end of human signing
work and from that point, dependent builders watching for that log
file can resume automation.  Recently we've added an additional
wrapper of automation around the signing steps. The release engineer
merely starts an automated signing process on the signing master then
goes off to do other work while a simple set of Makefile targets do
the heavy lifting.

The release engineer will open a Cygwin shell on the signing master
and set up a few environment variables pertaining to the release like
\code{VERSION}, \code{BUILD}, \code{TAG}, \code{RELEASE\_CONFIG} that
help the script find the right directories on ftp.mozilla.org and know
when all the deliverables for a release have been downloaded so that
the signing can start. After checking out the most recent production
version of our signing tools the release engineer will simply run
'make autosign'. The release engineer will need to manually enter 2
passphrases, one for gpg and one for signcode. After these passphrases
are automatically verified by the make scripts then the automation
starts a download loop that watches for uploaded builds and repacks
from the release automation and downloads them as they become
available.  Once all items have been downloaded, the automation will
begin signing immediately, without human intervention. Not needing a
human to sign is important for two reasons. Firstly, it reduces the
risk of human error and it saves the human manual work for an
hour. Secondly, and more importantly we believe, it allows signing to
proceed during non-work hours, without needing a release engineer
awake at his computer at the time.  The win32 installers are signed
with Authenticode and then we generate gpg signatures for all the
platforms.

All deliverables have an MD5SUM and SHA1SUM generated for them and
those hash values are written to files of the same name.  These files
will be uploaded back to the release-candidates directory as well as
synced into the final location of the release on ftp.mozilla.org once
it is live so that anyone who downloads a Firefox installer from our
mirrors can ensure they got the correct object. When all signed bits
are available and verified (we do a quick and more in-depth
verification process) they are uploaded back to ftp.mozilla.org along
with the signing log file that the automation is waiting for to
proceed.

Our next round of improvements to the signing process will create a
tool that allows us to sign bits at the time of build/repack. This
work requires creating a signing server application that can receive
requests to sign files from the release build machines. This work also
requires a signing client tool, which would contact the signing
server, authenticate itself as a machine trusted to request signing,
upload the files to be signed, wait for the build to be signed,
download the signed bits, and then include them as part of the
packaged build.  Once these enhancements are in production, we can
discontinue our current all-at-once signing process as well as our
all-at-once generate-updates process (more info below). We expect this
work to trim a few hours off our current end-to-end times for a
release.
  
\end{aosasect1}

\begin{aosasect1}{Updates}

Updates are created so users can update to the latest version of
Firefox quickly and easily using our built-in updater without having
to download a standalone installer, and then run the installer
program. From the user's perspective, the downloading of the update
package happens quietly in the background. Only after the update files
are downloaded, and ready to be applied, will Firefox prompt the user
and give the user the option to apply the update and restart.

The catch is, we generate a \emph{lot} of updates. For a given release
train, we generate updates from all supported previous releases along
that release train to the new latest release for that train.  That
means generating updates for every platform, every locale, every
installer from Firefox N, N-1, N-2, N-3, {\ldots} all up to Firefox
LATEST in both complete and partial forms. We do all this for several
release trains at a time.

Our update generation automation bumps the update configuration files
of each release's build off a branch to maintain our canonical list of
what version numbers, platforms, and localizations need to have
updates created to offer users this newest release. We offer updates
as ``snippets''. As you can see in the example below, this snippet is
simply an XML pointer file hosted on our AUS (Application Update
Service) that informs the user's Firefox browser where the complete
and/or partial \code{.mar} (Mozilla Archive) files are hosted.

\begin{aosasect2}{Major Updates vs.\ Minor Updates}

\begin{figure}  
\begin{verbatim}
<updates>
    <update type="minor"  version="7.0.1" extensionVersion="7.0.1" buildID="20110928134238" detailsURL="https://www.mozilla.com/en-US/firefox/7.0.1/releasenotes/">
        <patch type="complete" URL="http://download.mozilla.org/?product=firefox-7.0.1-complete&os=osx&lang=en-US&force=1" hashFunction="SHA512" hashValue="7ecdbc110468b9b4627299794d793874436353dc36c80151550b08830f9d8c5afd7940c51df9270d54e11fd99806f41368c0f88721fa17e01ea959144f473f9d" size="28680122"/>
        <patch type="partial" URL="http://download.mozilla.org/?product=firefox-7.0.1-partial-6.0.2&os=osx&lang=en-US&force=1" hashFunction="SHA512" hashValue="e9bb49bee862c7a8000de6508d006edf29778b5dbede4deaf3cfa05c22521fc775da126f5057621960d327615b5186b27d75a378b00981394716e93fc5cca11a" size="10469801"/>
    </update>
</updates>
\end{verbatim}
\caption{Sample Update Snippet}
\label{fig.ffreleng.snippet}
\end{figure}

As you can see in \aosafigref{fig.ffreleng.snippet}, update snippets
have a ``type'' attribute which can be either ``major'' or ``minor''.
Minor updates are used to keep people updated to the latest version
available in their specific release train; for example update all
3.6.* release users to the latest 3.6 release, all rapid-release beta
users to the latest beta, all mozilla-central nightly users to the
latest nightly build, etc.  Most of the time, updates are minor and
don't require any user interaction other than a confirmation to apply
and restart the browser.

Major updates are used when we need to advertise to our users that the
latest and greatest release was available and to prompt the user
letting them know ``A new version of Firefox is available, would you
like to update?'', with a billboard showcasing the leading new
features in the new release.  Our new rapid-release versions means no
longer needing to do as many major updates, we'll be able to stop
generating them once the 3.6.* release train is no longer supported.

\end{aosasect2}

\begin{aosasect2}{Complete Updates vs.\ Partial Updates}

At build time, we generate ``complete update'' \code{.mar} files which
contain all the files for the new release compressed with \code{bz2}
and then archived into a \code{.mar} file. Both complete and partial
updates are downloaded automatically through the update channel a user
is on.

Partial update mars are created by comparing the complete mar for the
old release with the complete mar for the new release and creating the
``partial update'' \code{.mar} file containing the binary diff of any
changed files and a manifest file. As you can see in the sample
snippet above, this results in a much smaller file size for partial
updates. This is very important for users in locations with slower or
dialup internet connections.

In older versions of our update automation the generation of partial
updates for all locales and platforms could take 6-7 hours for one
release as the complete mars downloaded, diffed, then packaged up in a
partial update .mar file. Eventually it was discovered that even
across platforms, many component changes were identical and so with a
script that cached the hash for each part of the diff our partial
update creation time was brought down to approximately 40 minutes
since so many diffs could be re-used. After the snippets have been
uploaded and are hosted on AUS, an update verification step is run to
a) test downloading the snippets and b) run the updater with the
downloaded mar to confirm that the updates apply correctly.

Generation of partial update mars as well as all the update snippets
is currently done after signing is complete. We do this because
generation of the partial updates must be done between signed files of
the two releases, and therefore generation of the snippets must wait
until the signed builds are available.  Once we're able to integrate
signing into the build process, we can generate partial updates
immediately after completing a build or repack. Together with
improvements to our AUS software, this means that once we're finished
builds and repacks we would be able to push immediately to
mirrors. This effectively parallelizes the creation of all the
updates, trimming several hours from our total time.
  
\end{aosasect2}

\end{aosasect1}

\begin{aosasect1}{Pushing to Internal Mirrors and QA}

Verifying that the release process is producing the expected
deliverables is key for producing the right bits for our users. This
is accomplished by QA's verification and sign offs process along the
way to ensure that everything is going according to the plan.

Once the signed builds are available, QA starts manual and automated
testing of the signed builds. QA relies on a mix of community members,
contractors and employees in different timezones to speed up this
verification process. Meanwhile, our release automation generates
updates for all languages and all platforms, across the appropriate
releases. Once the update snippets are ready, which is typically close
to the time that QA has finished verifying the signed builds, then QA
verifies that users can safely update from various previous releases
to the newest release one using these updates.

Mechanically, our automation pushes the binaries to our ``internal
mirrors'' (Mozilla hosted servers) in order for QA to verify
updates. Only after QA has finished verification of the builds, and
the updates, will we then push these same builds and updates to our
community mirrors. These community mirror are essential for Mozilla to
be able to handle the global load of users requesting their updates
from local mirror nodes instead of reaching ftp.mozilla.org
directly. Its worth noting that we do not make builds and updates
available on the community mirrors until after QA signoff because of
complications that arise if QA find a last minute showstopper is
found, and Mozilla needs to withdraw a candidate build from the
community mirrors.

The validation process after builds and updates are generated is:

\begin{aosaitemize}

\item QA, along with community and contractors in other timezones, do
  manual testing.

\item QA triggers the automation systems to do functional testing.

\item QA verifies that fixed problems for that release are indeed
  fixed.

\item The release automation meanwhile is generating the updates.

\item QA signs off the builds.

\item QA signs off the updates.

\end{aosaitemize}
  
Note that users don't get the updates until QA has signed off and the
Release Coordinator has sent the email asking to push the builds and
updates live. This is covered in the next section.
  
\end{aosasect1}

\begin{aosasect1}{Pushing to Public Mirrors and AUS}

Pushing this latest Firefox release public is then pretty
straightforward. The Release Coordinator gives the go-ahead to have
the files pushed to our community mirror network. We rely on our
community mirrors to be able to handle a few hundred million users
downloading updates over the next few days. All the installers,
complete and partial updates for all platforms and locales are already
on our internal mirror network at this point. Publishing the files to
our external mirrors involves changes to an rsync exclude file for the
public mirrors module.  Once this change is made, the mirrors will
start to synchronize the new release files. Each mirror has a score or
weighting associated with it, and we monitor which mirrors have
synchronized the files and sum their individual scores to compute a
total ``uptake'' score. Once a certain uptake is reached, we notify
the Release Coordinator that the mirrors have enough uptake to handle
the release.

This is the point at which the release becomes ``official''. After the
Release Coordinator sends the final ``go live'' email, Release
Engineering will update some symlinks on the web server so that
visitors to our web and ftp sites can find the latest new version of
Firefox. We also publish all the update snippets for past versions of
Firefox to AUS. Firefox on users' machines regularly checks our AUS
servers to see if there's an updated version of Firefox available for
them. Once we publish these update snippets, users are able to
automatically update Firefox on their machines to the latest version.
  
\end{aosasect1}

\begin{aosasect1}{Working in the Open}

At Mozilla we strive to do everything in the open. The Firefox
developers make all their code changes in the open. Similarly, all of
Release Engineering's code and blow-by-blow build notes are public,
available for everyone to see.  Here are some links for further
reading:

\begin{aosaitemize}  

\item Chris AtLee's blog: \url{http://atlee.ca/blog}.

\item Lukas Blakk's blog: \url{http://crashopensource.blogspot.com}.

\item John O'Duinn's blog: \url{http://oduinn.com}.

\item Armen Zambrano Gasparnian's blog: \url{http://armenzg.blogspot.com}.

\item Documentation on the design and flow of our Mercurial-based release process: \url{https://wiki.mozilla.org/Release:Release\_Automation\_on\_Mercurial:Documentation}.

\item Release Engineering's build repositories:
  \url{http://hg.mozilla.org/build}.  In particular, the
  buildbotcustom, buildbot-configs, and tools repositories are used
  heavily for releases.

\item The Firefox 7.0 Beta 4 Build Notes:
  \url{https://wiki.mozilla.org/Releases/Firefox_7.0b4/BuildNotes}.
  In addition to code, we document every aspect of a release. Here's
  an example from our 7.0b4 release, but you can find all our release
  notes if you edit the URL appropriately.

\end{aosaitemize}

In the last few years, we've made a lot of changes to our automation.
These changes have made a material difference to Mozilla's ability to
deliver ``new rapid release cadence'' releases, as well as ``urgent
security fixes'' releases to Firefox users. With this chapter, we've
tried to give a quick overview of how things work right now. We are
constantly improving our processes; a year from now we could probably
write this chapter all over again with a very different set of
strategies and workflow. Hopefully, people find this information
useful. If you have comments/feedback, or details on things you've
done differently that did/didn't work better for you, please contact
us at release\@mozilla.com. We'd love to hear from you.
%% NB: When converting to HTML make the email address read
%% "release [at] mozilla [dot] com" - that is how the authors formatted
%% it originally.
  
\end{aosasect1}

\end{aosachapter}
