\begin{aosachapter}{Processing.js}{s:pjs}{Mike Kamermans}

%% British English spellings used (en-gb)

Originally developed by Ben Fry and Casey Reas, the Processing
programming language started as an open source programming language
(based on Java) to help the electronic arts and visual design
communities learn the basics of computer programming in a visual
context. Offering a highly simplified model for 2D and 3D graphics
compared to most programming languages, it quickly became well suited
for a wide range of activities, from teaching programming through
writing small visualisations to creating multi-wall art installations,
and became able to perform a wide variety of tasks, from simply reading in a
sequence of strings to acting as the de facto IDE for programming and
operating the popular ``Arduino'' open source hardware prototyping
boards. Continuing to gain popularity, Processing has firmly taken
its place as an easy to learn, widely used programming language for
all things visual, and so much more.

The basic Processing program, called a ``sketch'', consists of two
functions: \code{setup} and \code{draw}. The first is the main program
entry point, and can contain any amount of initialization
instructions. After finishing \code{setup}, Processing programs can do
one of two things: 1) call \code{draw}, and
schedule another call to \code{draw} at a fixed
interval upon completion; or 2) call \code{draw}, and
wait for input events from the user. By default, Processing does the
former; calling \code{noLoop} results in the latter. This allows for two modes to
present sketches, namely a fixed framerate graphical environment, or
an interactive, event-based updating graphical environment. In both
cases, user events are monitored and can be handled either in their
own event handlers, or for certain events that set persistent global
values, directly in the \code{draw} function.

Processing.js is a sister project of Processing, designed to bring
it to the web without the need for Java or plugins. It started as an
attempt by John Resig to see if the Processing language could be
ported to the web, by using the---at the time brand new---HTML5
\code{{\textless}canvas{\textgreater}} element as a graphical context,
with a proof of concept library released to the public in
2008. Written with the idea in mind that ``your code should just
work'', Processing.js has been refined over the years to make your
data visualisations, digital art, interactive animations, educational
graphs, video games, etc. work using web standards and without any
plugins. You write code using the Processing language, either in the
Processing IDE or your favourite editor of choice, include it on a web
page using a \code{{\textless}canvas{\textgreater}} element, and
Processing.js does the rest, rendering everything in the
\code{{\textless}canvas{\textgreater}} element and letting users
interact with the graphics in the same way they would with a normal
standalone Processing program.

\begin{aosasect1}{How Does It Work?}

Processing.js is a bit unusual as an open source project, in that the
code base is a single file called \code{processing.js},
containing the code for Processing, the single object that makes up
the entire library. In terms of how the code is structured, we
constantly shuffle things around inside this object as we try to clean
it up a little bit with every release. Its design is relatively
straightforward, and its function can be described in a single
sentence: it rewrites Processing source code into pure JavaScript
source code, and every Processing API function call is mapped to a
corresponding function in the JavaScript Processing object, which
effects the same thing on a \code{{\textless}canvas{\textgreater}}
element as the Processing call would effect on a Java Applet canvas.

For speed, we have two separate code paths for 2D and 3D functions,
and when a sketch is loaded, either one or the other is used for
resolving function wrappers so that we don't add bloat to running
instances. However, in terms of data structures and code flow, knowing
JavaScript means you can read \code{processing.js}, with the possible
exception of the syntax parser.

\begin{aosasect2}{Unifying Java and JavaScript}

Rewriting Processing source code into JavaScript source code means
that you can simply tell the browser to execute the rewritten source,
and if you rewrote it correctly, things just work. But, making sure
the rewrite is correct has taken, and still occasionally takes, quite
a bit of effort. Processing syntax is based on Java, which means that
Processing.js has to essentially transform Java source code into
JavaScript source code. Initially, this was achieved by treating the
Java source code as a string, and iteratively replacing substrings of
Java with their JavaScript equivalents\footnote{For those interested
  in an early incarnation of the parser, it can be found at
  \url{https://github.com/jeresig/processing-js/blob/51d280c516c0530cd9e63531076dfa147406e6b2/processing.js},
  running from line 37 to line 266.}. For a small syntax set, this is
fine, but as time went on and complexity added to complexity, this
approach started to break down. Consequently, the parser was
completely rewritten to build an Abstract Syntax Tree (AST) instead, first
breaking down the Java source code into functional blocks, and then
mapping each of those blocks to their corresponding JavaScript
syntax. The result is that, at the cost of
readability\footnote{Readers are welcome to peruse
  \url{https://github.com/jeresig/processing-js/blob/v1.3.0/processing.js\#L17649},
  up to line 19217.}, Processing.js now effectively contains an
on-the-fly Java to JavaScript transcompiler.

%% QUERY: Should there be leading spaces/indentation for these examples?
%%        What about the {{...}} and @@@ blocks? -- TWB

\begin{verbatim}
    {{ two block comparison: "A Processing sketch and its Processing.js equivalent"}}
    @@@
      void setup() {
        size(200,200);
        noCursor();
        noStroke();
        smooth(); }

      void draw() {
        fill(255,10);
        rect(-1,-1,width+1,height+1);
        float f = frameCount*PI/frameRate;
        float d = 10+abs(60*sin(f));
        fill(0,100,0,50);
        ellipse(mouseX, mouseY, d,d); }
    @@@
      function($p) {
          function setup() {
              $p.size(200, 200);
              $p.noCursor();
              $p.noStroke();
              $p.smooth(); }
          $p.setup = setup;

          function draw() {
              $p.fill(255, 10);
              $p.rect(-1, -1, $p.width + 1, $p.height + 1);
              var f = $p.frameCount * $p.PI / $p.__frameRate;
              var d = 10 + $p.abs(60 * $p.sin(f));
              $p.fill(0, 100, 0, 50);
              $p.ellipse($p.mouseX, $p.mouseY, d, d); }
          $p.draw = draw; }
    @@@
    {{ two block comparison: "A Processing sketch and its Processing.js equivalent"}}
\end{verbatim}

This sounds like a great thing, but there are a few problems when
converting Java syntax to JavaScript syntax:

\begin{aosaenumerate}

\item Java programs are isolated entities. JavaScript programs share
  the world with a web page.

\item Java is strongly typed. JavaScript is not.

\item Java is a class/instance based object-oriented language. JavaScript is not.

\item Java has distinct variables and methods. JavaScript does not.

\item Java allows method overloading. JavaScript does not.

\item Java allows importing compiled code. JavaScript has no idea what
  that even means.

\end{aosaenumerate}

Dealing with these problems has been a tradeoff between what users
need, and what we can do given web technologies. The following
sections will discuss each of these issues in greater detail.

\end{aosasect2}

\end{aosasect1}

\begin{aosasect1}{Significant Differences}

%% QUERY: Should these \emph pseudo-headings be made into
%%        aosasect2 ? Maybe \paragraph ? --TWB

\emph{Java makes you wait for your program to load things. JavaScript can lock up your entire browser.}

Java programs are isolated entities, running in their own thread in
the greater pool of applications on your system. JavaScript programs,
on the other hand, live inside a browser, and compete with each other
in a way that desktop applications don't. When a Java program loads a
file, the program waits until the resource is done loading, and
operation resumes as intended. In a setting where the program is an
isolated entity on its own, this is fine. The operating system stays
responsive because it's responsible for thread scheduling, and even if
the program takes an hour to load all its data, you can still use your
computer. On a web page, this is not how things work. If you have a
JavaScript ``program'' waiting for a resource to be done loading, it
will lock its process until that resource is available. If you're
using a browser that uses one process per tab, it will lock up your tab,
and the rest of the browser is still usable. If you're using a browser
that doesn't, your entire browser will seem frozen. So, regardless of
what the process represents, the page the script runs on won't be
usable until the resource is done loading, and it's entirely possible
that your JavaScript will lock up the entire browser.

This is unacceptable on the modern web, where resources are
transferred asynchronously, and the page is expected to function
normally while resources are loaded in the background. While this is
great for traditional web pages, for web applications this is a real
brain twister: how do you make JavaScript idle, waiting for a resource
to load, when there is no explicit mechanism to make JavaScript idle?
While there is no explicit threading in JavaScript, there is an event
model, and there is an \code{XMLHTTPRequest} object for requesting arbitrary
(not just XML or HTML) data from arbitrary URLS. This object comes
with several different status events, and we can use it to
asynchronously get data while the browser stays responsive. Which is
great in programs in which you control the source code: you make it
simply stop after scheduling the data request, and make it pick up
execution when the data is available. However, this is near impossible
for code that was written based on the idea of synchronous resource
loading. Injecting ``idling'' in programs that are supposed to run at a
fixed framerate is not an option, so we have to come up with
alternative approaches.

For some things, we decided to force synchronous waiting
anyway. Loading a file with strings, for instance, uses a synchronous
\code{XMLHTTPRequest}, and will halt execution of the page until the data is
available. For other things, we had to get creative. Loading images,
for instance, uses the browser's built-in mechanism for loading
images; we build a new \code{Image} in JavaScript, set its \code{src}
attribute to the image URL, and the browser does the rest, notifying
us that the image is ready through the \code{onload} event. This
doesn't even rely on an \code{XMLHTTPRequest}, it simply exploits the
browser's capabilities.

To make matters easier when you already know which images you are
loading, we added preload directives so that the sketch does not start
execution until preloading is complete. A user can indicate any number
of images to preload via a comment block at the start of the sketch;
Processing.js then tracks outstanding image loading. The \code{onload}
event for an image tells us that it is done transferring and is
considered ready to be rendered (rather than simply having been
downloaded but not decoded to a pixel array in memory yet), after
which we can populate the corresponding Processing \code{PImage}
object with the correct values (\code{width}, \code{height}, pixel
data, etc.) and clear the image from the list. Once the list is empty,
the sketch gets executed, and images used during its lifetime will not
require waiting.

\begin{verbatim}
      {{preload directives example}}
      @@@
        /* @pjs preload="./worldmap.jpg"; */

        PImage img;

        void setup() {
          size(640,480);
          noLoop();
          img = loadImage("worldmap.jpg"); }

        void draw() {
          image(img,0,0); }
      @@@
      {{preload directives example}}
\end{verbatim}

For other things, we've had to build more complicated ``wait for me''
systems. Fonts, unlike images, do not have built-in browser loading
(or at least not a system as functional as image loading). While it is
possible to load a font using a CSS \code{@font-face} rule and rely on
the browser to make it all happen, there are no JavaScript events that
can be used to determine that a font finished loading. We are slowly
seeing events getting added to browsers to generate JavaScript events
for font download completion, but these events come ``too early'', as
the browser may need anywhere from a few to a few hundred more
milliseconds to actually parse the font for use on the page after
download. Thus, acting on these events will still lead to either no
font being applied, or the wrong font being applied if there is a
known fallback font. Rather than relying on these events, we embed
a tiny TrueType font that only contains the letter ``A'' with
impossibly small metrics, and instruct the browser to load this font
via an \code{@font-face} rule with a data URI that contains the font's
bytecode as a BASE64 string. This font is so small that we can rely on
it being immediately available. For any other font load
instruction we compare text metrics between the desired font and this
tiny font. A hidden \code{{\textless}div{\textgreater}} is set up with text styled using the desired
font, with our tiny font as fallback. As long as the text in that \code{{\textless}div{\textgreater}}
is impossibly small, we know the desired font is not available yet,
and we simply poll at set intervals until the
text has sensible metrics.

\emph{Java is strongly typed. JavaScript is not.}

In Java, the number 2 and the number 2.0 are different values, and
they will do different things during mathematical operations. For
instance, the code \code{i = 1/2} will result in \code{i} being 0,
because the numbers are treated as integers, whereas \code{i = 1/2.0},
\code{i = 1.0/2}, and even \code{i = 1./2.} will all result in
\code{i} being 0.5, because the numbers are considered decimal
fractions with a non-zero integer part, and a zero fractional part. Even if
the intended data type is a floating point number, if the arithmetic
uses only integers, the result will be an integer. This lets you write
fairly creative math statements in Java, and consequently in
Processing, but these will generate potentially wildly different
results when ported to Processing.js, as JavaScript only knows
``numbers''. As far as JavaScript is concerned, 2 and 2.0 are the same
number, and this can give rise to very interesting bugs when running a
sketch using Processing.js.

This might sound like a big issue, and at first we were convinced it
would be, but you can't argue with real world feedback: it turns out
this is almost never an issue for people who put their sketches online
using Processing.js. Rather than solving this in some cool and
creative way, the resolution of this problem was actually remarkably
straightforward; we didn't solve it, and as a design choice, we don't
intend to ever revisit that decision. Short of adding a symbol table
with strong typing so that we can fake types in JavaScript and switch
functionality based on type, this incompatibility cannot properly be
solved without leaving much harder to find edge case bugs, and so
rather than adding bulk to the code and slowdown to execution, we left
this quirk in. It is a well-documented quirk, and ``good code'' won't
try to take advantage of Java's implicit number type casting. That
said, sometimes you will forget, and the result can be quite
interesting.

\emph{Java is a class/instance-based object-oriented language, with separate variable and method spaces. JavaScript is not.}

JavaScript uses prototype objects, and the inheritance model that
comes with it. This means all objects are essentially key/value pairs
where each key is a string, and values are either primitives, arrays,
objects, or functions. On the inheritance side, prototypes can extend
other prototypes, but there is no real concept of ``superclass'' and
``subclass''. In order to make ``proper'' Java-style object-oriented code
work, we had to implement classical inheritance for JavaScript in
Processing.js, without making it super slow (we think we succeeded in
that respect). We also had to come up with a way to prevent variable
names and function names from stepping on each other. Because of the
key/value nature of JavaScript objects, defining a variable called
\code{line}, followed by a function like \code{line(x1,y1,x2,y2)} will leave
you with an object that uses whatever was declared last for a
key. JavaScript first sets \code{object.line = "some value"} for you, and
then sets \code{object.line = function(x1,y1,x2,y2)\{...\}}, overriding what
you thought your variable \code{line} was.

It would have slowed down the library a lot to create separate
administration for variables and methods/functions, so again the
documentation explains that it's a bad idea to use variables and
functions with the same name. If everyone wrote ``proper'' code, this
wouldn't be much of a problem, as you want to name variables and
functions based on what they're for, or what they do, but the real
world does things differently. Sometimes your code won't work, and
it's because we decided that having your code break due to a naming
conflict is preferable to your code always working, but always being
slow. A second reason for not implementing variable and function
separation was that this could break JavaScript code used inside
Processing sketches. Closures and the scope chain for JavaScript rely
on the key/value nature of objects, so driving a wedge in that by
writing our own administration would have also severely impacted
performance in terms of Just-In-Time compilation and compression based
on functional closures.

\emph{Java allows method overloading. JavaScript does not.}

One of Java's more powerful features is that you can define a
function, let's say \code{add(int,int)}, and then define another function
with the same name, but a different number of arguments, e.g. \code{add(int,int,int)}, or with different argument types, e.g. \code{add(ComplexNumber,ComplexNumber)}. Calling \code{add} with two or three integer arguments will
automatically call the appropriate function, and calling \code{add} with
floats or Car objects will generate an error. JavaScript, on the other
hand, does not support this. In JavaScript, a function is a property,
and you can dereference it (in which case JavaScript will give you a
value based on type coercion, which in this case returns \code{true} when
the property points to a function definition, or \code{false} when it
doesn't), or you can call it as a function using the execution
operators (which you will know as parentheses with zero or more
arguments between them). If you define a function as \code{add(x,y)} and then
call it as \code{add(1,2,3,4,5,6)}, JavaScript is okay with that. It will set \code{x}
to 1 and \code{y} to 2 and simply ignore the rest of the arguments. In order
to make overloading work, we rewrite functions with the same name but
different argument count to a numbered function, so that
\code{function(a,b,c)} in the source becomes \code{function\$3(a,b,c)} in the
rewritten code, and \code{function(a,b,c,d)} becomes \code{function\$4(a,b,c,d)},
ensuring the correct code paths.

We also mostly solved overloading of functions with the same number
but differently typed arguments, as long as the argument types can be
seen as \emph{different} by JavaScript. JavaScript can tell the functional
type of properties using the \code{typeof} operator, which will return
either \code{number}, \code{string}, \code{object} or \code{function} depending on what a
property represents. Declaring \code{var x = 3} followed by \code{x = '6'} will
cause \code{typeof x} to report \code{number} after the initial declaration, and
\code{string} after reassignment. As long as functions with the same
argument count differ in argument type, we rename them and switch
based on the result of the typeof operation. This does not work when
the functions take arguments of type \code{object}, so for these functions
we have an additional check involving the \code{instanceof} operator
(which returns the name of the function that was used to create the
object) to make function overloading work. In fact, the only place
where we cannot successfully transcompile overloaded functions is
where the argument count is the same between functions, and the
argument types are different numerical types. As JavaScript only has
one numerical type, declaring functions such as \code{add(int x, int y)},
\code{add(float x, float y)} and \code{add(double x, double y)} will
clash. Everything else, however, will work just fine.

\emph{Java allows importing compiled code.}

Sometimes, plain Processing is not enough, and additional
functionality is introduced in the form of a Processing library. These
take the form of a \code{.jarchive} with compiled Java code, and offer
things like networking, audio, video, hardware interfacing and other
exotic functions not covered by Processing itself.

This is a problem, because compiled Java code is Java byte code. This
has given us many headaches: how do we support library imports without
writing a Java byte code decompiler? After about a year of
discussions, we settled on what may seem the simplest solution. Rather
than trying to also cover Processing libraries, we decided to support
the import keyword in sketches, and create a Processing.js Library
API, so that library developers can write a JavaScript version of
their library (where feasible, given the web's nature), so that if
they write a package that is used via \code{import processing.video},
native Processing will pick the \code{.jarchive}, and Processing.js
will instead pick processing.video.js, thus ensuring that things
``just work''. This functionality is slated for Processing.js 1.4, and
library imports is the last major feature that is still missing from
Processing.js (we currently support the \code{import} keyword only in the
sense that it is removed from the source code before conversion), and
will be the last major step towards parity.

\emph{Why pick JavaScript if it can't do Java?}

This is not an unreasonable question, and it has multiple answers. The
most obvious one is that JavaScript comes with the browser. You don't
``install'' JavaScript yourself, there's no plugin to download first;
it's just there. If you want to port something to the web, you're
stuck with JavaScript. Although, given the flexibility of JavaScript,
``stuck with'' is really not doing justice to how powerful the language
is. So, one reason to pick JavaScript is ``because it's already
there''. Pretty much every device that is of interest comes with a
JavaScript-capable browser these days. The same cannot be said for
Java, which is being offered less and less as a preinstalled
technology, if it is available at all.

However, the proper answer is that it's not really true that
JavaScript ``can't do'' the things that Java does; it \emph{can}, it would
just be slower. Even though out of the box JavaScript can't do some of
the things Java does, it's still a Turing-complete programming
language and it can be made to emulate any other programming language,
at the cost of speed. We could, technically, write a full Java
interpreter, with a \code{String} heap, separate variable and method models,
class/instance object-orientation with rigid class hierarchies,
and everything else under the Sun (or, these days, Oracle), but that's
not what we're in it for: Processing.js is about offering a
Processing-to-the-web conversion, in as little code as is necessary
for that. This means that even though we decided not to make it do
certain Java things, our library has one huge benefit: it can cope
with embedded JavaScript really, really well.

In fact, during a meeting between the Processing.js and Processing
people at Bocoup in Boston, in 2010, Ben Fry asked John Resig why he
used regular expression replacement and only partial conversion
instead of doing a proper parser and compiler. John's response was
that it was important to him that people be able to mix Processing
syntax (Java) and JavaScript without having to choose between
them. That initial choice has been crucial in shaping the philosophy
of Processing.js ever since. We've worked hard to keep it true in our
code, and we can see a clear payoff when we look at all the ``purely
web'' users of Processing.js, who never used Processing, and will
happily mix Processing and JavaScript syntax with things working just
fine.

\begin{verbatim}
    {{an example of JavaScript and Processing working together}}
    @@@
      // JavaScript (would throw an error in native Processing)
      var cs = { x: 50,
                 y: 0,
                 label: "my label",
                 rotate: function(theta) {
                           var nx = this.x*cos(theta) - this.y*sin(theta);
                           var ny = this.x*sin(theta) + this.y*cos(theta);
                           this.x = nx; this.y = ny; }};

      // Processing
      float angle = 0;

      void setup() {
        size(200,200);
        strokeWeight(15); }

      void draw() {
        translate(width/2,height/2);
        angle += PI/frameRate;
        while(angle>2*PI) { angle-=2*PI; }
        jQuery('#log').text(angle); // JavaScript (error in native Processing)
        cs.rotate(angle);           // legal JavaScript as well as Processing
        stroke(random(255));
        point(cs.x, cs.y); }
    @@@
    {{an example of JavaScript and Processing working together}}
\end{verbatim}

A lot of things in Java are promises: strong typing is a content
promise to the compiler, visibility is a promise on who will call
methods and reference variables, interfaces are promises that
instances contain the methods the interface describes, etc. Break
those promises and the compiler complains. But, if you don't---and
this is a one of the most important thoughts for Processing.js---then
you don't need the additional code for those promises in order for a
program to work. If you stick a number in a variable, and your code
treats that variable as if it has a number in it, then at the end of
the day \code{var varname} is just as good as \code{int varname}. Do you need
typing? In Java, you do; in JavaScript, you don't, so why force it in?
The same goes for other code promises. If the Processing compiler
doesn't complain about your code, then we can strip all the explicit
syntax for your promises and it'll still work the same.

This has made Processing.js a ridiculously useful library for data
visualisation, media presentation and even entertainment. Sketches in
native Processing work, but sketches that mix Java and JavaScript also
work just fine, as do sketches that use pure JavaScript by treating
Processing.js as a glorified canvas drawing framework. In an effort to
reach parity with native Processing, without forcing Java-only syntax,
the project has been taken in by an audience as wide as the web
itself. We've seen activity all over the web using
Processing.js. Everyone from IBM to Google has built visualisations,
presentations and even games with Processing.js---Processing.js is
making a difference.

Another great thing about converting Java syntax to JavaScript while
leaving JavaScript untouched is that we've enabled something we
hadn't even thought about ourselves: Processing.js will work with
anything that will work with JavaScript. One of the really interesting
things that we're now seeing, for instance, is that people are using
CoffeeScript (a wonderfully simple, Ruby-like programming language
that transcompiles to JavaScript) in combination with Processing.js,
with really cool results. Even though we set out to build ``Processing
for the web'' based on parsing Processing syntax, people took what we
did and used it with brand new syntaxes. They could never have done
that if we had made Processing.js simply be a Java interpreter. By
sticking with code conversion rather than writing a code interpreter,
Processing.js has given Processing a reach on the web far beyond what
it would have had if it had stayed Java-only, or even if it had kept a
Java-only syntax, with execution on the web taken care of by
JavaScript. The uptake of our code not just by end users, but also by
people who try to integrate it with their own technologies, has been
both amazing and inspiring. Clearly we're doing something right, and
the web seems happy with what we're doing.

\begin{aosasect2}{The Result}

As we are coming up to Processing.js 1.4.0, our work has resulted in a
library that will run any sketch you give it, provided it does not
rely on compiled Java library imports. If you can write it in
Processing, and it runs, you can put it on a webpage and it will just
run. Due to the differences in hardware access and low level
implementations of different parts of the rendering pipeline there
will be timing differences, but in general a sketch that runs at 60
frames per seconds in the Processing IDE will run at 60 frames per
second on a modern computer, with a modern browser. We have reached a
point where bug reports have started to die down, and most work is no
longer about adding feature support, but more about bug fixing and
code optimization.

Thanks to the efforts of many developers working to resolve over 1800
bug reports, Processing sketches run using Processing.js ``just
work''. Even sketches that rely on library imports can be made to work,
provided that the library code is at hand. Under favourable circumstances,
the library is written in a way that lets you rewrite it to pure
Processing code with a few search-replace operations. In this case the
code can be made to work online virtually immediately. When the
library does things that cannot be implemented in pure Processing, but
can be implemented using plain JavaScript, more work is required to
effectively emulate the library using JavaScript code, but porting is
still possible. The only instances of Processing code that cannot be
ported are those that rely on functionality that is inherently
unavailable to browsers, such as interfacing directly with hardware
devices (such as webcams or Arduino boards) or performing unattended
disk writes, though even this is changing. Browsers are constantly adding
functionality to allow for more elaborate applications, and limiting
factors today may disappear a year from now, so that hopefully in the not
too distant future, even sketches that are currently impossible to run
online will become portable.

\end{aosasect2}

\end{aosasect1}

\begin{aosasect1}{The Code Components}

Processing.js is presented and developed as a large, single file, but
architecturally it represents three different components: 1) the
launcher, responsible for converting Processing source to
Processing.js flavoured JavaScript and executing it, 2) static
functionality that can be used by all sketches, and 3) sketch
functionality that has to be tied to individual instances.

\begin{aosasect2}{The Launcher}

The launcher component takes care of three things: code preprocessing,
code conversion, and sketch execution.

\begin{aosasect3}{Preprocessing}

In the preprocessing step, Processing.js directives are split off from
the code, and acted upon. These directives come in two flavours:
settings and load instructions. There is a small number of directives,
keeping with the ``it should just work'' philosophy, and the only
settings that sketch authors can change are related to page
interaction. By default a sketch will keep running if the page is not
in focus, but the \code{pauseOnBlur\,=\,true} directive sets up a sketch in
such a way that it will halt execution when the page the sketch is
running on is not in focus, resuming execution when the page is in
focus again. Also by default, keyboard input is only routed to a
sketch when it is focussed. This is especially important when people
run multiple sketches on the same page, as keyboard input intended for
one sketch should not be processed by another. However, this
functionality can be disabled, routing keyboard events to every sketch
that is running on a page, using the \code{globalKeyEvents\,=\,true} directive.

Load instructions take the form of the aforementioned image preloading
and font preloading. Because images and fonts can be used by multiple
sketches, they are loaded and tracked globally, so that different
sketches don't attempt multiple loads for the same resource.

\end{aosasect3}

\begin{aosasect3}{Code Conversion}

The code conversion component decomposes the source code into AST
nodes, such as statements and expressions, methods, variables,
classes, etc. This AST then expanded to JavaScript source code that
builds a sketch-equivalent program when executed. This converted
source code makes heavy use of the Processing.js instance framework
for setting up class relations, where classes in the Processing source
code become JavaScript prototypes with special functions for
determining superclasses and bindings for superclass functions and
variables.

\end{aosasect3}

\begin{aosasect3}{Sketch Execution}

The final step in the launch process is sketch execution, which
consists of determining whether or not all preloading has finished,
and if it has, adding the sketch to the list of running instances
and triggering its JavaScript \code{onLoad} event so that any sketch
listeners can take the appropriate action. After this the Processing
chain is run through: \code{setup}, then \code{draw}, and if the
sketch is a looping sketch, setting up an interval call to \code{draw}
with an interval length that gets closest to the desired framerate for
the sketch.

\end{aosasect3}

\end{aosasect2}

\begin{aosasect2}{Static Library}

Much of Processing.js falls under the ``static library'' heading,
representing constants, universal functions, and universal data
types. A lot of these actually do double duty, being defined as global
properties, but also getting aliassed by instances for quicker code
paths. Global constants such as key codes and color mappings are
housed in the Processing object itself, set up once, and then
referenced when instances are built via the Processing
constructor. The same applies to self-contained helper functions,
which lets us keep the code as close to ``write once, run anywhere'' as
we can without sacrificing performance.

Processing.js has to support a large number of complex data types, not
just in order to support the data types used in Processing, but also
for its internal workings. These, too, are defined in the Processing
constructor:

\begin{aosaitemize}

%% QUERY Should these be \emph, \code, or perhaps make this an
%%   definitions list rather than an itemized list? -- TWB

\item Char, an internal object used to overcome some of the
  behavioural quirks of Java's \code{char} datatype.

\item PShape, which represents shape objects.

\item PShapeSVG, an extension for PShape objects, which is built from and
  represents SVG XML.

For PShapeSVG, we implemented our own
SVG-to-\code{{\textless}canvas{\textgreater}}-instructions code. Since
Processing does not implement full SVG support, the code we saved by
not relying on an external SVG library means that we can account for
every line of code relating to SVG imports. It only parses what it has
to, and doesn't waste space with code that follows the spec, but is
unused because native Processing does not support it.

\item XMLElement, an XML document object.

For XMLElement, too, we implemented our own code, relying on the
browser to first load the XML element into a Node-based structure,
then traveling the node structure to build a leaner object. Again,
this means we don't have any dead code sitting in Processing.js,
taking up space and potentially causing bugs because a patch
accidentally makes use of a function that shouldn't be there.

\item PMatrix2D and PMatrix3D, which performs matrix operations in 2D and 3D
  mode.

\item PImage, which represents an image resource.

This is effectively a wrapper of the Image object, with some
additional functions and properties so that its API matches the
Processing API.

\item PFont, which represents a font resource.

There is no Font object defined for JavaScript (at least for now), so
rather than actually storing the font as an object, our PFont
implementation loads a font via the browser, computes its metrics
based on how the browser renders text with it, and then caches the
resultant PFont object. For speed, PFonts have a reference to the
canvas that was used to determine the font properties, in case
\code{textWidth} must be calculated, but because we track PFont objects based
on name/size pair, if a sketch uses a lot of distinct text sizes, or
fonts in general, this will consume too much memory. As such, PFonts
will clear their cached canvas and instead call a generic \code{textWidth}
computation function when the cache grows too large. As a secondary
memory preservation strategy, if the font cache continues to grow
after clearing the cached canvas for each PFont, font caching is
disabled entirely, and font changes in the sketch simply build new
throwaway PFont objects for every change in font name, text size or
text leading.

\item DrawingShared, Drawing2D, and Drawing3D, which house all the
  graphics functions.

The DrawingShared object is actually the biggest speed trap in
Processing.js. It determines if a sketch is
launching in 2D or 3D mode, and then rebinds all graphics functions
to either the Drawing2D or Drawing3D object. This ensures short code
path for graphics instructions, as 2D Processing sketches cannot used
3D functions, and vice versa. By only binding one of the two sets of
graphics functions, we gain speed from not having to switch on the
graphics mode in every function to determine the code path, and we
save space by not binding the graphics functions that are guaranteed
not to be used.

\item ArrayList, a container that emulates Java's \code{ArrayList}.

\item HashMap, a container that emulates Java's \code{HashMap}.

ArrayList, and HashMap in particular, are special data structures
because of how Java implements them. These containers rely on the Java
concepts of equality and hashing, and all objects in Java have an
\code{equals} and a \code{hashCode} method that allow them to be
stored in lists and maps.

For non-hashing containers, objects are resolved based on equality
rather than identity. Thus, \code{list.remove(myobject)} iterates through the
list looking for an element for which \code{element.equals(myobject)}, rather
than \code{element\,==\,myobject}, is true. Because all objects must have an
\code{equals} method, we implemented a ``virtual equals'' function on
the JavaScript side of things. This function takes two objects as
arguments, checks whether either of them implements their own
\code{equals} function, and if so, falls through to that function.
If they don't, and the passed objects are primitives,
primitive equality is checked. If they're not, then there is no
equality.

For hashing containers, things are even more interesting, as hashing
containers act as shortcut trees. The container actually wraps a
variable number of lists, each tied to a specific hash code. Objects
are found based on first finding the container that matches their hash
code, in which the object is then searched for based on equality
evaluation. As all objects in Java have a \code{hashCode} method, we also
wrote a ``virtual hashcode'' function, which takes a single object as
an argument. The function checks whether the object implements its own
\code{hashCode} function, and if so falls through to that function.
If it doesn't, the hash code is computed based on the same
hashing algorithm that is used in Java.

\end{aosaitemize}

\end{aosasect2}

\begin{aosasect2}{Administration}

The final piece of functionality in the static code library is the
instance list of all sketches that are currently running on the
page. This instance list stores sketches based on the canvas they have
been loaded in, so that users can call
Processing.getInstanceById('canvasid') and get a reference to their
sketch for page interaction purposes.

\begin{aosasect3}{Instance Code}

Instance code takes the form of \code{p.functor = function(arg, ...)}
definitions for the Processing API, and \code{p.constant = ...} for sketch
state variables (where \code{p} is our reference to the sketch being set
up). Neither of these are located in dedicated code blocks. Rather,
the code is organized based on function, so that instance code
relating to PShape operations is defined near the PShape object, and
instance code for graphics functions are defined near, or in, the
Drawing2D and Drawing3D objects.

In order to keep things fast, a lot of code that could be written as
static code with an instance wrapper is actually implemented as purely
instance code. For instance, the \code{lerpColor(c1,\,c2,\,ratio)} function,
which determines the color corresponding to the linear interpolation
of two colors, is defined as an instance function. Rather than having
\code{p.lerpColor(c1,\,c2,\,ratio)} acting as a wrapper for some static function
\code{Processing.lerpColor(c1,\,c2,\,ratio)}, the fact that nothing else in
Processing.js relies on \code{lerpColor} means that code execution is
faster if we write it as a pure instance function. While this does
``bloat'' the instance object, most functions for which we insist on an
instance function rather than a wrapper to the static library are
small. Thus, at the expense of memory we create really fast code
paths. While the full Processing object will take up a one-time memory
slice worth around 5 MB when initially set up, the prerequisite code
for individual sketches only takes up about 500 KB.

\end{aosasect3}

\end{aosasect2}

\end{aosasect1}

\begin{aosasect1}{Developing Processing.js}

Processing.js is worked on intensively, which we can only do because
our development approach sticks to a few basic rules. As these rules
influence the architecture of Processing.js, it's worth having a brief
look at them before closing this chapter.

\begin{aosasect2}{Make It Work}

Writing code that works sounds like a tautological premise; you write
code, and by the time you're done your code either works, because
that's what you set out to do, or it doesn't, and you're not done
yet. However, ``make it work'' comes with a corollary:

``Make it work. And when you're done, prove it.''

If there is one thing above all other things that has allowed
Processing.js to grow at the pace it has, it is the presence of
tests. Any ticket that requires touching the code, be it either by
writing new code or rewriting old code, cannot be marked as resolved
until there is a unit or reference test that allows others to verify
not only that the code works the way it should, but also that it
breaks when it should. For most code, this typically involves a unit
test---a short bit of code that calls a function and simply tests
whether the function returns the correct values, for both legal and
illegal function calls. Not only does this allow us to test code
contributions, it also lets us perform regression tests.

Before any code is accepted and merged into our stable development
branch, the modified Processing.js library is validated against an
ever-growing battery of unit tests. Big fixes and performance tests in
particular are prone to pass their own unit tests, but may end up
breaking parts that worked fine before the rewrite. Having tests for
every function in the API, as well as internal functions, means that
as Processing.js grows, we don't accidentally break compatibility with
previous versions. Barring destructive API changes, if none of the
tests failed before a code contribution or modification, none of the
tests are allowed to fail with the new code in.

\begin{verbatim}
    {{ example of a unit test: verifying inline object creation }}
    @@@
      interface I {
        int getX();
        void test(); }

      I i = new I() {
        int x = 5;
        public int getX() {
          return x; }
        public void test() {
          x++; }};

      i.test();

      _checkEqual(i.getX(), 6);
      _checkEqual(i instanceof I, true);
      _checkEqual(i instanceof Object, true);
    @@@
    {{ example of a unit test: verifying inline object creation }}
\end{verbatim}

In addition to regular code unit tests, we also have visual reference
(or ``ref'') tests. As Processing.js is a port of a visual programming
language, some tests cannot be performed using just unit
tests. Testing to see whether an ellipse gets drawn on the correct
pixels, or whether a single-pixel-wide vertical line is drawn crisp or
smoothed cannot be determined without a visual reference. Because all
mainstream browsers implement the
\code{{\textless}canvas{\textgreater}} element and Canvas2D API with
subtle differences, these things can only be tested by running code in
a browser and verifying that the resulting sketch looks the same as
what native Processing generates. To make life easier for developers,
we use an automated test suite for this, where new test cases are run
through Processing, generating ``what it should look like'' data to be
used for pixel comparison. This data is then stored as a comment
inside the sketch that generated it, forming a test, and these tests
are then run by Processing.js on a visual reference test page which
executes each test and performs pixel comparisons between ``what it
should look like'' and ``what it looks like''. If the pixels are off, the
test fails, and the developer is presented with three images:
what it should look like, how Processing.js
rendered it, and the difference between the two, marking
problem areas as red pixels, and correct areas as white. Much like
unit tests, these tests must pass before any code contribution can be
accepted.

\end{aosasect2}

\begin{aosasect2}{Make It Fast}

In an open source project, making things work is only the first step
in the life of a function. Once things work, you want to make sure
things work fast. Based on the ``if you can't measure it, you can't
improve it'' principle, most functions in Processing.js don't just come
with unit or ref tests, but also with performance (or ``perf'')
tests. Small bits of code that simply call a function, without testing
the correctness of the function, are run several hundred times in a
row, and their run time is recorded on a special performance test web
page. This lets us quantify how well (or not!) Processing.js performs
in browsers that support HTML5's
\code{{\textless}canvas{\textgreater}} element. Every time an
optimization patch passes unit and ref testing, it is run through our
performance test page. JavaScript is a curious beast, and beautiful
code can, in fact, run several orders of magnitude slower than code
that contains the same lines several times over, with inline code
rather than function calls. This makes performance testing crucial. We
have been able to speed up certain parts of the library by three
orders of magnitude simply by discovering hot loops during perf
testing, reducing the number of function calls by inlining code, and
by making functions return the moment they know what their return
value should be, rather than having only a single return at the very
end of the function.

Another way in which we try to make Processing.js fast is by looking
at what runs it. As Processing.js is highly dependent on the
efficiency of JavaScript engines, it makes sense to also look at which
features various engines offer to speed things up. Especially now that
browsers are starting to support hardware accelerated graphics,
instant speed boosts are possible when engines offer new and more
efficient data types and functions to perform the low level operations that
Processing.js depends on. For instance, JavaScript technically has no
static typing, but graphics hardware programming environments do. By
exposing the data structures used to talk to the hardware directly to
JavaScript, it is possible to significantly speed up sections of code
if we know that they will only use specific values.

\end{aosasect2}

\begin{aosasect2}{Make It Small}

There are two ways to make code small. First, write compact code. If
you're manipulating a variable multiple times, compact it to a single
manipulation (if possible). If you access an object variable multiple
times, cache it. If you call a function multiple times, cache the
result. Return once you have all the information you need, and
generally apply all the tricks a code optimiser would apply
yourself. JavaScript is a particularly nice language for this, since
it comes with an incredible amount of flexibility. For example, rather
than using:

\begin{verbatim}
if ((result = functionresult)!==null) {
  var = result;
} else {
  var = default;
}
\end{verbatim}

\noindent in JavaScript this becomes:

\begin{verbatim}
var = functionresult || default
\end{verbatim}

There is also another form of small code, and that's in terms of
runtime code. Because JavaScript lets you change function bindings on
the fly, running code becomes much smaller if you can say ``bind the
function for line2D to the function call for \code{line}'' once you
know that a program runs in 2D rather than 3D mode, so that you don't
have to perform:

\begin{verbatim}
if(mode==2D) { line2D() } else { line3D() }
\end{verbatim}

\noindent for every function call that might be either in 2D or 3D mode.

Finally, there is the process of minification. There are a number of
good systems that let you compress your JavaScript code by
renaming variables, stripping whitespace, and applying certain code
optimisations that are hard to do by hand while still keeping the code
readable. Examples of these are the YUI minifier and Google's closure
compiler. We use these technologies in Processing.js to offer end
users bandwidth convenience---minification after stripping comments
can shrink the library by as much as 50\%, and taking advantage of
modern browser/server interaction for gzipped content, we can offer
the entire Processing.js library in gzipped form in 65 KB.

\end{aosasect2}

\begin{aosasect2}{If All Else Fails, Tell People}

Not everything that can currently be done in Processing can be done in
the browser. Security models prevent certain things like saving files
to the hard disk and performing USB or serial port I/O, and a lack of
typing in JavaScript can have unexpected consequences (such as all
math being floating point math). Sometimes we're faced with the choice
between adding an incredible amount of code to enable an edge case, or
mark the ticket as a ``wontfix'' issue. In such cases, a new ticket gets
filed, typically titled ``Add documentation that explains why...''.

In order to make sure these things aren't lost, we have documentation
for people who start using Processing.js with a Processing
background, and for people who start using Processing.js with a JavaScript
background, covering the differences between what is expected, and
what actually happens. Certain things just deserve special mention,
because no matter how much work we put into Processing.js, there are
certain things we cannot add without sacrificing usability. A good
architecture doesn't just cover the way things are, it also covers
why; without that, you'll just end up having the same discussions
about what the code looks like and whether it should be different
every time the team changes.

\end{aosasect2}

\end{aosasect1}

\begin{aosasect1}{Lessons Learned}

The most important lesson we learned while writing Processing.js is
that when porting a language, what matters is that the result is
correct, not whether or not the code used in your port is similar to
the original. Even though Java and JavaScript syntax are fairly similar,
and modifying Java code to legal JavaScript code is
fairly easy, it often pays to look at what JavaScript can natively
do and exploit that to get the same functional result. Taking
advantage of the lack of typing by recycling variables, using certain
built-in functions that are fast in JavaScript but slow in Java, or
avoiding patterns that are fast in Java but slow in JavaScript means
your code may look radically different, but has the exact same
effect. You often hear people say not to reinvent the wheel, but that
only applies to working with a single programming language. When
you're porting, reinvent as many wheels as you need to obtain the
performance you require.

Another important lesson is to return early, return often, and branch
as little as possible. An if/then statement followed by a return can
be made (sometimes drastically) faster by using an if-return/return
construction instead, using the return statement as a conditional
shortcut. While it's conceptually pretty to aggregate your entire
function state before calling the ultimate return statement for that
function, it also means your code path may traverse code that is
entirely unrelated to what you will be returning. Don't waste cycles;
return when you have all the information you need.

A third lesson concerns testing your code. In Processing.js we had the
benefit of starting with very good documentation outlining how
Processing was ``supposed'' to work, and a large set of test cases, most
of which started out as ``known fail''. This allowed us to do two things:
1) write code against tests, and 2) create tests before writing
code. The usual process, in which code is written and then test
cases are written for that code, actually creates biased tests. Rather than
testing whether or not your code does what it should do, according to
the specification, you are only testing whether your code is
bug-free. In Processing.js, we instead start by creating test cases
based on what the functional requirements for some function or set of
functions is, based on the documentation for it. With these unbiased
tests, we can then write code that is functionally complete, rather
than simply bug-free but possibly deficient.

The last lesson is also the most general one: apply the rules of agile
development to individual fixes as well. No one benefits from you
retreating into dev mode and not being heard from for three days
straight while you write the perfect solution. Rather, get your
solutions to the point where they work, and not even necessarily for
all test cases, then ask for feedback. Working alone, with a test
suite for catching errors, is no guarantee for good or complete
code. No amount of automated testing is going to point out that you
forgot to write tests for certain edge cases, or that there is a
better algorithm than the one you picked, or that you could have
reordered your statements to make the code better suited for JIT
compilation. Treat fixes like releases: present fixes early, update
often, and work feedback into your improvements.

\end{aosasect1}

\end{aosachapter}
