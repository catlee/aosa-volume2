\begin{aosachapter}{GDB}{s:gdb}{Stan Shebs}

GDB, the GNU Debugger, was among the first programs to be written for
the Free Software Foundation, and it has been a staple of GNU systems
ever since.  Originally designed as a plain Unix-type debugger, it has
since been expanded to a wide range of uses, including with many
embedded systems, and has grown by an order of magnitude.

This chapter will retrace the development of GDB, showing how the
internal structure derives from new demands, and new features added
over time.

Scope

GDB is designed to be a symbolic debugger for programs written in
compiled imperative languages such as C, C++, Ada, Fortran and so
forth.  As usual for an unaptly-named ``debugger'', it doesn't
actually remove bugs from programs; instead, it helps the developer
study the program's behavior closely, hopefully in enough detail to
find the bugs and figure out how to fix them.  The most common usage
looks something like this:

\begin{verbatim}
% gdb myprog
[...]
(gdb) break buggy_function
Breakpoint 1 ...
(gdb) run
Stopped at buggy_function
(gdb) print variable_that_should_be_positive
$$1 = -34
(gdb)
\end{verbatim}

So GDB shows something that is not right, and then it is up to the developer
to figure out why.

The important point for design is that GDB is an interactive toolbox
for poking around in a program, and as such it should be responsive
to an unpredictable series of requests.  It also needs to deal with
programs that have been optimized by the compiler, and programs that
exploit every hardware option for performance, so it needs to have
detailed knowledge down to the lowest levels of a system.

History

GDB is an old program.  It came into existence around 1985-6, written
by Richard Stallman along with GCC, GNU Emacs, and other early
components of GNU.  (In those days, there were no public source
control repositories, and much of the detailed development history
has simply been lost.)

Block Diagram

(big picture)

GDB can be said to have two sides to it:

1) The ``symbol side'' is concerned with symbolic information about
the program.  Symbolic information includes function and variable
names and types, line numbers, machine register usage, and so forth.
The symbol side reads symbolic information from the file, parses
expressions, finds the memory address of a given line number, and so
forth.

2) The ``target side'' is concerned with the manipulation of the
target system.  It has facilities to start and stop the program, to
read memory and registers, to modify them, to catch signals, and so
on.

The two sides are somewhat independent of each other; you can look
around your program's machine code, display variable types, etc,
without having anything running.  Conversely, it is possible to do
pure machine-language debugging even if no symbols are avaiable.

In the middle, tying the two sides together, is the command interpreter
and the main execution control loop.

To take a simple case of how it all ties together, consider the
print command from above.  The command interpreter find the printing
code, which parses the expression into a simple tree structure and
then evaluates it by walking the tree.  At some point the evaluator
will consult the symbol table to find out that variable\_... is a global
variable that is stored at memory address 0x123456.  It then uses a
target side routine to read the contents of memory at that address,
which is then handed to a print function that does the formatting.

A more complicated example is single-stepping, in which the simple
command ``step'' conceals the complicated dance going on behind the
scenes.  When the user asks to single-step to the next line in the
program, the target side is asked to execute only a single instruction
of the program and then stop it again (most systems have this
capability built into hardware and/or the operating system).  Upon
being informed that the program has stopped, GDB asks for the program
counter (target side) and then compares it with the range of addresses
that the symbol side says is associated with the current line.  It the
PC is outside that range, then GDB leaves the program stopped, figures
out the new source line, and reports that to the user.  If the PC is
still in the range, then GDB single-steps again and checks again until
the PC gets to a different line.  This basic algorithm has the
advantage that it always does the right thing, whether the line has
jumps, subroutine calls, etc, and does not require interpretation of
each instruction.  A disadvantage is that there is a lot of interaction
with the target for each single-step, which for some embedded targets
results in slow single-stepping.

Portability

From the beginning, and as a program needing extensive access to the
lowest levels, GDB was designed to be portable across a variety of
systems.  However, its portability strategy has changed considerably
over the years.

Originally, GDB started out similar to the other GNU programs of the
time; code in a common subset of C (no GCC features for instance), and
use a combination of preprocessor macros and makefile fragments to
adapt to a specific architecture and operating system.  Although the
stated goal was a self-contained ``GNU operating system'',
bootstrapping would have to be done on a variety of systems; the Linux
kernel was still several years in the future.  The ``configure'' script
was the first step of the process, adding links and constructing files,
in particular the makefile used for the actual build.

As GDB developed, we eventually came to understand that there were three
distinct classes of portability bits to consider.

* ''Host'' definitions are for the machine that GDB is running on, and
might include things like the size of integer types.  Nearly all of these
are easily discovered just by automated testing of the host, and the
formerly-handwritten host definition files have been superseded by
an ''autoconf''-generated configure script.

* ''Target'' definitions are specific to the machine that the program
being debugged is running on.  If same as the host, then we are doing
''native'' debugging, otherwise it is ''cross'' debugging, using some
kind of wire connecting the two systems.  Target definitions fall into
two main classes:

** ''Architecture'' definitions - the machine language byte(s) for a
trap instructions, how to disassemble machine code, how to walk
through the call stack.  These are currently handled by ''gdbarch''
objects, of which more later.

** ''Native'' definitions - these define the specifics of arguments to
ptrace() (which vary considerably between flavors of Unix), how to
find shared libraries that have been loaded, and so forth, which only
apply to the native debugging case.  Native definitions are the last
holdout of old-style C macros, although most are now figured out using
autoconf.

Symbol Side

The symbol side starts by reading the compiled program and extracting
any symbolic information it finds.

The process starts with BFD, which is the common binary file reading
library shared with the GNU assembler, linker, and other minor tools
(the ``binutils'').  BFD offers a single API to read the original Unix
a.out format, COFF (used on System V Unix and MS Windows), and ELF
(modern Unix, GNU/Linux, and most embedded systems), among others.

BFD just pulls blocks of data from the program into GDB's memory, then
for each format, GDB has two levels of reader.  The first is for basic
symbols, which is just the names that the linker uses to build
executables and library; so these are names with addresses and not
much else.  The second level is detailed symbolic information, which
may be in its own format; for instance, DWARF information is
encapsulated in an ELF file.

The code for reading symbolic information is somewhat tedious and
uninteresting; the different format encode the same information in
their own ways, so it's just a matter of interpretation and construction
of the appropriate GDB structures.

Partial Symbol Tables

However, for a program of significant size (such as Emacs or Firefox),
construction of the symbol table can take quite a while, maybe even
several minutes.  Measurements show that the time is *not* in file
reading, but in the in-memory construction of GDB symbols; there are
literally millions of small interconnected objects, and the time adds
up.

Most of the symbolic information will never be looked at in a session,
since it is local to functions that are not of debugging interest.  So
when GDB first pulls in a program, it scans through the symbolic
information, looking for just the globally-visible symbols and
recording only them in the symbol table.  Complete symbolic info for a
function is filled in only when the user stops inside it.

Partial symbol tables allow GDB to start up in only a few seconds, even
for large programs.

Language Support

Source language support mainly consists of expression parsing and value
printing.

(describe expressions and values earlier?)

The details of expression parsing are left up to each language, but in
the general the parser is based on a yacc grammar fed by a handwritten
lexer.  In keeping with GDB's goal of providing more flexibility to the
interactive user, the parser is not expected to be especially stringent;
for instance, if it can guess at a reasonable type for an expression,
it will simply assume that type, rather than require the user to add
a cast or type conversion.

Also, since the parser need not handle statements or type declarations,
it is considerably simpler than the full language grammar.

Similarly for printing; there are just a handful of types of values that
need to be displayed, and oftentimes the language-specific printer can
call out to generic function to finish the job.

Target Side

Target Vectors and the Target Stack

Originally the target side of GDB was composed of a handful of
platform-specific files that handled the details of calling ptrace(),
launching executables, and so forth.

This is not sufficiently flexible for long-running debugging session,
in which the user might switch from local to remote debugging, switch
from files to live programs, attach and detach, and so forth.

So in 1990 John Gilmore redesigned the target side of GDB to send all
platform-specific operation through the ''target vector'', which is
basically a class of objects, each of what defines the the specifics
of a type of target.  In practice, each target vector is a structure
of several dozen function pointers.  Methods ranging from reading of
memory and registers, to setting parameters of a tracing run.

In addition, it is often useful to blend methods from several target
vectors.  For instance, consider the printing of an initialized global
variable on Unix; before the program is running, the bytes should come
from the executable's .data section, and in fact there is a target
vector for executable programs for which the method reads from the
file.  But while the program is running, the bytes should instead come
from the process's address space.  So GDB has a ''target stack'' where
the live process target vector gets pushed on top of the executable's
target vector when the process starts running, and is then popped when
it exits.

(picture?)

In practice, the target vector has become a key abstraction in GDB,
while the target stack has not worked so well.  The reality of the
target stack is that target vectors are not really orthogonal to each
other; it never makes sense to have a exec,process,exec,corefile
combination for example, and it would be very confusing if the user
could get into that situation.  So GDB has a notion of ``stratum'' in
which process-like target vectors are all one level, while file-like
target vectors get assigned to a lower level.

While GDBers grumble about the target stack regularly, nobody has yet
come up with (and implemented!) a better replacement.

Gdbarch

As disassembly is a generic feature also needed for objdump in the
binutils, it was separated into its own library early on.

The general definitions for an architecture gradually became more and
more complicated, as chip designers introduced dozens of extensions
and variants on each basic architecture, and as compiler designers
introduced new kinds of architecture-specific optimizations.  Worse,
GDB has to look at both correct and incorrect state (after all, the
program has bugs), and architecture-specific code often needs
heuristic analysis code to make sense of messes like
partially-overwritten call stacks.

The original macro definition scheme was simply unequal to the task,
and worse, multiple-architecture designs were on the horizon, for
which macros would not work at all.  I proposed changing to an
object-based design in 1995, and starting in 1998 Cygnus Solutions
funded Andrew Cagney to start the changeover.  It took several years,
and contributions from dozens of hackers to finish the job, affecting
perhaps 80,000 lines of code.

The introduced objects are called ''gdbarch'' objects, and at this
point are now up to 130 methods and variables available to define a
target architecture.  On the average, most aren't needed, and a
simple target might only need a dozen or so methods.

Execution control

The heart of GDB is its execution control loop.  We touched on it earlier
when describing single-stepping over a line - the process entailed looping
over multiple instructions until finding one associated with a different
source line.

The loop is called ''wait\_for\_inferior'', or ``wfi'' for short.

Conceptually it is inside the main command loop, and is only entered for
commands that cause the program to resume execution.  When the user types
``continue'' or ``step'' and then waits while nothing seems to be happening,
GDB may in fact be quite busy.  In addition to the stepping loop mentioned
above, the program may be hitting trap instructions and reporting the exception
to GDB.  If the exception is due to the trap being a breakpoint inserted by
GDB, it then tests the breakpoint's condition, and if false, it removes the
trap, single-steps the original instruction, re-inserts the trap, and then
lets the program resume.  Similarly, if a signal is raised, GDB may choose
to ignore it, or handle it one of several ways specified in advance.

All of this activity is managed by wait\_for\_inferior.  Originally this
was a simple loop, waiting for the target to stop and then deciding
what to do about it, but as ports to unusual systems needed special
handling, it grew to a thousand lines, with gotos criss-crossing it
for poorly-understood reasons.  It was also a problem for any kind of
asynchronous handling or debugging of threaded programs, in which the
user wants to interact with GDB while some or all of the program is
still running.

The conversion to an event-oriented model took several years.  I broke
up wait\_for\_inferior in 1999, introducing an execution control state
structure to replace the pile of local and global variables, and
converting the tangle of jumps into smaller independent functions.  At
the same time Elena Zannoni and others introduced event queues that
included both input from the user and notifications from the inferior.

\end{aosachapter}
