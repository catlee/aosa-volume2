\begin{aosachapter}{OSCAR}{s:oscar}{Jennifer Ruttan}

Since their initial adoption, EMR (electronic medical record) systems
have attempted to bridge the gap between the physical and digital
worlds of patient care. Governments in countries around the world have
attempted to come up with a solution that enables better care for
patients at a lower cost while reducing the paper trail that medicine
typically generates. Many countries have been very successful in their
attempts to create such a system---some, like the Canadian province of
Ontario, have not (some may remember the so-called ``eHealth Scandal''
in Ontario that, according to the Auditor General, cost taxpayers \$1
billion CAD).

An EMR permits the digitization of a patient chart and when used
properly should make it easier for a physician to deliver care. A good
system should provider a physician a bird's eye view of a patient's
current and ongoing conditions, their prescription history, their
recent lab results, history of their previous visits, and so on. OSCAR
(Open Source Clinical Application Resource), an approximately
10 year old project of McMaster University in Hamilton, Ontario,
Canada, is the open source community's attempt to provide such a system
to physicians at low or no cost.

OSCAR has many subsystems that provide functionality on a
component-by-component basis. For example, oscarEncounter provides an
interface for interacting with a patient's chart directly; Rx3 is a
prescription module that checks for allergies and drug interactions
automatically and allows a physician to directly fax a prescription to
a pharmacy from the UI; the Integrator is a component to enable data
sharing between multiple compatible EMRs. All of these separate
components come together to build the typical OSCAR user experience.

OSCAR won't be for every physician; for example, a specialist may not
find all the features of the system useful, and it is not easily
customizable. However, it offers a complete set of features for a
general physician interacting with patients on a day-to-day basis.

In addition, OSCAR is CMS 3.0 certified (and has applied for CMS 4.0
certification)---which allows physicians to receive funding for
installing it in their
clinic\footnote{See \url{https://www.emradvisor.ca/} for
details.}. Receiving CMS certification involves passing a set of
requirements from the Government of Ontario and paying a fee.

This chapter will discuss the architecture of OSCAR in fairly general
terms, describing the hierarchy, major components, and most
importantly the impact that past decisions have made on the
project. As a conclusion and to wrap up, there will be a discussion on
how OSCAR might have been designed today if there was an opportunity to do so.

\begin{aosasect1}{System Hierarchy}

As a Tomcat web application, OSCAR generally follows the typical
model-view-controller design pattern. This means that the model code
(Data Access Objects, or DAOs) are separate from the controller code
(servlets) and those are separated from the views (Java Server Pages,
or JSPs). The most significant difference between the two is that
servlets are classes and JSPs are HTML pages marked up with Java
code. Data gets placed into memory when a servlet executes and the JSP
reads that same data, usually done via reads and writes to the
attributes of the request object. Just about every JSP page in OSCAR
has this kind of design.

\end{aosasect1}

\begin{aosasect1}{Past Decision Making}

I mentioned that OSCAR is a fairly old project. This has implications
for how effectively the MVC pattern has been applied. In short, there
are sections of the code that completely disregard the pattern as they
were written before tighter enforcement of the MVC pattern began. Some
of the most common features are written this way---for example,
performing many actions related to demographics (patient records) are
done via the \code{demographiccontrol.jsp} file---this includes
creating patients and updating their data.

OSCAR's age is a hurdle for tackling many of the problems that are
facing the source tree today. Indeed, there has been significant
effort made to improve the situation, including enforcing design rules
via a code review process. This is an approach that the community at
present has decided will allow better collaboration in the future and
will prevent poor code from becoming part of the code base, which has
been a problem in the past.

This is by no means a restriction on how we could design parts of the
system now; it does, however, make it more complicated when deciding
to fix bugs in a dated part of OSCAR. Do you, as somebody tasked to
fix a bug in the Demographic Creation function, fix the bug with code
in the same style as it currently exists? Or do you re-write the
module completely so that it closely follows the MVC design pattern?

As developers we must carefully weigh our options in situations like
those. There is no guarantee that if you re-architect a part of the
system you will not create new bugs, and when patient data is on the
line, we must make the decision carefully.

\end{aosasect1}

\begin{aosasect1}{Version Control}

With OSCAR, a CVS repository has been used for much of the project's
life. Commits weren't often checked for consistency and it would be
possible to commit code that could break the build. It was tough for
developers to keep up with changes---especially new developers joining
the project late in its lifecycle. A new developer could see something
that they would want to change, make the change, and get it into the
source branch several weeks before anybody would notice that something
significant had been modified (this was especially prevalent during
long holidays, such as Christmas break, when not many people were
watching the source tree).

Things have changed now. OSCAR's source tree is now controlled by
git. Any commits to the main branch have to pass code style checking,
unit testing, successfully compile, and be code reviewed by the
developers (much of this is handled by the combination of Hudson and
Gerrit). The project has become much more tightly controlled. Many
or all of the issues caused by poor handling of the
source tree have been solved.

\end{aosasect1}

\begin{aosasect1}{Data Model/DAOs}

When looking through the OSCAR source, you may notice that there are
many different ways to access the database---you can use a direct
connection to the database via a class called \code{DBHandler}, use a
legacy Hibernate model, or use a generic JPA model. As new and easier
database access models became available, they were integrated into
OSCAR. The result is that there is now a slightly noisy picture of how
OSCAR interacts with data in MySQL, and the differences between the
three types of data access methods are best described with examples.

\end{aosasect1}

\begin{aosasect1}{EForms (\code{DBHandler})}

The EForm system allows users to create their own forms to attach to
patient records---this feature is usually used to replace a
paper-based form with a digital version. The form's template file is
loaded on each creation of a form of that type and the data in the
form is stored in the database for each instance. Each instance is
attached to a patient record.

EForms allow you to pull in certain types of data from a patient chart
or other area of the system via free-form SQL queries (which are
defined in a file called \code{apconfig.xml}). This can be extremely
useful as a form can load and then be immediately populated with
demographic or other relevant information without intervention from
the user; for example, you wouldn't have to type in a patient's name,
age, date of birth, hometown, phone number, or the last note that was
recorded for the patient (and so on).

A design decision was made when originally developing the EForm module
to use raw database queries to populate a POJO (plain-old Java object)
called \code{EForm} in the controller that is then passed to the view
layer to display data on the screen, sort of like a Java Bean. Using a
POJO in this case is actually closer in design to the Hibernate or JPA
architecture as I'll discuss in the next sections.

All of the functionality regarding saving EForm instances and
templates is done via raw SQL queries run through the \code{DBHandler}
class. Ultimately, \code{DBHandler} is a wrapper for a simple JDBC
object and does not scrutinize a query before sending it to the SQL
server. It should be added here that \code{DBHandler} is a potential
security flaw as it allows unchecked SQL to be sent to the server. Any
class that uses
\code{DBHandler} must implement its own checking to make sure that SQL
injection doesn't occur.

Depending on the type of application you're writing, direct access of
a database is sometimes fine. In certain cases, removing that
restriction can even speed development up. Using this method to access
the database doesn't conform to the model-view-controller design
pattern, though: if you're going to change your database structure
(the model), you have to change the SQL query elsewhere (in the
controller). Sometimes, adding certain columns or changing their type
in OSCAR's database tables requires this kind of invasive procedure
just to implement small features.

It may not surprise you to find out that the \code{DBHandler} object
is one of the oldest pieces of code still intact in the source. I
personally don't know where it originated from but I consider it to be
the most ``primitive'' of database access types that exist in the
OSCAR source. No new code is permitted to use this class, and if there
is code that gets committed that does use it, the commit will be
rejected automatically.

\end{aosasect1}

\begin{aosasect1}{Demographic Records (Hibernate)}

A demographic record contains general metadata about a patient: for
example, their name, age, address, language, and sex. Consider it to
be the result of an intake form that a patient fills out during their
first visit to a doctor. All of this data is retrieved and displayed
as part of OSCAR's Master Record for a specific demographic.

Using Hibernate to access the database is far safer than
using \code{DBHandler}. For one, you have to explicitly define which
columns match to which fields in your model object (in this case, the
Demographic class). If you want to perform complex joins, they have to
be done as prepared statements. Finally, you will only ever receive an
object of the type you ask for when performing a query, which is very
convenient.

The process of working with a Hibernate-style DAO and Model pair is
quite simple. In the case of the Demographic object, there's a file
called \code{Demographic.hbm.xml} that describes the mapping between object
field and database column. The file describes which table to look at
and what type of object to return. When OSCAR starts, this file will
be read and a sanity check occurs to make sure that this kind of
mapping can actually be made (server startup fails if it can't). Once
running, you grab and instance of the DemographicDao object and run
queries against it.

The best part about using Hibernate over \code{DBHandler} is that all
of the queries to the server are prepared statements. This restricts
you from running free-form SQL during runtime, but it also prevents
any type of SQL injection attack. Hibernate will often build large
queries to grab the data, and it doesn't always perform in an
extremely efficient way.

I mentioned in the previous section an example about the EForm module
using \code{DBHandler} to populate a POJO. This is the next logical
step to preventing that kind of code from being written. If the model
has to change, only the \code{.hbm.xml} file and the model class have
to change (a new field and getter/setter for the new column), and
doing so won't impact the rest of the application.

While newer than \code{DBHandler}, the Hibernate method is also
starting to show its age. It's not always convenient to use and
requires a big configuration file for each table you want to
access. Setting up a new object pair takes time and if you do it
incorrectly OSCAR won't even start. For this reason, nobody should be
writing new code that uses pure Hibernate, either. Instead, generic
JPA is being embraced in new development.

\end{aosasect1}

\begin{aosasect1}{Integrator Consent (JPA)}

The newest form of database access is done via generic JPA. If the
OSCAR project decided to switch from Hibernate to another database
access API, conforming to the JPA standard for DAOs and Model objects
makes it very easy to migrate. Unfortunately, because this is so
``new'' to the OSCAR project, there are almost no areas of the system
that actually use this method to get data!

In any case, let me explain how it works. Instead of a \code{.hbm.xml}
file, you add annotations to your Model and DAO objects. These
annotations describe the table to look in, column mappings for fields,
and join queries. Everything is contained inside the two files and
nothing else is necessary for their operation. Hibernate still runs
behind the scenes, though, in actually retrieving the data from the
database.

All of the Integrator's models are written using JPA---and they are
pretty good examples of both the new style of database access as well
as demonstrating that as a new technology to be implemented into
OSCAR, it hasn't been used in very many places yet. The Integrator is
a relatively new addition to the source. It makes quite a lot of sense
to use this new data access model as opposed to Hibernate.

Touching on a now-common theme in this section of the chapter, the
annotated POJOs that JPA uses allow for a far more streamlined
experience. For example, during the Integrator's build process, an SQL
file is created that sets up all of the tables for you---an enormously
useful thing to have. With that ability, it's impossible to create
mismatching tables and model objects (as you can do with any other
type of database access method) and you never have to worry about
naming of columns and tables. There are no direct SQL queries, so it's
not possible to create SQL injection attacks. In short, it ``just
works.''

The way that JPA works can be considered to be fairly similar to the
way that ActiveRecord works in Ruby on Rails. The model class defines
the data type and the database stores it; what happens in between
that---getting data in and out---is not up to the user.

\end{aosasect1}

\begin{aosasect1}{Issues with Hibernate and JPA}

Both Hibernate and JPA offer some significant benefits to typical use
cases. For simple retrieval and storage, they really cut time out of
development and debugging.

However, that doesn't mean that their implementation into OSCAR has
been without issue. Because the user doesn't define the SQL between
the database and the POJO referencing a specific row, Hibernate gets
to choose the best way to do it. The best way can manifest itself in a
couple of ways: Hibernate can choose to just retrieve the simple data
from the row, or it can perform a join and retrieve a lot of
information at once. Sometimes these joins get out of hand.

I'll talk about another example. The \code{casemgmt\_note} table
stores all patient notes. Each note object stores lots of metadata
about the note---but it also stores a list of all of the issues that
the note deals with (issues can be things like, ``smoking cessation''
or ``diabetes,'' which describe the contents of the note). The list of
issues is represented in the note object as a
\code{List{\textless}CaseManagementIssue{\textgreater}}, which is then
joined with the \code{casemgmt\_issue\_notes} table (which acts as a
mapping table) and finally the \code{casemgmt\_issue} table (to find
the actual issue rows).

When you want to write a custom query in Hibernate, which this
situation requires, you don't write standard SQL---you write HQL
(Hibernate Query Language) that is then translated to SQL (by
inserting internal column names for all the fields to be selected)
before parameters are inserted and the query is sent to the database
server. In this specific case, the query was written with basic joins
with no join columns---meaning that when the query was eventually
translated to SQL, it was so large that it wasn't immediately obvious
what the query was gathering. Additionally, in almost all cases, this
never created a large enough temporary table for it to matter.  
For most users, this query actually runs quickly enough that it's not
noticeable. In fact this works fine for nearly all cases. However,
this query is unbelievably inefficient.

Let's step back for a second. When you perform a join on two tables,
the server has to create a temporary table in memory. In the most
generic type of joins, the number of rows is equal to the number of
rows in the first table multiplied by the number of rows in the second
table. So if your table has 500,000 rows, and you join it with a table
that has 10,000,000 rows, you've just created a $5{\times}10^{12}$ row
temporary table in memory, which the select statement is then run against and
that temporary table is discarded.

For one very specific extreme case that we ran into, the join across
three tables caused a temporary table to be created that was created
was somewhere around $7{\times}10^{12}$ rows in length, of which about
1000 rows were eventually selected. This operation took about 5
minutes and locked the \code{casemgmt\_note} table while it was running.

The problem was solved, eventually, through the use of a prepared
statement that restricted the scope of the first table before joining
with the other two. The newer, far more efficient query brought the
number of rows to select down to a very manageable 300,000 and
enormously improved performance of the notes retrieval operation (down
to about 0.10 seconds to perform the same select statement as
mentioned previously).

The moral of the story is simply that while Hibernate does a fairly
good job, unless the join is very explicitly defined and controlled
(either in the \code{.hbm.xml} file or a join annotation in the object class
for a JPA model), it can very quickly get out of control. Dealing with
objects instead of SQL queries requires you to leave the actual
implementation of the query up to the database access library and only
really allows you to control definition. Unless you're careful with
how you define things, it can all fall apart under extreme
conditions. Furthermore, if you're a database programmer with lots of
SQL knowledge, it won't really help too much when designing a
JPA-enabled class and it removes some of the control that you would
have if you were writing an SQL statement manually. Ultimately, a good
knowledge of both SQL as well as JPA annotations and how they affect
queries is required.

\end{aosasect1}

\begin{aosasect1}{Permissions}

CAISI was originally a standalone product---a fork of OSCAR---to help
manage homeless shelters in Toronto. A decision was eventually made to
merge the code from CAISI into the main source branch. The original
CAISI project may no longer exist, but what it gave to OSCAR is very
important.

With that said, the permissions model in OSCAR is extremely powerful
and can be used to create just about as many roles and permission sets
as possible.

\emph{Providers} belong to \emph{programs} (as \emph{staff}) where
they have a specific \emph{role}. Each \emph{program} takes place at
a \emph{facility}. Each \emph{role} has a description (for example,
``doctor'', ``nurse'', ``social worker'', and so on) and a set of
attached global permissions. The permissions are written in a format
that makes them very easy to understand: ``read nurse notes'' may be a
permission that a doctor role may have, but the nurse role may not
have the ``read doctor notes'' permission.

This format may be easy to understand, but under the hood it requires
quite a bit of heavy lifting to actually check for these types of
permissions. The name of the role that the current provider has is
checked against its list of permissions for a match with the action
that they are trying to perform. For example, reading a doctor's notes
would have ``read doctor notes'' checked for each and every note
written by a doctor.

Another problem is the reliance on English. Anybody using OSCAR in a
language other than English would still need to write their
permissions in a format such as ``read \emph{[role]} notes''.

CAISI's permission model is a significant part of OSCAR, but it's not
the only model in place. Before CAISI was implemented, another
role-based (but not program-based) system was developed and is still
in use in many parts of the system today.

For this second system, providers are assigned one or many roles (for
example, ``doctor'', ``nurse'', ``admin'', and so on). They can be
assigned as many roles as you want---each role's permissions stack on
top of each other. These permissions are generally used for
restricting access to parts of the system, as opposed to CAISI's
permissions where you are being restricted to certain pieces of data
on a patient's chart. For example, a user has to have the ``\_admin''
``read'' permission on a role that they have assigned to them to be
able to access the Admin panel. Having the ``read'' permission will
exempt them from being able to perform administrative tasks,
however. They'll need the ``write'' permission as well for that.

Both of these systems accomplish roughly the same goal. It's due to
CAISI's merge later in the project lifecycle that they both
exist. They don't always exist happily together, so in reality it can
be a lot easier to just focus on using one for day-to-day operations
of OSCAR. You can generally date code in OSCAR by knowing which
permissions model preceded which other permissions model:

% TA: this kind of comes out of nowhere -- is there a better way to format this?
Provider Type $\rightarrow$ Provider Roles $\rightarrow$ CAISI Programs/Roles

The oldest type of permissions model, ``Provider Type'', is so dated
that it's actually not used in most parts of the system and is in fact
defaulted to ``doctor'' during new provider creation---having it as
any other value (such as ``receptionist'') causes significant issues
throughout the system. It's easier and more fine-grained to control
permissions via Provider Roles instead.

\end{aosasect1}

\begin{aosasect1}{Integrator}

OSCAR's Integrator component is a separate web application that
independent OSCAR instances use to exchange patient, program and
provider information over a secure link. It can be optionally
installed as a component for an installation in an environment such as
a LHN or a hospital. The easiest way to describe the Integrator is as
a temporary storage facility.

Consider the following use case and argument for use of the
Integrator: in Hospital Ex, there is an ENT (ear, nose, and throat)
clinic as well as an Endocrinology clinic. If an ENT doctor refers
their patient to an Endocrinologist upstairs, they may be required to
send along patient history and records. This is inconvenient and
generates more paper than is necessary---perhaps the patient is only
seeing the endocrinologist once. By using the Integrator, the
patient's data can be accessed on the endocrinologist's EMR, and
access to the contents of the patient's chart can be revoked after the
visit.

A more extreme example is if an unconscious man shows up in an ER with
nothing but his health card; because the home clinic and the
hospital's system are connected via the Integrator, the man's record
can be pulled and it can be very quickly realized that he has been
prescribed the blood thinner Warfarin. Ultimately, information
retrieval like this is what an EMR like OSCAR paired with the
Integrator can achieve.

\begin{aosasect2}{Technical Details}

The Integrator is available in source code form only, which requires
the user to retrieve and build it manually. Like OSCAR, it runs on a
standard installation of Tomcat with MySQL.

When the URL where the Integrator lives is accessed, it doesn't appear
to display anything useful. This component is almost purely a web
service; OSCAR communicates via POST and GET requests to the
Integrator URL.

As an independently developed project (initially as part of the CAISI
project, actually), the Integrator is fairly strict in adhering to the
MVC design pattern. The original developers have done an excellent job
of setting it up with very clearly defined lines between the models,
views, and controllers. The most recently implemented type of database
access layer that I mentioned earlier---generic JPA---is the only such
layer in the project. (As an interesting side note: because the entire
project is properly set up with JPA annotations on all the model
classes, an SQL script is created at build time that can be used to
initialize the structure of the database; the Integrator, therefore,
doesn't ship with a stand-alone SQL script.)

Communication is handled via web service calls described in WSDL XML
files that are available on the server. A client could query the
Integrator to find out what kind of functions are available and adapt
to it. This really means that the Integrator is compatible with any
kind of EMR that somebody decides to write a client for; the data
format is generic enough that it could easily be mapped to local
types.

For OSCAR, though, a client library is built and included in the main
source tree, for simplicity's sake. That library only ever needs to be
updated if new functions become available on the Integrator. A bug fix
on the Integrator doesn't require an update of that file.

\end{aosasect2}

\begin{aosasect2}{Design}

Data for the Integrator comes in from all of the connected EMRs at
scheduled times and once there another EMR can request that data. None
of the data on the Integrator is stored permanently, though---its
database could be erased and it could be rebuilt from the client data.

What data is sent is configured individually at each OSCAR sending
data to the Integrator, and except in situations where the entire
patient database has to be sent to the Integrator server, only patient
records that have been viewed since the previous push to the server
are sent. The process isn't exactly like delta patching, but it's
close.

\aosafigure{../images/oscar/integratoroscar.png}{Data exchange between OSCARs and Integrator.}{fig.oscar.integrator}

Let me go into a little more detail about how the Integrator works
with an example: a remote clinic seeing another clinic's patient. When
that clinic wants to access that patient's record, the clinics first
have to have been connected to the same Integrator server. The
receptionist can search the Integrator for the remote patient (by name
and optionally date of birth or sex) and find their record stored on
the server. They initiate the copy of a limited set of the patient's
demographic information and then double-check with the patient to make
sure that it's ok to retrieve their record from their home clinic by
completing a consent form. Once completed, the Integrator server will
deliver whatever information the Integrator knows about that
patient---notes, prescriptions, allergies, vaccinations, documents,
and so on. This data is cached locally so that the local OSCAR doesn't
have to send a request to the Integrator every time it wants to see
this data, but the local cache expires every hour.

\aosafigure{../images/oscar/integratorupload.png}{The Demographic information and associated data is sent to the Integrator during a data push from the home clinic. The record on the Integrator may not be a representation of the complete record from the home clinic as the OSCAR can choose not to send all patient data.}{fig.oscar.datatransfer}

After the initial setup of a remote patient by copying their
demographic data to the local OSCAR, that patient is set up as any other
on the system. All of the remote data that is retrieved from the
Integrator is marked as such (and the clinic from which it came from
is noted), but it's only temporarily cached on the local OSCAR. Any
local data that is recorded is recorded just like any other patient
data---to the patient record, and sent to the Integrator---but not
permanently stored on any remote machine.

\aosafigure{../images/oscar/integratordemodownload.png}{A remote OSCAR requests data from the Integrator by asking for a specific patient record. The Integrator server sends only the demographic information, which is stored permanently on the remote OSCAR.}{fig.oscar.request}

This has a very important implication, especially for patient consent
and how that factors into the design of the Integrator. Let's say that
a patient sees a remote physician and is fine with them having access
to their record, but they only want to do so temporarily. After their
visit, they can revoke the consent for that clinic to be able to view
that patient's record and the next time that clinic opens the
patient's chart there won't be any data there (with the exception of
any data that was locally recorded). This ultimately gives control
over how and when a record is viewed directly to the patient and is
similar to walking into a clinic carrying a copy of your paper
chart. They can see the chart while they're interacting with you, but
you take it home with you once you're done there.

\aosafigure{../images/oscar/integratordataretrieval.png}{A remote clinic can see the contents of a patient chart by asking for the data; if the appropriate consent is present, the data is sent. The data is never stored permanently on the remote OSCAR.}{fig.oscar.reply}

Another very important ability is for physicians to decide what kinds
of data they want to share with the other connected clinics via their
Integrator server. A clinic can choose to share all of a demographic
record or only parts of it, such as notes but not documents, allergies
but not prescriptions, and so on. Ultimately it's up to the group of
physicians who set up the Integrator server to decide what kinds of
data they're comfortable with sharing with each other.

As I mentioned briefly before, the Integrator is only a temporary
storage warehouse and no data is ever stored permanently there. This
is another very important decision that was made during development;
it allows clinics to back out of sharing any and all data via the
Integrator very easily---and in fact if necessary the entire
Integrator database can be wiped. If the database is wiped, no user of
a client will ever notice because the data will be accurately
reconstructed from the original data on all of the various connected
clients. An implication is that the OSCAR provider needs to trust the
Integrator provider to have wiped the database when they say so---it
is therefore best to deploy an Integrator to a group of physicians
already in a legal organization such as a Family Health Organization
or Family Health Team; the Integrator server would be housed at one of
these physician's clinics.

\end{aosasect2}

\begin{aosasect2}{Data Format}

The Integrator's client libraries are built via \code{wsdl2java}, which
creates a set of classes representing the appropriate data types the
web service communicates in. There are classes for each data type as
well as classes representing keys for each of these data types.

It's outside the scope of this chapter to describe how to build the
Integrator's client library. What's important to know is that once the
library is built, it must be included with the rest of the JARs in
OSCAR. This JAR contains everything necessary to set up the Integrator
connection and access all of the data types that the Integrator server
will return to OSCAR, such as CachedDemographic,
CachedDemographicNote, and CachedProvider, among many others. In
addition to the data types that are returned, there are ``WS'' classes
that are used for the retrieval of such lists of data in the first
place---the most frequently used being DemographicWs.

Dealing with the Integrator data can sometimes be a little
tricky. OSCAR doesn't have anything truly built-in to handle this kind
of data---so what is usually done is that when retrieving a certain
kind of patient data (for example, notes for a patient's chart) the
Integrator client is asked to retrieve data from the server. That data
is then manually transformed into a local class representing that data
(in the case of notes, it's a CaseManagementNote). A boolean flag is
set inside the class to indicate that it's a piece of remote content
and that is used to change how the data is displayed to the user on
the screen. On the opposite end, CaisiIntegratorUpdateTask handles taking local
OSCAR data, converting it into the Integrator's data format, and then
sending that data to the Integrator server.

This design may not be as efficient or as clean as possible, but it
does enable older parts of the system to become ``compatible'' with
Integrator-delivered data without much modification. In addition,
keeping the view as simple as possible by referring to only one type
of class improves the readability of the JSP file and makes it easier
to debug in the event of an error.

\end{aosasect2}

\end{aosasect1}

\begin{aosasect1}{Lessons Learned}

As you can probably imagine, OSCAR has its share of issues when it
comes to overall design. It does, however, provide a complete feature
set that most users will find no issues with. That's ultimately the
goal of the project---provide a good solution that works in most
situations.

I can't speak for the entire OSCAR community, so this section will be
highly subjective and from my point of view. I feel that there are
some important takeaways from an architectural discussion about the
project, though.

First, it's clear that poor source control in the past has caused the
architecture of the system to become highly chaotic in parts,
especially in areas where the controllers and the views blend
together. The way that the project was run in the past didn't prevent
this from happening, but the process has changed since and hopefully
we won't have to deal with such a problem again.

Next, because the project is so old, it's difficult to upgrade (or
even change) libraries without causing significant disruption
throughout the code base. That's exactly what has happened, though. I
find it often difficult to figure out what's necessary and what isn't
when I'm looking in the library folder. In addition to that, sometimes
when libraries undergo major upgrades they break backwards
compatibility (changing package names is a common offense). There are
often several libraries included with OSCAR that accomplish the same
task---this goes back to poor source control, but also that there has
been no list or documentation describing which library is required by
which component.

Additionally, OSCAR is a little inflexible when it comes to adding new
features to existing subsystems. For example, if you want to add a new
box to the E-Chart, you'll have to create a new JSP page, a new
servlet, modify the layout of the E-Chart (in a few places), and
modify the configuration file of the application so that your servlet
can load.

Next, due to the lack of documentation, sometimes it is nearly
impossible to figure out how a part of the system works---the original
contributor may not even be part of the project anymore---and often
the only tool you have to figure it out is a debugger. As a project of
this age, this is costing the community the potential for new
contributors to get involved. However, it's something that, as a
collaborative effort, the community is working on.

Finally, OSCAR is a repository for medical information and its
security is compromised by the inclusion of the \code{DBHandler} class
(discussed in a previous section). I personally feel that freeform
database queries that accept parameters should never be acceptable in
an EMR because it's so easy to perform SQL injection attacks. While
it's good that no new code is permitted that uses this class, it
should be a priority of the development team to remove all instances
of its use.

All of that may sound like some harsh criticism of the project. In the
past, all of these problems have been significant and, like I said,
prevent the community from growing as the barrier to entry is so
high. This is something that is changing, so in the future, these
issues won't be so much of a hindrance.

In looking back over the project's history (and especially over the
past few versions) we can come up with a better design for how the
application would be built. The system still has to provide a base
level of functionality (mandated by the Ontario government for
certification as an EMR), so that all has to be baked in by
default. But if I were to design OSCAR today, I would design it in a
truly modular fashion that would allow modules to be treated as
plugins; if you didn't like the default E-Form module, you could write
your own (or even another module entirely). It should be able to speak
to more systems (or more systems should be able to speak to it),
including medical hardware that you see in increasing use throughout
the industry, such as devices for measuring visual acuity. This also
means that it would be easy to adapt OSCAR to the requirements of
local and federal governments around the world for storing medical
data. Since every region has a different set of laws and requirements,
this kind of design would be crucial for making sure that OSCAR
develops a worldwide userbase.

I also believe that security should be the most important feature of
all. An EMR is only as secure as its least secure component, so there
should be focus on abstracting away as much data access as possible
from the application so that it stores and retrieves data in a
sandbox-style environment through a main data access layer API that
has been audited by a third-party and found to be adequate for storing
medical information. Other EMRs can hide behind obscurity and
proprietary code as a security measure (which isn't really a security
measure at all), but being open source, OSCAR should lead the charge
with better data protection.

I stand firmly as a believer in the OSCAR project. Of the many
hundreds of users that we know about (and the many hundreds that we
don't), we receive valuable feedback from the physicians who are
interacting with our project on a daily basis. Through the development
of new processes and new features, we hope to grow the installed base
and to support users from other regions. It is our intention to make
sure that what we deliver is something that improves the lives of the
physicians who use OSCAR as well as the lives of their patients by
creating better tools to help manage healthcare.

\end{aosasect1}

\end{aosachapter}
