\begin{aosachapter}{ZeroMQ}{s:zeromq}{Martin Sustrik}

% FIXME: too many one-sentence paragraphs

{\O}MQ is a messaging system, or ``message-oriented middleware'', if you
will. It's used in environments as diverse as financial services, game
development, embedded systems, academic research and aerospace.

Messaging systems work basically as instant messaging for
applications. An application decides to communicate an event to
another application (or multiple applications), it assembles the data
to be sent, hits the ``send'' button and there we go---the messaging
system takes care of the rest.

Unlike instant messaging, though, messaging systems have no GUI and
assume no human beings at the endpoints capable of intelligent
intervention when something goes wrong. Messaging systems thus have to
be both fault-tolerant and much faster than common instant messaging.

{\O}MQ was originally conceived as an ultra-fast messaging system
for stock trading and so the focus was on extreme optimization. The
first year of the project was spent devising benchmarking methodology
and trying to define an architecture that was as efficient as possible.

Later on, approximately in the second year of development, the focus
shifted to providing a generic system for building distributed
applications and supporting arbitrary messaging patterns, various
transport mechanisms, arbitrary language bindings, etc.

During the third year the focus was mainly on improving usability and 
flattening the learning curve. We've adopted the BSD Sockets API, tried to
clean up the semantics of individual messaging patterns, and so on.

Hopefully, this chapter will give an insight into how the three goals
above translated into the internal architecture of {\O}MQ, and provide
some tips for those who are struggling with the same problems.

Since its third year {\O}MQ has outgrown its codebase; there is
an initiative to standardise the wire protocols it uses, and an
experimental implementation of a {\O}MQ-like messaging system inside the Linux
kernel, etc.  These topics are not covered in this book. However, you can check
online
resources\footnote{\url{http://www.250bpm.com/concepts}}\footnote{\url{http://groups.google.com/group/sp-discuss-group}}\footnote{\url{http://www.250bpm.com/hits}}
for further details.

\begin{aosasect1}{Application vs.\ Library}

{\O}MQ is a library, not a messaging server. It took us several years
working on AMQP protocol, a financial industry attempt to standardise
the wire protocol for business messaging---writing a reference
implementation for it and participating in several large-scale
projects heavily based on messaging technology---to realise that there's
something wrong with the classic client/server model of smart
messaging server (broker) and dumb messaging clients.

Our primary concern at the time was with the performance: If there's a
server in the middle, each message has to pass the network twice (from
the sender to the broker and from the broker to the receiver) inducing
a penalty in terms of both latency and throughput. Moreover, if all
the messages are passed through the broker, at some point it's bound
to become the bottleneck.

A secondary concern was related to large-scale deployments: when the
deployment crosses organisational boundaries the concept of a central
authority managing the whole message flow doesn't apply any more. No
company is willing to cede control to a server in different
company; there are trade secrets and there's legal liability. The
result in practice is that there's one messaging server per company,
with hand-written bridges to connect it to messaging systems in other
companies. The whole ecosystem is thus heavily fragmented, and
maintaining a large number of bridges for every company involved doesn't
make the situation better. To solve this problem, we need a fully
distributed architecture, an architecture where every component can be
possibly governed by a different business entity. Given that the unit
of management in server-based architecture is the server, we can solve
the problem by installing a separate server for each component. In
such a case we can further optimize the design by making the server and
the component share the same processes. What we end up with is a
messaging library.

{\O}MQ was started when we got an idea about how to make
messaging work without a central server. It required turning the whole
concept of messaging upside down and replacing the model of an autonomous
centralised store of messages in the center of the network with a
``smart endpoint, dumb network'' architecture based on the end-to-end
principle\footnote{\url{http://en.wikipedia.org/wiki/End-to-end_principle}}.
The technical consequence of that decision was that {\O}MQ, from the very
beginning, was a library, not an application.

In the meantime we've been able to prove that this architecture is
both more efficient (lower latency, higher throughput) and more
flexible (it's easy to build arbitrary complex topologies instead of
being tied to classic hub-and-spoke model).

One of the unintended consequences, however, was that opting for the
library model improved the usability of the product. Over and over
again users express their happiness about the fact that they don't
have to install and manage a stand-alone messaging server. It turns
out that not having a server is a preferred option as it cuts
operational cost (no need to have a messaging server admin) and
improves time-to-market (no need to negotiate the need to run the
server with the client, the management or the operations team).

The lesson learned is that when starting a new project, you should opt
for the library design if at all possible. It's pretty easy to create
an application from a library by invoking it from a trivial program;
however, it's almost impossible to create a library from an existing
executable. A library offers much more flexibility to the users, at the
same time sparing them non-trivial administrative effort.

\end{aosasect1}

\begin{aosasect1}{Global State}

Global variables don't play well with libraries. A library may be loaded
several times in the process but even then there's
only a single set of global
variables. \aosafigref{fig.zeromq.multiuse} shows a {\O}MQ library being
used from two different and independent libraries. The application
then uses both of those libraries.

\aosafigure[150pt]{../images/zeromq/aosa1.png}{{\O}MQ being used by different libraries}{fig.zeromq.multiuse}

When such a situation occurs, both instances of {\O}MQ access the same
variables, resulting in race conditions, strange failures and undefined
behaviour.

To prevent this problem, the {\O}MQ library has no global variables. Instead,
a user of the library is responsible for creating the global state
explicitly. The object containing the global state is called
\emph{context}. While from the user's perspective context looks more or
less like a pool of worker threads, from {\O}MQ's perspective it's just
an object to store any global state that we happen to need. 
In the picture above, \code{libA} would have its own context and \code{libB}
would have its own as well. There would be no way for one of them to
break or subvert the other one.

The lesson here is pretty obvious: Don't use global state in
libraries. If you do, the library is likely to break when
it happens to be instantiated twice in the same process.

\end{aosasect1}

\begin{aosasect1}{Performance}

When {\O}MQ was started, its primary goal was to optimize performance.  
Performance of messaging systems is expressed using two metrics:
throughput---how many messages can be passed during a given amount
of time; and latency---how long it takes for a message to get from
one endpoint to the other.

Which metric should we focus on? What's the relationship between the
two?
Isn't it obvious? Run the test, divide the overall time of the test by
number of messages passed and what you get is latency. Divide the
number of messages by time and what you get is throughput. In other
words, latency is the inverse value of throughput. Trivial, right?

Instead of starting coding straight away we spent some weeks 
investigating the performance metrics in detail and we found out
that the relationship between throughput and latency is much more
subtle than that, and often the metrics are quite counter-intuitive.

%% FIXME - Might want to rearrange this so that the diagram below stays with
% its explanation. Revisit this closer to publication. --ARB
\aosafigure[175pt]{../images/zeromq/aosa8.png}{Sending messages from A to B}{fig.zeromq.a2b}

%% QUERY: Some of the numbers in this passage looked wrong so I corrected
% them -- please have a look and make sure the numbers are now correct. --ARB
Imagine A sending messages to B. (See \aosafigref{fig.zeromq.a2b}.)  The
overall time of the test is 6 seconds. There are 5 messages
passed. Therefore the throughput is 0.83 msgs/sec ($\frac{5}{6}$) and 
the latency is 1.2 sec ($\frac{6}{5}$), right?

Have a look at the diagram again. It takes a different time for each
message to get from A to B: 2 sec, 2.5 sec, 3 sec, 3.5 sec, 4 sec. The
average is 3 seconds, which is pretty far away from our original
calculation of 1.2 second.  
This example shows the misconceptions people are intuitively
inclined to make about performance metrics.

Now have a look at the throughput. The overall time of the test is 6
seconds. However, at A it takes just 2 seconds to send all the
messages. From A's perspective the throughput is 2.5 msgs/sec
($\frac{5}{2}$). At B it takes 4 seconds to receive all messages. So from B's
perspective the throughput is 1.25 msgs/sec ($\frac{5}{4}$). Neither of these
numbers matches our original calculation of 1.2 msgs/sec.

To make a long story short, latency and throughput are two different
metrics; that much is obvious. The important thing is to understand
the difference between the two and their mutual relationship. Latency
can be measured only between two different points in the
system; There's no such thing as latency at point A. Each message has
its own latency. You can average the latencies of multiple messages;
however, there's no such thing as latency of a stream of
messages. 

Throughput, on the other hand, can be measured only at a
single point of the system. There's a throughput at the sender,
there's a throughput at the receiver, there's a throughput at any
intermediate point between the two, but there's no such thing as
overall throughput of the whole system. And throughput make sense only for
a set of messages; there's no such thing as throughput of a single
message.

As for the relationship between throughput and latency, it turns out
there really is a relationship; however, the formula involves
integrals and we won't discuss it here. For more information, read the
literature on queueing
theory\footnote{\url{http://en.wikipedia.org/wiki/Queueing_theory}}.

There are many more pitfalls in benchmarking the messaging systems
that we won't go further into. The stress should rather be placed on
the lesson learned: Make sure you understand the problem you are
solving. Even a problem as simple as ``make it fast'' can take lot of
work to understand properly. What's more, if you don't understand the
problem, you are likely to build implicit assumptions and popular
myths into your code, making the solution either flawed or at least
much more complex or much less useful than it could possibly be.

\end{aosasect1}

\begin{aosasect1}{Critical Path}

We discovered during the optimization process that three factors have a crucial
impact on performance:

\begin{aosaitemize}
\item Number of memory allocations
\item Number of system calls
\item Concurrency model
\end{aosaitemize}

However, not every memory
allocation or every system call has the same effect on 
performance.  
The performance we are interested in in messaging systems is the number of
messages we can transfer between two endpoints during a given amount
of time. Alternatively, we may be interested in how long it takes for
a message to get from one endpoint to another.

However, given that {\O}MQ is designed for scenarios with long-lived
connections, the time it takes to establish a connection or the time
needed to handle a connection error is basically irrelevant. These
events happen very rarely and so their impact on overall performance
is negligible.

The part of a codebase that gets used very frequently, over and over
again, is called the \emph{critical path}; optimization should focus on
the critical path.

Let's have a look at an example: {\O}MQ is not extremely optimized with
respect to memory allocations. For example, when manipulating strings,
it often allocates a new string for each intermediate phase of the
transformation. However, if we look strictly at the critical path---the
actual message passing---we'll find out that it uses almost no memory
allocations. If messages are small, it's just one memory allocation
per 256 messages (these messages are held in a single large allocated
memory chunk). If, in addition, the stream of messages is steady,
without huge traffic peaks, the number of memory allocations on the
critical path drops to zero (the allocated memory chunks are not
returned to the system, but re-used over and over again).

Lesson learned: optimize where it makes difference. Optimizing pieces
of code that are not on the critical path is wasted effort.

\end{aosasect1}

\begin{aosasect1}{Allocating Memory}

Assuming that all the infrastructure was initialised and a connection
between two endpoints has been established, there's only one thing
to allocate when sending a message: the message itself. Thus, to
optimize the critical path we had to look into how messages are
allocated and passed up and down the stack.

It's common knowledge in the high-performance networking field that the best
performance is achieved by carefully balancing the cost of message
allocation and the cost of message copying\footnote{For example,
  \url{http://hal.inria.fr/docs/00/29/28/31/PDF/Open-MX-IOAT.pdf}.
  See different handling of ``small'', ``medium'' and ``large''
  messages.}. For small messages, copying is much cheaper than
allocating memory. It makes sense to allocate no new memory chunks at
all and instead to copy the message to preallocated memory whenever
needed. For large messages, on the other hand, copying is much more
expensive than memory allocation. It makes sense to allocate the
message once and pass a pointer to the allocated block, instead of
copying the data. This approach is called ``zero-copy''.

{\O}MQ handles both cases in a transparent manner. A {\O}MQ message is
represented by an opaque handle. The content of very small messages is
encoded directly in the handle. So making a copy of the handle
actually copies the message data. When the message is larger, it's
allocated in a separate buffer and the handle contains just a pointer to
the buffer. Making a copy of the handle doesn't result in copying the
message data, which makes sense when the message
is megabytes long (\aosafigref{fig.zeromq.ref}). It should be noted
that in the latter case the buffer is reference-counted so that it can
be referenced by multiple handles without the need to copy the data.

\aosafigure[250pt]{../images/zeromq/aosa2.png}{Message copying (or not)}{fig.zeromq.ref}

Lesson learned: When thinking about performance, don't assume there's
a single best solution. It may happen that there are several
subclasses of the problem (e.g., small messages vs. large messages),
each having its own optimal algorithm.

\end{aosasect1}

\begin{aosasect1}{Batching}

It has already been mentioned that the sheer number of system calls in
a messaging system can result in a performance bottleneck. Actually,
the problem is much more generic than that. There's a non-trivial
performance penalty associated with traversing the call stack and
thus, when creating high-performance applications, it's wise to avoid
as much stack traversing as possible.

\aosafigure[150pt]{../images/zeromq/aosa3.png}{Sending four messages}{fig.zeromq.send4}

Consider \aosafigref{fig.zeromq.send4}.  To send four messages, you
have to traverse the entire network stack four times (i.e., {\O}MQ, glibc,
user/kernel space boundary, TCP implementation, IP implementation,
Ethernet layer, the NIC itself and back up the stack again). 

However,
if you decide to join those messages into a single batch, there would
be only one traversal of the stack (\aosafigref{fig.zeromq.joinmessage}).
The impact on message throughput can be overwhelming: up to two orders
of magnitude, especially if the messages are small and hundreds of
them can be packed into a single batch.

\aosafigure[150pt]{../images/zeromq/aosa4.png}{Batching messages}{fig.zeromq.joinmessage}

On the other hand, batching can have negative impact on latency. Let's
take, for example, the well-known Nagle's
algorithm\footnote{\url{http://en.wikipedia.org/wiki/Nagle's_algorithm}},
as implemented in TCP. It delays the outbound messages for a certain
amount of time and merges all the accumulated data into a single
packet. Obviously, the end-to-end latency of the first message in the
packet is much worse than the latency of the last one. Thus, it's
common for applications that need consistently low latency to switch
Nagle's algorithm off. It's even common to switch off batching on
all levels of the stack (e.g., NIC's interrupt coalescing feature).

But again, no batching means extensive traversing of the stack and
results in low message throughput. We seem to be caught in a throughput
versus latency dilemma.

{\O}MQ tries to deliver consistently low latencies combined with high
throughput using the following strategy: when message flow is sparse and
doesn't exceed the network stack's bandwidth, {\O}MQ turns all the batching
off to improve latency. The trade-off here is somewhat higher CPU
usage---we still have to traverse the stack frequently. However, that
isn't considered to be a problem in most cases.

When the message rate exceeds the bandwidth of the network stack, the
messages have to be queued---stored in memory till the stack is ready
to accept them. Queueing means the latency is going to grow. If the
message spends one second in the queue, end-to-end latency will be at
least one second. What's even worse, as the size of the queue grows,
latencies will increase gradually. If the size of the queue is not
bound, the latency can exceed any limit.

It has been observed that even though the network stack is tuned for
lowest possible latency (Nagle's algorithm switched off, NIC interrupt
coalescing turned off, etc.) latencies can still be dismal because of
the queueing effect, as described above.

In such situations it makes sense to start batching
aggressively. There's nothing to lose as the latencies are already
high anyway. On the other hand, aggressive batching improves
throughput and can empty the queue of pending messages---which in
turn means the latency will gradually drop as the queueing delay
decreases. Once there are no outstanding messages in the queue, the
batching can be turned off to improve the latency even further.

One additional observation is that the batching should only be done on
the topmost level. If the messages are batched there, the lower layers
have nothing to batch anyway, and so all the batching algorithms
underneath do nothing except introduce additional latency.

%% FIXME: Think about putting these Lesson Learneds into sidebars, 
%% as long as they don't float away. --ARB
Lesson learned: To get optimal throughput combined with optimal
response time in an asynchronous system, turn off all the batching
algorithms on the low layers of the stack and batch on the topmost
level. Batch only when new data are arriving faster than they can be
processed.

\end{aosasect1}

\begin{aosasect1}{Architecture Overview}

Up to this point we have focused on generic principles that make {\O}MQ
fast. From now on we'll have a look at the actual architecture of the
system (\aosafigref{fig.zeromq.arch}).

\aosafigure[275pt]{../images/zeromq/aosa9.png}{{\O}MQ architecture}{fig.zeromq.arch}

The user interacts with {\O}MQ using so-called ``sockets''. They are pretty
similar to TCP sockets, the main difference being that each socket can
handle communication with multiple peers, a bit like unbound UDP
sockets do.

The socket object lives in the user's thread (see the discussion of
threading models in the next section). Aside from that, {\O}MQ is running
multiple worker threads that handle the asynchronous part of the
communication: reading data from the network, enqueueing messages,
accepting incoming connections, etc.

There are various objects living in the worker threads. Each of these
objects is owned by exactly one parent object (ownership is denoted by 
a simple full line in the
diagram). The parent can live in a
different thread than the child. Most objects are owned directly by
sockets; however, there are couple of cases where an object is owned
by an object which is owned by the socket. What we get is a tree of
objects, with one such tree per socket. The tree is used during 
shut down; no object can shut itself down until it closes all its
children. This way we can ensure that the shut down process works as
expected; for example, that pending outbound messages are pushed to
the network prior to terminating the sending process.

Roughly speaking, there are two kinds of asynchronous objects; there
are objects that are not involved in message passing and there are
objects that are.  
The former have to do mainly with connection management. For example,
a TCP listener object listens for incoming TCP connections and creates
an engine/session object for each new connection. Similarly, a TCP
connecter object tries to connect to the TCP peer and when it succeeds
it creates an engine/session object to manage the connection. When
such connection fails, the connecter object tries to re-establish it.

The latter are objects that are handling data transfer itself. These
objects are composed of two parts: the \emph{session object} is responsible for
interacting with the {\O}MQ socket, and the \emph{engine object} is responsible for
communication with the network. There's only one kind of the session
object, but there's a different engine type for each underlying
protocol {\O}MQ supports. Thus, we have TCP engines, IPC (inter-process
%% QUERY: Should we expand PGM to "Pragmatic General Multicast" or is it one
%% of those things which is better known by its acronym? --ARB
communication) engines, PGM\footnote{Reliable multicast protocol, see RFC 3208.}
engines, etc. The set of engines is extensible---in the future we may
choose to implement, say, a WebSocket engine or an SCTP engine.

The sessions are exchanging messages with the sockets. There are two
directions to pass messages in and each direction is handled by a pipe
object. Each pipe is basically a lock-free queue optimized for fast passing
of messages between threads.

Finally, there's a context object (discussed in the previous sections but
not shown on the diagram) that holds the global state and is
accessible by all the sockets and all the asynchronous objects.

\end{aosasect1}

\begin{aosasect1}{Concurrency Model}

One of the requirements for {\O}MQ was to take advantage of multi-core
boxes; in other words, to scale the throughput linearly with the number of
available CPU cores.

Our previous experience with messaging systems showed that using
multiple threads in a classic way (critical sections, semaphores, etc.)
doesn't yield much performance improvement. In fact, a multi-threaded
version of a messaging system can be slower than a single-threaded
one, even if measured on a multi-core box. Individual threads are
simply spending too much time waiting for each other while, at the
same time, eliciting a lot of context switching that slows the system
down.

Given these problems, we've decided to go for a different model. The
goal was to avoid locking entirely and let each thread run at full
speed. The communication between threads was to be provided via
asynchronous messages (events) passed between the threads. This, as
insiders know, is the classic \emph{actor
model}\footnote{\url{http://en.wikipedia.org/wiki/Actor_model}}.

The idea was to launch one worker thread per CPU core---having two
threads sharing the same core would only mean a lot of context
switching for no particular advantage. Each internal {\O}MQ object, such
as say, a TCP engine, would be tightly bound to a particular worker
thread. That, in turn, means that there's no need for critical
sections, mutexes, semaphores and the like. Additionally, these {\O}MQ
objects won't be migrated between CPU cores so would thus avoid the
negative performance impact of cache pollution
(\aosafigref{fig.zeromq.cache}).

\aosafigure[250pt]{../images/zeromq/aosa5.png}{Multiple worker threads}{fig.zeromq.cache}

This design makes a lot of traditional multi-threading problems
disappear. Nevertheless, there's a need to share the worker thread
among many objects, which in turn means there has to be some kind of
cooperative multitasking. This means we need a scheduler; objects need 
to be event-driven
rather than being in control of the entire event loop; we have to take
care of arbitrary sequences of events, even very rare ones; we have to
make sure that no object holds the CPU for too long; etc.

In short, the whole system has to become fully asynchronous. No object can
afford to do a blocking operation, because it would not only block
itself but also all the other objects sharing the same worker thread. All
objects have to become, whether explicitly or implicitly, state
machines. With hundreds or thousands of state machines running in
parallel you have to take care of all the possible interactions
between them and---most importantly---of the shutdown process.

It turns out that shutting down a fully asynchronous system in a clean
way is a dauntingly complex task. Trying to shut down a thousand moving
parts, some of them working, some idle, some in the process of being
initiated, some of them already shutting down by themselves, is prone
to all kinds of race conditions, resource leaks and similar. The
shutdown subsystem is definitely the most complex part of {\O}MQ. A quick
check of the bug tracker indicates that some 30--50\% of reported bugs
are related to shutdown in one way or another.

Lesson learned: When striving for extreme performance and scalability,
consider the actor model; it's almost the only game in town in such
cases. However, if you are not using a specialised system like Erlang or
{\O}MQ itself, you'll have to write and debug a lot of infrastructure by
hand. Additionally, think, from the very beginning, about the procedure 
to shut down the system. It's going to be the most complex part of the codebase and
if you have no clear idea how to implement it, you should probably
reconsider using the actor model in the first place.

\end{aosasect1}

\begin{aosasect1}{Lock-Free Algorithms}

Lock-free algorithms have been in vogue lately. They are simple
mechanisms for inter-thread communication that don't rely on the
kernel-provided synchronisation primitives, such as mutexes or
semaphores; rather, they do the synchronisation using atomic CPU
operations, such as atomic compare-and-swap (CAS). It should be
understood that they are not literally lock-free---instead, locking is
done behind the scenes on the hardware level.

{\O}MQ uses a lock-free queue in pipe objects to pass messages between the
user's threads and {\O}MQ's worker threads. There are two interesting
aspects to how {\O}MQ uses the lock-free queue.

First, each queue has exactly one writer thread and exactly one reader
thread. If there's a need for 1-to-\code{N} communication, multiple queues
are created (\aosafigref{fig.zeromq.multiqueue}).  Given that this way
the queue doesn't have to take care of synchronising the writers
(there's only one writer) or readers (there's only one reader) it can
be implemented in an extra-efficient way.

\aosafigure[275pt]{../images/zeromq/aosa6.png}{Queues}{fig.zeromq.multiqueue}

Second, we realised that while lock-free algorithms were more
efficient than classic mutex-based algorithms, atomic CPU operations
are still rather expensive (especially when there's contention between
CPU cores) and doing an atomic operation for each message written
and/or each message read was slower than we were willing to accept.

The way to speed it up---once again---was batching.  Imagine you had
10 messages to be written to the queue. It can happen, for example,
when you received a network packet containing 10 small
messages. Receiving a packet is an atomic event; you cannot get 
half of it. This atomic event results in the need to write 10 messages to
the lock-free queue. There's not much point in doing an atomic
operation for each message. Instead, you can accumulate the messages
in a ``pre-write'' portion of the queue that's accessed solely by the
writer thread, and then flush it using a single atomic operation.

The same applies to reading from the queue. Imagine the 10 messages above
were already flushed to the queue. The reader thread can extract each
message from the queue using an atomic operation. However, it's 
overkill; instead, it can move all the pending messages to a
``pre-read'' portion of the queue using a single atomic
operation. Afterwards, it can retrieve the messages from the ``pre-read''
buffer one by one. ``Pre-read'' is owned and accessed solely by the
reader thread and thus no synchronisation whatsoever is needed in that
phase.

%% FIXME: This figure is orphaned from the paragraph which explains it.
%% Try and work it out. --ARB
\aosafigure[300pt]{../images/zeromq/aosa7.png}{Lock-free queue}{fig.zeromq.flush}

The arrow on the left of \aosafigref{fig.zeromq.flush} shows how
the pre-write buffer can be flushed to the queue simply by modifying a
single pointer. The arrow on the right shows how the whole content of the
queue can be shifted to the pre-read by doing nothing but modifying
another pointer.

Lesson learned: Lock-free algorithms are hard to invent, troublesome
to implement and almost impossible to debug. If at all possible, use
an existing proven algorithm rather than inventing your own. When
extreme performance is required, don't rely solely on lock-free
algorithms. While they are fast, the performance can be significantly
improved by doing smart batching on top of them.

\end{aosasect1}

\begin{aosasect1}{API}

The user interface is the most important part of any product. It's the
only part of your program visible to the outside world and if you get
it wrong the world will hate you. In end-user products it's either the GUI
or the command line interface. In libraries it's the API.

In early versions of {\O}MQ the API was based on AMQP's model of
exchanges and queues\footnote{See the AMQP specification at
  \url{https://www.amqp.org/confluence/download/attachments/720900/amqp0-9-1.pdf}. From
  historical perspective it's interesting to have a look at the
  whitepaper from 2007 that tries to reconcile AMQP with brokerless
  model of messaging, which is at
  \url{http://www.zeromq.org/whitepapers:messaging-enabled-network}.}. I
spent the end of 2009 rewriting it almost from scratch to use the BSD
Socket
API\footnote{\url{http://en.wikipedia.org/wiki/Berkeley_sockets}}
instead. That was the turning point; {\O}MQ adoption soared from that
point on. While before it was a niche product used by a bunch of
messaging experts, afterwards it became a handy commonplace tool for
anybody. In a year or so the size of the community increased tenfold,
some 20 bindings to different languages were implemented, etc.

The user interface defines the perception of a product. With basically
no change to the functionality---just by changing the API---{\O}MQ 
changed from an ``enterprise messaging'' product to a ``networking''
product. In other words, the perception changed from ``a complex piece
of infrastructure for big banks'' to ``hey, this helps me to send my
10-byte-long message from application A to application B''.

Lesson learned: Understand what you want your project to be and design
the user interface accordingly. Having a user interface that doesn't align with
the vision of the project is a 100\% guaranteed way to fail.

One of the important aspects of the move to the BSD Sockets API was that it
wasn't a revolutionary freshly-invented API, but an existing and
well-known one. Actually, the BSD Sockets API is one of the oldest APIs
still in active use today; it dates back to 1983 and 4.2BSD Unix. It's been
widely used and stable for literally decades.

The above fact brings a lot of advantages.  Firstly, it's an API that
everybody knows, so the learning curve is ludicrously flat. Even if
you've never heard of {\O}MQ, you can build your first application
in couple of minutes thanks to the fact that you are able to reuse
your BSD Sockets knowledge.

Secondly, using a widely implemented API enables integration of {\O}MQ
with existing technologies. For example, exposing {\O}MQ objects as
``sockets'' or ``file descriptors'' allows for processing TCP, UDP,
pipe, file and {\O}MQ events in the same event loop. Another example: the
experimental project to bring {\O}MQ-like functionality to the Linux
kernel\footnote{\url{https://github.com/250bpm/linux-2.6}} turned out
to be pretty simple to implement. By sharing the same conceptual
framework it can re-use a lot of infrastructure already in place.

Thirdly and probably most importantly, the fact that the BSD Sockets
API survived almost three decades despite numerous attempts to
replace it means that there is something inherently right in the
design. BSD Sockets API designers have---whether deliberately or by
chance---made the right design decisions. By adopting the API we can
automatically share those design decisions without even knowing what
they were and what problem they were solving.

Lesson learned: While code reuse has been promoted from time
immemorial and pattern
reuse\footnote{\url{http://en.wikipedia.org/wiki/Design_pattern}}
joined in later on, it's important to think of reuse in an
even more generic way. When designing a product, have a look at similar
products. Check which have failed and which have succeeded; learn from
the successful projects. Don't succumb to Not Invented Here
syndrome. Reuse
the ideas, the APIs, the conceptual frameworks, whatever you find
appropriate. By doing so you are allowing users to reuse their
existing knowledge. At the same time you may be avoiding technical
pitfalls you are not even aware of at the moment.

\end{aosasect1}

\begin{aosasect1}{Messaging Patterns}

In any messaging system, the most important design problem is that of how to
provide a way for the user to specify which messages are routed to
which destinations.  There are two main approaches, and I believe this
dichotomy is quite generic and applicable to basically any problem
encountered in the domain of software.

One approach is to adopt the Unix
philosophy
of ``do one thing and do it well''. What this means is that the problem
domain should be artificially restricted to a small and well-understood 
area. The program should then solve this restricted problem
in a correct and exhaustive way. An example of such approach in
the messaging area is MQTT\footnote{\url{http://mqtt.org/}}. It's a
protocol for distributing messages to a set of consumers. It can't be
used for anything else (say for RPC) but it is easy to use and does
message distribution well.

The other approach is to focus on generality and provide a powerful
and highly configurable system. AMQP is an example of such a
system. Its model of queues and exchanges provides the user with the
means to programatically define almost any routing algorithm they can
think of. The trade-off, of course, is a lot of options to take care
of.

{\O}MQ opts for the former model because it allows the resulting
product to be used by basically anyone, while the generic model
requires messaging experts to use it. To demonstrate the point, let's
have a look how the model affects the complexity of the API. What
follows is implementation of RPC client on top of a generic system
(AMQP):

\begin{verbatim}
connect ("192.168.0.111")
exchange.declare (exchange="requests", type="direct", passive=false,
    durable=true, no-wait=true, arguments={})
exchange.declare (exchange="replies", type="direct", passive=false,
    durable=true, no-wait=true, arguments={})
reply-queue = queue.declare (queue="", passive=false, durable=false,
    exclusive=true, auto-delete=true, no-wait=false, arguments={})
queue.bind (queue=reply-queue, exchange="replies",
    routing-key=reply-queue)
queue.consume (queue=reply-queue, consumer-tag="", no-local=false,
    no-ack=false, exclusive=true, no-wait=true, arguments={})
request = new-message ("Hello World!")
request.reply-to = reply-queue
request.correlation-id = generate-unique-id ()
basic.publish (exchange="requests", routing-key="my-service",
    mandatory=true, immediate=false)
reply = get-message ()
\end{verbatim}

On the other hand, {\O}MQ splits the messaging landscape into so-called ``messaging
patterns''. Examples of the patterns are ``publish/subscribe'',
``request/reply'' or ``parallelised pipeline''. Each messaging pattern
is completely orthogonal to other patterns and can be thought of as a
separate tool.

What follows is the re-implementation of the above application using
{\O}MQ's request/reply pattern. Note how all the option tweaking is
reduced to the single step of choosing the right messaging pattern
(``\code{REQ}''):

\begin{verbatim}
s = socket (REQ)
s.connect ("tcp://192.168.0.111:5555")
s.send ("Hello World!")
reply = s.recv ()
\end{verbatim}

Up to this point we've argued that specific solutions are better than
generic solutions. We want our solution to be as specific as
possible. However, at the same time we want to provide our customers
with as wide a range of functionality as possible. How can we solve this
apparent contradiction?

The answer consists of two steps:

\begin{aosaenumerate}

\item Define a layer of the stack to deal with a particular problem
  area (e.g. transport, routing, presentation, etc.).

\item Provide multiple implementations of the layer. There should be a
  separate non-intersecting implementation for each use case.

\end{aosaenumerate}

Let's have a look at the example of the transport layer in the
Internet stack. It's meant to provide services such as transferring
data streams, applying flow control, providing reliability, etc., on the
top of the network layer (IP). It does so by defining multiple
non-intersecting solutions: TCP for connection-oriented reliable
stream transfer, UDP for connectionless unreliable packet transfer,
SCTP for transfer of multiple streams, DCCP for unreliable connections
and so on.

Note that each implementation is completely orthogonal: a UDP endpoint
cannot speak to a TCP endpoint. Neither can a SCTP endpoint speak to a DCCP
endpoint. It means that new implementations can be added to the stack
at any moment without affecting the existing portions of the
stack. Conversely, failed implementations can be forgotten and
discarded without compromising the viability of the transport layer as a
whole.

The same principle applies to messaging patterns as defined by
{\O}MQ. Messaging patterns form a layer (the so-called ``scalability layer'')
on top of the transport layer (TCP and friends). Individual messaging
patterns are implementations of this layer. They are strictly
orthogonal---the publish/subscribe endpoint can't speak to the request/reply
endpoint, etc. Strict separation between the patterns in turn means
that new patterns can be added as needed and that failed experiments
with new patterns won't hurt the existing patterns.

Lesson learned: When solving a complex and multi-faceted problem it
may turn out that a monolithic general-purpose solution may not be the
best way to go. Instead, we can think of the problem area as an
abstract layer and provide multiple implementations of this layer,
each focused on a specific well-defined use case. When doing so,
delineate the use case carefully. Be sure about what is in the scope
and what is not. By restricting the use case too aggressively the
application of your software may be limited. If you define the problem
too broadly, however, the product may become too complex, blurry and
confusing for the users.

\end{aosasect1}

\begin{aosasect1}{Conclusion}

As our world becomes populated with lots of small computers connected
via the Internet---mobile phones, RFID readers, tablets and
laptops, GPS devices, etc.---the problem of distributed computing ceases
to be the domain of academic science and becomes a common everyday
problem for every developer to tackle. The solutions, unfortunately,
are mostly domain-specific hacks. This article summarises our
experience with building a large-scale distributed system in a
systematic manner. It focuses on problems that are interesting from
a software architecture point of view, and we hope that designers and
programmers in the open source community will find it useful.

\end{aosasect1}

\end{aosachapter}
