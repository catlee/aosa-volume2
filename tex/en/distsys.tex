\begin{aosachapter}{Scalable Web Architecture and Distributed Systems}{s:distsys}{Kate Matsudaira}

Open source software has become a fundamental building block for some
of the biggest websites. And as those websites have grown,
best practices and guiding principles around their architectures have
emerged. This chapter seeks to cover some of the key issues to
consider when designing large websites, as well as some of the
building blocks used to achieve these goals.

This chapter is largely focused on web systems, although some of the
material is applicable to other distributed systems as well.

\begin{aosasect1}{Principles of Web Distributed Systems Design}

What exactly does it mean to build and operate a scalable web site or
application? At a primitive level it's just connecting users with remote
resources via the Internet---the part that makes it scalable is
that the resources, or access to those resources, are distributed
across multiple servers.

Like most things in life, taking the time to plan ahead when building a
web service can help in
the long run; 
understanding some of the considerations and tradeoffs behind big
websites can result in smarter decisions at the creation of
smaller web sites. Below are some of the key principles that influence
the design of large-scale web systems:

\begin{aosadescription}

\item{Availability:} The uptime of a website is absolutely critical to
  the reputation and functional of many companies. For some of the
  larger online retail sites, being unavailable for even minutes can
  result in thousands or millions of dollars in lost revenue, so
  designing their systems to be constantly available and resilient to
  failure is both a fundamental business and a technology
  requirement. High availability in distributed systems requires the
  careful consideration of redundancy for key components, rapid
  recovery in the event of partial system failures, and graceful
  degradation when problems occur.

\item{Performance:} Website performance has become an important
  consideration for most sites. The speed of a website affects
  usage and user satisfaction, as well as search engine rankings, a
  factor that directly correlates to revenue and retention. As a
  result, creating a system that is optimized for fast responses and
  low latency is key.

\item{Reliability:} A system needs to be reliable, such that a request
  for data will consistently return the same data. In the event the
  data changes or is updated, then that same request should return the
  new data. Users need to know that if something is written to the
  system, or stored, it will persist and can be relied on to be in
  place for future retrieval.

\item{Scalability:} When it comes to any large distributed system, 
  size is just one aspect of scale that needs to be considered. Just as
  important is the effort required to increase capacity to handle
  greater amounts of load, commonly referred to as the
  scalability of the system. Scalability can refer to many different
  parameters of the system: how much additional traffic can it
  handle, how easy is it to add more storage capacity, or even how
  many more transactions can be processed.

\item{Manageability:} Designing a system that
  is easy to operate is another important consideration.  The manageability of
  the system equates to the scalability of operations: maintenance and
  updates. Things to consider for manageability are the ease of diagnosing and
  understanding problems when they occur, ease of making updates or
  modifications, and how simple the system is to operate. (I.e., does it
  routinely operate without failure or exceptions?)

\item{Cost:} Cost is an important factor. This obviously
  can include hardware and software costs, but it is also important to
  consider other facets needed to deploy and maintain the
  system. The amount of developer time the system takes to
  build, the amount of operational effort required to run the system,
  and even the amount of training required should all be
  considered. Cost is the total cost of ownership.

\end{aosadescription}

Each of these principles provides the basis for decisions 
in designing a distributed web architecture. However, they also can
be at odds with one another, such that achieving one objective comes
at the cost of another. A basic example: choosing to address
capacity by simply adding more servers (scalability) can come at the
price of manageability (you have to operate an additional server) and
cost (the price of the servers).

When designing any sort of web application it is
important to consider these key principles, even if it is to
acknowledge that a design may sacrifice one or more of them.

\end{aosasect1}

\begin{aosasect1}{The Basics}

When it comes to system architecture there are a few things to
consider: what are the right pieces, how these pieces fit together,
and what are the right tradeoffs. 
Investing in scaling before it is needed is generally not a smart
business proposition; however, some forethought into the design can
save substantial time and resources in the future.

This section is focused on some of the core factors that are central to
almost all large web applications: \emph{services},
\emph{redundancy}, \emph{partitions}, and \emph{handling
failure}. Each of these factors involves choices and compromises,
particularly in the context of the principles described in the
previous section. In order to explain these in detail it is
best to start with an example.

\begin{aosasect2}{Example: Image Hosting Application}

At some point you have probably posted an image online. For big
sites that host and deliver lots of images, there are 
challenges in building an architecture that is cost-effective, highly
available, and has low latency (fast retrieval).

Imagine a system where users are able to upload their images to a
central server, and the images can be requested via a web link or
API, just like Flickr or Picasa. For the sake of simplicity, let's
assume that this application has two key parts: the ability to upload
(write) an image to the server, and the ability to query for an
image. While we certainly want the upload to be efficient, we care 
most about having very fast delivery when someone requests an image
(for example, images could be requested for a web page or other
application). This is very similar functionality to what a web server
or Content Delivery Network (CDN) edge server (a server
CDN uses to store content in many locations so content 
is geographically/physically closer to users, resulting in faster
performance) might provide.

Other important aspects of the system are:

\begin{aosaitemize}

\item There is no limit to the number of images that will be
  stored, so storage scalability, in terms of image count needs to be
  considered.

\item There needs to be low latency for image downloads/requests.

\item If a user uploads an image, the image should always be there
  (data reliability for images).

\item The system should be easy to maintain (manageability).

\item Since image hosting doesn't have high profit margins, the system
  needs to be cost-effective

\end{aosaitemize}

\aosafigref{fig.distsys.1} is a simplified diagram of the functionality.

\aosafigure{../images/distsys/imageHosting1.jpg}{Simplified architecture diagram for image hosting application}{fig.distsys.1}

In this image hosting example, the system must be perceivably fast,
its data stored reliably and all of these attributes highly
scalable. Building a small version of this application would be
trivial and easily hosted on a single server; however, that would not be
interesting for this chapter. Let's assume that we want to build
something that could grow as big as Flickr.

\end{aosasect2}

\begin{aosasect2}{Services}

When considering scalable system design, it helps to decouple
functionality and think about each part of the system as its own
service with a clearly-defined interface. In practice, systems
designed in this way are said to have a Service-Oriented Architecture
(SOA). For these types of systems, each service has its own distinct
functional context, and interaction with anything outside of that
context takes place through an abstract interface, typically the
public-facing API of another service.

Deconstructing a system into a set of complementary services decouples
the operation of those pieces from one another. This abstraction helps
establish clear relationships between the service, its underlying
environment, and the consumers of that service. Creating these
clear delineations can help isolate problems, but also allows each
piece to scale independently of one another. This sort of
service-oriented design for systems is very similar to object-oriented
design for programming.

In our example, all requests to upload and retrieve images are
processed by the same server; however, as the system needs to
scale it makes sense to break out these two functions into
their own services.

Fast-forward and assume that the service is in heavy use; such a
scenario makes it easy to see how longer writes will impact the time it takes
to read the images (since they two functions will be competing for
shared resources). Depending on the architecture this effect can be
substantial. Even if the upload and download speeds are the same
(which is not true of most IP networks, since most are designed for at
least a 3:1 download-speed:upload-speed ratio), read files will typically be read
from cache, and writes will have to go to disk eventually (and perhaps
be written several times in eventually consistent situations).
Even if everything is in memory or read from disks (like SSDs),
database writes will almost always be slower than reads\footnote{Pole
  Position, an open source tool for DB benchmarking,
  \url{http://polepos.org/} and results
  \url{http://polepos.sourceforge.net/results/PolePositionClientServer.pdf}.}.

Another potential problem with this design is that a web server like
Apache or lighttpd typically has an upper limit on the number of
simultaneous connections it can maintain
(defaults are around 500, but can go much higher) and in
high traffic, writes can quickly consume all of those. Since reads can
be asynchronous, or take advantage of other performance optimizations
like gzip compression or chunked transfer encoding, the web server can
switch serve reads faster and switch between clients quickly serving
many more requests per second than the max number of connections (with
Apache and max connections set to 500, it is not uncommon to serve
several thousand read requests per second). Writes, on the other hand,
tend to maintain an open connection for the duration for the upload,
so uploading a 1MB file could take more than 1 second on most home networks,
so that web server could only handle 500 such simultaneous
writes.

Planning for this sort of bottleneck makes a good case
to split out reads and writes of images into their own
services, shown in \aosafigref{fig.distsys.2}. This allows us to scale each of them independently (since it
is likely we will always do more reading than writing), but also helps
clarify what is going on at each point. Finally, this separates future
concerns, which would make it easier to troubleshoot and scale a
problem like slow reads.

\aosafigure{../images/distsys/imageHosting2.png}{Splitting out reads and writes}{fig.distsys.2}

The advantage of this approach is that we are able to solve 
problems independently of one another---we don't have to worry about
writing and retrieving new images in the same context. Both of
these services still leverage the global corpus of images, but they
are free to optimize their own performance with service-appropriate
methods
(for example, queuing up requests, or caching
popular images---more on this below). And from a maintenance and cost
perspective each service can scale independently as needed, which is
great because if they were combined and intermingled, one could
inadvertently impact the performance of the other as in the scenario
discussed above.

Of course, the above example can work well when you have two different
endpoints (in fact this is very similar to several cloud storage
providers' implementations and Content Delivery Networks). There are
lots of ways to address these types of bottlenecks though, and each
has different tradeoffs.

For example, Flickr solves this read/write issue by federating users
%% FIXME make sure "federating" is the right word here - ARB
across different shards such that each shard can only handle a set
number of users, and as users increase more shards are added to the
cluster\footnote{Presentation on Flickr's scaling,
  http://mysqldba.blogspot.com/2008/04/mysql-uc-2007-presentation-file.html}. In
the first example it is easier to scale hardware based on actual usage
(the number of reads and writes across the whole system), whereas
Flickr scales with their user base (but forces the assumption of equal
usage across users so there can be extra capacity). In the former an
outage or issue with one of the services brings down functionality
across the whole system (no-one can write files, for example), whereas
an outage with one of Flickr's shards will only affect those users. In
the first example it is easier to perform operations across the whole
dataset---for example, updating the write service to include new
metadata or searching across all image metadata---whereas with the
Flickr architecture each shard would need to be updated or searched
(or a search service would need to be created to collate that
metadata---which is in fact what they do).

When it comes to these systems there is no right answer, but it helps
to go back to the principles at the start of this chapter, determine
the system needs (heavy reads or writes or both, level of concurrency,
queries across the data set, ranges, sorts, etc.), benchmark different
alternatives, understand how the system will fail, and have a solid plan
for when failure happens.

\end{aosasect2}

\begin{aosasect2}{Redundancy}

In order to handle failure gracefully a web architecture must have
redundancy of its services and data. For example, if there is only one
copy of a file stored on a single server, then losing that server 
means losing that file. Losing data is seldom a good thing, and a
common way of handling it is to create multiple, or redundant, copies.

This same principle also applies to services. If there is a core piece
of functionality for an application, ensuring that multiple copies or
versions are running simultaneously can secure against the failure of
a single node.

Creating redundancy in a system can remove single points of failure
and provide a backup or spare functionality if needed in a crisis. For
example, if there are two instances of the same service running in
production, and one fails or degrades, the system can \emph{failover}
to the healthy copy. Failover can happen
automatically or require manual intervention.

Another key part of service redundancy is creating a \emph{shared-nothing 
architecture}. With this architecture, each node is able to
operate independently of one another and there is no central ``brain''
managing state or coordinating activities for the other nodes. This
helps a lot with scalability since new nodes can be added without
special conditions or knowledge. However, and most importantly, there
is no single point of failure in these systems, so they are much more
resilient to failure.

For example, in our image server application, all images would have
redundant copies on another piece of hardware somewhere (ideally in a
different geographic location in the event of a catastrophe like an
earthquake or fire in the data center), and the services to access the
images would be redundant, all potentially servicing requests. (See 
\aosafigref{fig.distsys.3}.)
(Load balancers are a great way to make this possible, but there is
more on that below).

\aosafigure{../images/distsys/imageHosting3.png}{Image hosting application with redundancy}{fig.distsys.3}

\end{aosasect2}

\begin{aosasect2}{Partitions}

There may be very large data sets that are unable to fit on a single
server. It may also be the case that an operation requires too many
computing resources, diminishing performance and making it
necessary to add capacity. In either case you have two choices: scale
vertically or horizontally.

Scaling vertically means adding more resources to an individual
server. So for a very large data set, this might mean adding more (or
bigger) hard drives so a single server can contain the entire data set. In
the case of the compute operation, this could mean moving the
computation to a bigger server with a faster CPU or more memory. In
each case, vertical scaling is accomplished by making the individual
resource capable of handling more on its own.

To scale horizontally, on the other hand, is to add more nodes. In the
case of the large data set, this might be a second server to store 
parts of the data set, and for the computing resource it
would mean splitting the operation or load across some additional
nodes. To take full advantage of horizontal scaling, it should be
included as an intrinsic design principle of the system
architecture, otherwise it can be quite cumbersome to modify and
separate out the context to make this possible.

When it comes to horizontal scaling, one of the more common
techniques is to break up your services into partitions, or shards. 
The partitions can be federated such
that each logical set of functionality is separate; this could be
done by geographic boundaries, or by another criteria like non-paying versus
paying users. The advantage of these schemes is that they provide a service
or data store with added capacity.

In our image server example, it is possible that the single file
server used to store images could be replaced by multiple file
servers, each containing its own unique set of images. 
(See \aosafigref{fig.distsys.4}.) Such an
architecture would allow the system to fill each file server with
images, adding additional servers as the disks become full. The
design would require a naming scheme that tied an image's filename
to the server containing it. An image's name could be formed from a
consistent hashing scheme mapped across the servers. Or alternatively,
each image could be assigned an incremental ID, so that when a client
makes a request for an image, the image retrieval service only needs
to maintain the range of IDs that are mapped to each of the servers
(like an index).

\aosafigure{../images/distsys/imageHosting4.png}{Image hosting application
with redundancy and partitioning}{fig.distsys.4}

Of course there are challenges distributing data or functionality
across multiple servers. One of the key issues is \emph{data
locality}; in distributed systems the closer the data to the
operation or point of computation, the better the performance of the
system. Therefore it is potentially problematic to have data
spread across multiple servers, as any time it is needed it may not be
local, forcing the servers to perform a costly fetch of the required
information across the network.

Another potential issue comes in the form of
\emph{inconsistency}. When there are different services reading and
writing from a shared resource, potentially another service or data
store, there is the chance for race conditions---where some data is 
supposed to be updated, but the read happens prior to the update---and
in those cases the data is inconsistent. For example, in the image
hosting scenario, a race condition could occur if one client sent a
request to update the dog image with a new title, changing it from
``Dog'' to ``Gizmo'', but at the same time another client was reading
the image. In that circumstance it is unclear which title, ``Dog'' or
``Gizmo'', would be the one received by the second client.

There are certainly some obstacles associated with partitioning data,
but partitioning allows each problem to be split---by data, load, usage
patterns, etc.---into manageable chunks. This can help with scalability
and manageability, but is not without risk.  
There are lots of ways to mitigate risk and handle failures; however,
in the interest of brevity they are not covered in this chapter. If
you are interested in reading more, you can check out my blog post
on fault tolerance and monitoring\footnote{\url{http://katemats.com/2011/11/13/distributed-systems-basics-handling-failure-fault-tolerance-and-monitoring/}}.

\end{aosasect2}

\end{aosasect1}

\begin{aosasect1}{The Building Blocks of Fast and Scalable Data Access}

Having covered some of the core considerations in designing
distributed systems, let's now talk about the hard part: scaling
access to the data.

Most simple web applications, for example, LAMP stack applications,
look something like \aosafigref{fig.distsys.5}.

\aosafigure{../images/distsys/simpleWeb.png}{Simple web applications}{fig.distsys.5}

As they grow, there are two main challenges: scaling access to the
app server and to the database. In a highly scalable application
design, the app (or web) server is typically minimized and often
embodies a shared-nothing architecture. This makes the app server
layer of the system horizontally scalable. As a result of this design,
the heavy lifting is pushed down the stack to the database server and
supporting services; it's at this layer where the real scaling and
performance challenges come into play.

The rest of this chapter is devoted to some of the more common
strategies and methods for making these types of services fast and
scalable by providing fast access to data.

Most systems can be oversimplified to \aosafigref{fig.distsys.6}.

\aosafigure{../images/distsys/overSimpleWeb.png}{Oversimplified web application}{fig.distsys.6}

This is a great place to start. If you have a lot of data, you want
fast and easy access, like keeping a stash of candy in the top drawer
of your desk. Though overly simplified, the previous statement hints
at two hard problems: scalability of storage and fast access of data.

For the sake of this section, let's assume you have many terabytes (TB)
of data and you want to allow users to access small portions 
of that data at random. (See \aosafigref{fig.distsys.7}.) 
This is similar to locating an image file
somewhere on the file server in the image application example.

\aosafigure{../images/distsys/accessingData.png}{Accessing specific data}{fig.distsys.7}

This is particularly challenging because it can be very costly to load
TBs of data into memory; this directly translates to disk IO. Reading
from disk is many times slower than from memory---memory access is
as fast as Chuck Norris, whereas disk access is slower than the
line at the DMV. This speed difference really adds up for large
data sets; in real numbers memory access is as little as 6 times
faster for sequential reads, or 100,000 times faster for random
reads\footnote{The Pathologies of Big Data,
  \url{http://queue.acm.org/detail.cfm?id=1563874}.}, than reading from
disk. Moreover, even with unique IDs, solving the problem of
knowing where to find that little bit of data can be an arduous
task. It's like 
trying to get that last Jolly Rancher from your candy stash without
looking.

Thankfully there are many options that you can employ to make this
easier; four of the more important ones are caches, proxies,
indexes and load balancers. The rest of this section 
discusses how each of these concepts can be used to make data
access a lot faster.

\begin{aosasect2}{Caches}

Caches take advantage of the locality of reference
principle: recently requested data is likely to be requested
again. They are used in almost every layer of computing:
hardware, operating systems, web browsers, web applications and
more. A cache is like short-term memory: it has a limited amount of
space, but is typically faster than the original data source and
contains the most recently accessed items. Caches can exist at all
levels in architecture, but are often found at the level nearest
to the front end, where they are implemented to return data
quickly without taxing downstream levels.

How can a cache be used to make your data access faster in our API example?  
In this case, there are a couple of places you can insert a cache.  One 
option is to insert a cache on your request layer node, as in
\aosafigref{fig.distsys.8}.

\aosafigure{../images/distsys/cache.png}{Inserting a cache on your request layer node}{fig.distsys.8}

Placing a cache directly on a request layer node enables the local
storage of response data. Each time a request is made to the service,
the node will quickly return local, cached data if it exists. If it
is not in the cache, the request node will query the data from disk. The
cache on one request layer node could also be located both in memory
(which is very fast) and on the node's local disk (faster than going
to network storage).


\aosafigure{../images/distsys/multipleCaches.png}{Multiple caches}{fig.distsys.9}

What happens when you expand this to many nodes?
As you can see in \aosafigref{fig.distsys.9}, if the request layer is expanded to multiple nodes, it's still quite
possible to have each node host its own cache. However, if your load
balancer randomly distributes requests across the nodes, the same
request will go to different nodes, thus increasing cache misses. Two
choices for overcoming this hurdle are global caches and distributed
caches. Read on to learn more!

\end{aosasect2}

\begin{aosasect2}{Global Cache}

A global cache is just as it sounds: all the nodes use the same single cache
space. This involves adding a server, or file store of some sort,
faster than your original store and accessible by all the request
layer nodes. Each of the request nodes queries the cache in the same
way it would a local one. This kind of caching scheme can get a bit
complicated because it is very easy to overwhelm a single cache as the
number of clients and requests increase, but is very effective in some
architectures (particularly ones with specialized hardware that make
this global cache very fast, or that have a fixed dataset that needs to be
cached).

\aosafigure{../images/distsys/globalCache1.png}{Global cache where cache is responsible for retrieval}{fig.distsys.10}

\aosafigure{../images/distsys/globalCache2.png}{Global cache where request nodes are responsible for retrieval}{fig.distsys.11}

There are two common forms of global caches depicted in the
diagrams. In \aosafigref{fig.distsys.10}, when a cached response is not found in
the cache, the cache itself becomes responsible for retrieving the
missing piece of data from the underlying store. In \aosafigref{fig.distsys.11}
it is the responsibility of request nodes to retrieve any data that is
not found in the cache.

The majority of applications leveraging global caches tend to use the
first type, where the cache itself manages eviction and fetching data to
prevent a flood of requests for the same data from the
clients. However, there are some cases where the second
implementation makes more sense. For example, if the cache is being
used for very large files, a low cache hit percentage would cause the
cache buffer to become overwhelmed with cache misses; in this
situation it helps to have a large percentage of the total data set
(or hot data set) in the cache. Another example is an architecture where
the files stored in the cache are static and shouldn't be evicted.
(This could be because of application requirements around that data 
latency---certain pieces of data might need to be very fast for large
data sets---where the application logic understands the eviction
strategy or hot spots better than the cache.)

\end{aosasect2}

\begin{aosasect2}{Distributed Cache}

In a distributed cache (\aosafigref{fig.distsys.12}), each of its nodes 
own part of the cached data,
so if a refrigerator acts as a cache to the grocery store, a
distributed cache is like putting your food in several locations---your
fridge, cupboards, \emph{and} lunch box---convenient locations for
retrieving snacks from, without a trip to the store. Typically the cache is divided up
using a consistent hashing function, such that if a request node is
looking for a certain piece of data it can quickly know where to look
within the distributed cache to determine if that data is
available. In this case, each node has a small piece of the cache, and
will then send a request to another node for the data before going to
the origin. Therefore, one of the advantages of a distributed cache is
the increased cache space that can be had just by adding 
nodes to the request pool.

A disadvantage of distributed caching is remedying a missing
node. Some distributed caches get around this by storing multiple
copies of the data on different nodes; however, you can imagine how
this logic can get complicated quickly, especially when you add or
remove nodes from the request layer. Although even if a node
disappears and part of the cache is lost, the requests will just pull
from the origin---so it isn't necessarily catastrophic!

\aosafigure{../images/distsys/distributedCaching.png}{Distributed cache}{fig.distsys.12}

The great thing about caches is that they usually make things much
faster (implemented correctly, of course!) The methodology you choose
just allows you to make it faster for even more requests. However, all
this caching comes at the cost of having to maintain additional
storage space, typically in the form of expensive memory; nothing is
free. Caches are wonderful for making things generally faster, and
moreover provide system functionality under high load conditions when
otherwise there would be complete service degradation.

One example of a popular open source cache is
Memcached\footnote{\url{http://memcached.org/}} (which can work both
as a local cache and distributed cache); however, there are many other
options (including many language- or framework-specific options).

Memcached is used in many large web sites, and even though it can be
very powerful, it is simply an in-memory key value store, optimized
for arbitrary data storage and fast lookups (\code{O(1)}).

Facebook uses several different types of caching to
obtain their site performance\footnote{Facebook caching and
performance, \url{http://sizzo.org/talks/}.}. They use \$GLOBALS and
APC caching at the language level (provided in PHP at the cost of a function call) which helps make intermediate
function calls and results much faster. (Most languages have these
types of libraries to improve web page performance and they should almost
always be used.) Facebook then use a global cache that is
distributed across many servers\footnote{Scaling memcached at
  Facebook,
  \url{http://www.facebook.com/note.php?note_id=39391378919}.}, such
that one function call accessing the cache could make many requests in
parallel for data stored on different Memcached servers. This allows
them to get much higher performance and throughput for their user
profile data, and have one central place to update data (which is
important, since cache invalidation and maintaining consistency can be
challenging when you are running thousands of servers).

Now let's talk about what to do when the data isn't in the cache{\ldots}

\end{aosasect2}

\begin{aosasect2}{Proxies}

At a basic level, a proxy server is an intermediate piece of
hardware/software that receives requests from clients and relays them
to the back end origin servers. Typically, proxies are used to filter
requests, log requests, or sometimes transform requests (by
adding/removing headers, encrypting/decrypting, or compression).

\aosafigure{../images/distsys/proxies.png}{Proxy server}{fig.distsys.13}

Proxies are also immensely helpful when coordinating requests from
multiple servers, providing opportunities to optimize request traffic
from a system-wide perspective. One way to use a proxy to speed up
data access is to collapse the same (or similar) requests together
into one request, and then return the single result to the requesting
clients. This is known as collapsed forwarding.

Imagine there is a request for the same data (let's call it littleB)
across several nodes, and that piece of data is not in the cache. If
that request is routed thought the proxy, then all of those requests
can be collapsed into one, which means we only have to read littleB
off disk once. (See \aosafigref{fig.distsys.14}.) There is some cost associated with this design, since
each request can have slightly higher latency, and some requests may
be slightly delayed to be grouped with similar ones. But it will
improve performance in high load situations, particularly when that
same data is requested over and over. This is similar to a cache, but
instead of storing the data/document like a cache, it is optimizing
the requests or calls for those documents and acting as a proxy for
those clients. 

In a LAN proxy, for example, the clients do not need
their own IPs to connect to the Internet, and the LAN will collapse
calls from the clients for the same content. It is easy to get
confused here though, since many proxies are also caches (as it is a
very logical place to put a cache), but not all caches act as proxies.

\aosafigure{../images/distsys/collapseRequests.png}{Using a proxy server to collapse requests}{fig.distsys.14}

Another great way to use the proxy is to not just collapse requests
for the same data, but also to collapse requests for data that is
spatially close together in the origin store (consecutively on disk).
Employing such a strategy maximizes data locality for the requests,
which can result in decreased request latency. For example, let's say
a bunch of nodes request parts of B: partB1, partB2, etc. We can
set up our proxy to recognize the spatial locality of the individual
requests, collapsing them into a single request and returning only
bigB, greatly minimizing the reads from the data origin. (See \aosafigref{fig.distsys.15}.) This can make
a really big difference in request time when you are randomly
accessing across TBs of data! Proxies are especially helpful under
high load situations, or when you have limited caching, since they
can essentially batch several requests into one.

\aosafigure{../images/distsys/collapseRequestsSpatial.png}{FIXME}{fig.distsys.15}

It is worth noting that you can use proxies and caches together, but
generally it is best to put the cache in front of the proxy,
for the same reason that it is best to let the faster runners start first in a
crowded marathon race. This is because the cache is serving data from
memory, it is very fast, and it doesn't mind multiple requests for the
same result. But if the cache was located on the other side of the
proxy server, then there would be additional latency with every
request before the cache, and this could hinder performance.

If you are looking at adding a proxy to your systems, there are many
options to
consider\footnote{\url{http://en.wikipedia.org/wiki/Web_accelerator}};
Squid\footnote{\url{http://www.squid-cache.org/}} and
Varnish\footnote{\url{https://www.varnish-cache.org/}} have both been
road tested and are widely used in many production web sites. These
proxy solutions offer many optimizations to make the most of
client-server communication. Installing one of these as a reverse
proxy (explained in the load balancer section below) at the web server
layer can improve web server performance considerably, reducing the
amount of work required to handle incoming client requests.

\end{aosasect2}

\begin{aosasect2}{Indexes}

Using an index to access your data quickly is a well-known strategy
for optimizing data access performance; probably the most well known 
when it comes to databases. An index makes the trade-offs of
increased storage overhead and slower writes (since you must both
write the data and update the index) for the benefit of faster reads.

Just as to a traditional relational data store, you can also apply
this concept to larger data sets. The trick with indexes is you must
carefully consider how users will access your data. In the case of
data sets that are many TBs in size, but with very small payloads
(e.g., 1 KB), indexes are a necessity for optimizing data
access. Finding a small payload in such a large data set can be a real
challenge since you can't possibly iterate over that much data in any
reasonable time. Furthermore, it is very likely that such a large data
set is spread over several (or many!) physical devices---this means
you need some way to find the correct physical location of the desired
data. Indexes are the best way to do this. 

An index can be used like a table of contents that directs you to the
location where your data lives. 
For example, let's say you are looking for a
piece of data, part 2 of B---how will you know where to find it? If
you have an index that is sorted by data type---say data A, B, C---it
would tell you the location of data B at the origin. Then you just
have to seek to that location and read the part of B you want. 
(See \aosafigref{fig.distsys.16}.)

\aosafigure{../images/distsys/indexes.jpg}{Indexes}{fig.distsys.16}

These indexes are often stored in memory, or somewhere very local to
the incoming client request. Berkeley DBs (BDBs) and tree-like
data structures are commonly used to store data in ordered lists,
ideal for access with an index.

Often there are many layers of indexes that serve as a
map, moving you from one location to the next, and so forth, until
you get the specific piece of data you want. (See \aosafigref{fig.distsys.17}.)

\aosafigure{../images/distsys/multipleIndexes.jpg}{Many layers of indexes}{fig.distsys.17}

Indexes can also be used to create several different views of the same
data. For large data sets, this is a great way to define different
filters and sorts without resorting to creating many additional copies
of the data.

For example, imagine that the image hosting system from earlier is
actually hosting images of book pages, and the service allows client
queries across the text in those images, searching all the book
content about a topic, in the same way search engines allow you to
search HTML content. In this case, all those book images take many,
many servers to store the files, and finding one page to render to the
user can be a bit involved. First, inverse indexes to
query for arbitrary words and word tuples need to be easily
accessible; then there is the challenge of navigating to the exact
page and location within that book, and retrieving the right image for
the results. So in this case the inverted index would map to a
location (such as book B), and then B may contain an index with all
the words, locations and number of occurrences in each part.

An inverted index, which could represent Index1 in the diagram above,
might look something like the following---each word or tuple of words
provide an index of what books contain them.

\begin{tabular}{|l|l|}
\hline
Word(s) & Book(s) \\
\hline
being awesome & Book B, Book C, Book D \\
always & Book C, Book F \\
believe & Book B \\
\hline
\end{tabular}

The intermediate index would look similar but would contain just the
words, location, and information for book B. This nested index
architecture allows each of these indexes to take up less space than
if all of that info had to be stored into one big inverted index.

And this is key in large-scale systems because even compressed, these
indexes can get quite big and expensive to store. In this system if we
assume we have a lot of the books in the 
world---100,000,000\footnote{Inside Google Books blog post,
  \url{http://booksearch.blogspot.com/2010/08/books-of-world-stand-up-and-be-counted.html}.}---and that each book is only 10 pages long (to make the 
math easier),
with 250 words per page, that means there are 250 billion words. If
we assume an average of 5 characters per word, and each character takes 8 bits (or
1 byte, even though some characters are 2 bytes), so 5 bytes per word,
then an index containing only each word once is over a terabyte of storage.
So you can see creating indexes that have a lot
of other information like tuples of words, locations for the data,
and counts of occurrences, can add up very quickly.

Creating these intermediate indexes and representing the data in
smaller sections makes big data problems tractable. Data can
be spread across many servers and still accessed quickly. Indexes are
a cornerstone of information retrieval, and the basis for today's
modern search engines. Of course, this section only scratched the
surface, and there is a lot of research being done on how to make
indexes smaller, faster, contain more
information (like relevancy), and update seamlessly. (There are some
manageability challenges with race conditions, and with the sheer number of
updates required to add new data or change existing
data, particularly in the event where relevancy or scoring is
involved).

Being able to find your data quickly and easily is important; indexes
are an effective and simple tool to achieve this.

\end{aosasect2}

\begin{aosasect2}{Load Balancers}

Finally, another critical piece of any distributed system is a load
balancer. Load balancers are a principal part of any architecture, as their
role is to distribute load across a set of nodes responsible for
servicing requests. This allows multiple nodes to transparently
service the same function in a system. (See \aosafigref{fig.distsys.18}.) Their main purpose is to handle
a lot of simultaneous connections and route those connections to one
of the request nodes, allowing the system to scale to service more
requests by just adding nodes.

\aosafigure{../images/distsys/loadBalancer.png}{Load balancer}{fig.distsys.18}

There are many different algorithms that can be used to service
requests, including picking a random node, round robin\footnote{\url{http://en.wikipedia.org/wiki/Round-robin}}, or even selecting the node
based on certain criteria, such as memory or CPU utilization. Load balancers can
be implemented as software or hardware appliances. One open source
software load balancer that has received wide adoption is
HAProxy\footnote{\url{http://haproxy.1wt.eu/}}.

In a distributed system, load balancers are often found at the very
front of the system, such that all incoming requests are routed
accordingly. In a complex distributed system, it is not uncommon for a
request to be routed to multiple load balancers as shown in 
\aosafigref{fig.distsys.19}.

\aosafigure{../images/distsys/multipleLoadBalancers.png}{Multiple load balancers}{fig.distsys.19}

Like proxies, some load balancers can also route a request
differently depending on the type of request it is. (Technically these are
also known as reverse proxies.)

One of the challenges with load balancers is managing user-session-specific 
data. In an e-commerce site, when you only have one client it
is very easy to allow users to put things in their shopping cart and
persist those contents between visits (which is important, because it
is much more likely you will sell the product if it is still in the
user's cart when they return). However, if a user is routed to one
node for a session, and then a different node on their next visit,
there can be inconsistencies since the new node may be missing that
user's cart contents. (Wouldn't you be upset if you put a 6 pack of
Mountain Dew in your cart and then came back and it was empty?) One way
around this can be to make sessions sticky so that the user is always
routed to the same node, but then it is very hard to take advantage of
some reliability features like automatic failover. In this case, the
user's shopping cart would always have the contents, but if their
sticky node became unavailable there would need to be a special case
and the assumption of the contents being there would no longer be
valid (although hopefully this assumption wouldn't be built into the
application). Of course, this problem can be solved using other
strategies and tools in this chapter, like services, and many not
covered (like browser caches, cookies, and URL rewriting).

If a system only has a couple of a nodes, systems like round robin DNS
may make more sense since load balancers can be expensive and add an
unneeded layer of complexity. Of course in larger systems there are
all sorts of different scheduling and load-balancing algorithms,
including simple ones like random choice or round robin, and more
sophisticated mechanisms that take things like utilization and
capacity into consideration. All of these algorithms allow traffic and
requests to be distributed, and can provide helpful reliability tools
like automatic failover, or automatic removal of a bad node (such as
when it becomes unresponsive). However, these advanced features can
make problem diagnosis cumbersome. For example, when it comes to high
load situations, load balancers will remove nodes that may be slow or
timing out (because of too many requests), but that only exacerbates
the situation for the other nodes. In these cases extensive monitoring
is important, because overall system traffic and throughput may look
like it is decreasing (since the nodes are serving less requests) but
the individual nodes are becoming maxed out.

Load balancers are an easy way to allow you to expand system capacity, and like
the other techniques in this article, play an essential role in
distributed system architecture. Load balancers also provide the
critical function of being able to test the health of a node, such
that if a node is unresponsive or over-loaded, it can be removed from
the pool handling requests, taking advantage of the redundancy of
different nodes in your system.

\end{aosasect2}

\begin{aosasect2}{Queues}

So far we have covered a lot of ways to read data quickly, but
another important part of scaling the data layer is effective
management of writes. When systems are simple, with minimal
processing loads and small databases, writes can be predictably fast;
however, in more complex systems writes can take an almost
non-deterministically long time. For example, data may have to be
written several places on different servers or indexes, or the system
could just be under high load. In the cases where writes, or any task
for that matter, may take a long time, achieving performance and
availability requires building asynchrony into the system; a
common way to do that is with queues.

Imagine a system where each client is requesting a task to be remotely
serviced. Each of these clients sends their request to the server,
where the server completes the tasks as quickly as possible and
returns the results to their respective clients. In small systems
where one server (or logical service) can service incoming clients
just as fast as they come, this sort of situation should work just
fine. However, when the server receives more requests than it can
handle, then each client is forced to wait for the other clients'
requests to complete before a response can be generated. This is an
example of a synchronous request, depicted in \aosafigref{fig.distsys.20}.

\aosafigure{../images/distsys/synchronousRequest.png}{Synchronous request}{fig.distsys.20}

This kind of synchronous behavior can severely degrade client
performance; the client is forced to wait, effectively performing zero
work, until its request can be answered. Adding additional servers to
address system load does not solve the problem either; even with
effective load balancing in place it is extremely difficult to ensure
the even and fair distribution of work required to maximize client
performance. Further, if the server handling requests is unavailable,
or fails, then the clients upstream will also fail. Solving this
problem effectively requires abstraction between the client's request
and the actual work performed to service it.

Enter queues. A queue is as simple as it sounds: a task comes in, is
added to the queue and then workers pick up the next task as they have
the capacity to process it. (See \aosafigref{fig.distsys.21}.) These tasks 
could represent simple writes to a
database, or something as complex as generating a thumbnail preview
image for a document. When a client submits task requests to a queue
they are no longer forced to wait for the results; instead they need
only acknowledgement that the request was properly received. This
acknowledgement can later serve as a reference for the results of the
work when the client requires it.

Queues enable clients to work in an asynchronous manner, providing a
strategic abstraction of a client's request and its response. On the
other hand, in a synchronous system, there is no differentiation
between request and reply, and they therefore cannot be managed
separately. In an asynchronous system the client requests a task, the
service responds with a message acknowledging the task was received,
and then the client can periodically check the status of the task,
only requesting the result once it has completed. While the client is
waiting for an asynchronous request to be completed it is free to
perform other work, even making asynchronous requests of other
services. The latter is an example of how queues and messages are
leveraged in distributed systems.

Queues also provide some protection from service outages and
failures. For instance, it is quite easy to create a highly robust
queue that can retry service requests that have failed due to transient
server failures. It is more preferable to use a queue to enforce
quality-of-service guarantees than to expose clients directly to
intermittent service outages, requiring complicated and
often-inconsistent client-side error handling.

%% Maybe move this figure? It could go anywhere in the previous few
%% paragraphs. -ARB
\aosafigure{../images/distsys/queues.png}{Using queues to manage requests}{fig.distsys.21}

Queues are fundamental in managing distributed communication between
different parts of any large-scale distributed system, and there are
lots of ways to implement them. There are quite a few open source
queues like RabbitMQ\footnote{\url{http://www.rabbitmq.com/}},
ActiveMQ\footnote{\url{http://activemq.apache.org/}},
BeanstalkD\footnote{\url{http://kr.github.com/beanstalkd/}}, but some
also use services like
Zookeeper\footnote{\url{http://zookeeper.apache.org/}}, or even data
stores like Redis\footnote{\url{http://redis.io/}}.

\end{aosasect2}

\end{aosasect1}

\begin{aosasect1}{Conclusion}

Designing efficient systems with fast access to lots of data is
exciting, and there are lots of great tools that enable all kinds of
new applications. This chapter covered just a few examples, barely
scratching the surface, but there are many more---and there will only
continue to be more innovation in the space.

\end{aosasect1}

\end{aosachapter}
