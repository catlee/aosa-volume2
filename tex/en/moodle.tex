\begin{aosachapter}{Moodle}{s:moodle}{Tim Hunt}

Moodle is a web application used in educational settings. While I will
try to give an overview of all aspects of how Moodle works, I will
focus on those areas where I think Moodle's design is particularly
interesting:

\begin{aosaenumerate}

\item the way the application is divided into plugins;

\item the permission system, which controls which users can perform
  which actions in different parts of the system;

\item the way output is generated, so that different themes (skins)
  can be used to give different appearances, and so that the interface
  can be localised; and

\item the database abstraction layer.

\end{aosaenumerate}

Moodle\footnote{\url{http://moodle.org/}} provides a place online
where students and teachers can come together to teach and learn. A
Moodle site is divided into Courses. A course will have certain users
enrolled in it in different roles, such as Student or Teacher. Each
course will comprise a number of resource and activities. A resource
might be a PDF file, a page of HTML within Moodle, or a link to
something else on the web. An activity might be a forum, a quiz or a
wiki. Within the course, these resources and activities will be
structured in some way. For example they may be grouped into logical
topics, or into weeks on a calendar.

\aosafigure{../images/moodle/course.png}{Moodle Course}{fig.moodle.course}

Moodle can be used as a stand-alone application. If you wanted to
teach software architecture (for example) you could download Moodle to
your web hosting, install it, start creating courses, and wait for
students to come and self-register. Alternatively, if you are a large
institution, Moodle would be just one of the systems you run. You
would probably also have the infrastructure shown in
\aosafigref{fig.moodle.uni}:

\aosafigure{../images/moodle/university.png}{University Infrastructure}{fig.moodle.uni}

\begin{aosaitemize}

\item an authentication/identity provider (for example LDAP) to
  control user accounts across all your systems.

\item a student information system. That is, a database of all your
  students, which program of study they are on, and hence which
  courses they need to complete; and their transcript, that is a high
  level summary of the results of the courses they have
  completed. This would also deal with other administrative functions
  like tracking whether they have paid their fees.

\item a document repository (for example Alfresco). To store files,
  and track the workflow as users collaborate to create files.

\item an ePortfolio. This is a place where students can put assets and
  assemble them. This might be in order to build a CV (resume), or to
  assemble evidence that they have met the requirements of a
  practice-based course.

\item a reporting or analytics tools. To generate high level
  information about what is going on in your institution.

\end{aosaitemize}

Moodle focusses on providing a space online where people can engage in
teaching and learning, rather than any of the other systems that an
educational organisation might need. Moodle provides a basic
implementation of the other functionality, so that it can function as
a stand-alone system, or will integrate with other systems. The role
Moodle plays is normally called a virtual learning environment (VLE),
or learning or course management system (LMS, CMS or even LCMS).

Moodle is open source or free software (GPL). It is written in PHP. It
will run on most common web servers, on common platforms. It requires
a database, and will work with MySQL, PostgreSQL, MS SQL Server or
Oracle.

The Moodle project was started by Martin Dougiamas in 1999, while he
was working at Curtin University, Australia. Version 1.0 was released
in 2002, at which time PHP4.2 and MySQL 3.23 were the technologies
available. That limited the kind of architecture that was possible
initially, but much has changed since then. The current release is the
Moodle 2.2.x series.

\begin{aosasect1}{An Overview of How Moodle Works}

A Moodle installation comprises three parts:

\begin{aosaenumerate}

\item The code, typically in a folder like \code{/var/www/moodle} or
  \code{{\textasciitilde}/htdocs/moodle}. This should not be writable
  by the web server.

\item The database, managed by one of the supported RDMSs. In fact,
  moodle adds a prefix to all the table names, so it can share a
  database with other applications, if desired.

\item The \code{moodledata} folder. This is a folder where Moodle
  stores uploaded and generated files, and so needs to be writable by
  the web server. For security reasons, the should be outside the web
  root.

\end{aosaenumerate}

These can all be on a single server. Alternatively, in a load-balanced
set-up, there will be multiple copies of the code on each web-server,
but just one shared copy of the database and moodledata, probably on
other servers.

The configuration information about these three parts is stored in a
file called config.php in the root of the \code{moodle} folder when
Moodle is installed.

\begin{aosasect2}{Request Dispatching}

Moodle is a web applications, so users interact with it using their
web browser. From Moodle's point of view that means responding to HTTP
requests. An important aspect of Moodle's design is, therefore, the
URL namespace, and how URLs get dispatched to different scripts.

Moodle uses the standard PHP approach to this. To view the main page
for a course, the URL would be \code{.../course/view.php?id=123},
where \code{123} is the unique id of the course in the database. To
view a forum discussion, the URL would be something like
\code{.../mod/forum/discuss.php?id=456789}. That is, these particular
scripts, \code{course/view.php} or \code{mod/forum/discuss.php} would
handle these requests. This is simple for the developer. To understand
how Moodle handles a particular request, you look at the URL and start
reading code there. It is ugly from the user's point of view. These
URLs are, however, permanent. The URLs do not change if the course is
renamed, or if a moderator moves a discussion to a different forum.

The alternative approach one could take is to have a single entry
point \code{.../index.php/\[extra information to make the request
  unique\]}. The single script \code{index.php} would then dispatch
the requests in some way. This approach adds a layer of indirection,
which is something software developers always like to do. The lack of
this layer of indirection does not seem to hurt Moodle.

\end{aosasect2}

\begin{aosasect2}{Plugins}

Like many successful open source projects, Moodle is built out of many
plugins, working together with the core of the system. This is a good
approach because at allows people to change an enhance Moodle in
defined ways. An important advantage of an open source system is that
you can tailor it to your particular needs. Making extensive
customisations to the code can, however, lead to big problems when the
time comes to upgrade, even when using a good version control
system. By allowing as many customisations and new features as
possible to be implemented as self-contained plugins that interact
with Moodle core through a defined API, it easier for people to
customise Moodle to their needs, and to share customisations, while
still being able to upgrade the core Moodle system.

There are various ways a system can be built as a core surrounded by
plugins. Moodle has a relatively fat core, and the plugins are
strongly-typed. When I say a fat core, I mean that there is a lot of
functionality in the core. This contrasts to the kind of architecture
where just about everything, except for a small plugin-loader stub, is
a plugin.

When I say plugins are strongly type, I mean that depending on which
type of functionality you want to implement, you have to write a
different type of plugin, and implement a different API. For example,
an new Activity module plugin would be very different from a new
Authentication plugin or a new Question type. At the last count there
are about 35 different types of plugin. This contrasts with the kind
of architecture where all plugins use basically the same API and then,
perhaps, subscript the the subset of hooks or events they are
interested in.

Generally, the trend in Moodle has been to try to shrink the core, by
moving more functionality into plugins. This is only somewhat
successful, however, because an increasing feature-set tends to expand
the core. The other trend has been to try to standardise the different
types of plugin as much as possible, so that in areas of common
functionality, like install and upgrade, all types of plugins work the
same way.

A plugin is Moodle takes the form of a folder containing files. The
plugin has a type and a name, which together make up the
``Frankenstyle'' component name of the plugin\footnote{The word
  ``Frankenstyle'' arouse out of an argument in the developers' Jabber
  channel, but everyone liked it and it stuck.}. The plugin type and
name determine the path to the plugin folder. The plugin type gives a
prefix, and then the foldername is the plugin name. Here are some
examples.

\begin{tabular}{llll}
Plugin type           & Plugin name & Frankenstyle      & Folder \\
mod (Activity module) & \code{forum}       & \code{mod\_forum}         & \code{mod/forum} \\
mod (Activity module) & \code{quiz}        & \code{mod\_quiz}          & \code{mod/quiz} \\
block (Side-block)    & \code{navigation}  & \code{block\_navigation}  & \code{blocks/navigation} \\
qtype (Question type) & \code{shortanswer} & \code{qtype\_shortanswer} & \code{question/type}/shortanswer \\
quiz (Quiz report)    & \code{statistics}  & \code{quiz\_statistics}   & \code{mod/quiz/report/statistics}
\end{tabular}

The last example shows that each activity module is allowed to declare
sub-plugin types. At the moment only activity modules can do this, for
two reasons. If all plugins could have sub-plugins that might cause
performance problems. Activity modules are the main educational
activities in Moodle, and so the most important type of plugin, so
they get special privilidges.

\end{aosasect2}

\begin{aosasect2}{An Example Plugin}

I will explain a lot of details of the Moodle architecture by
considering a specific example plugin. As is traditional, I have
chosen to implement a plugin that displays ``Hello world''.

This plugin does not really fit naturally into any of the standard
Moodle plugin types. It is just a script, with no connection to
anything else, so I will choose to implement it as a ``local''
plugin. This is a catch-all plugin type for miscellaneous
functionality that does not fit anywhere better. I will name my plugin
\code{greet}, to give a Frankensyle name of \code{local\_greet}, and a
folder path of \code{local/greet}. The plugin code can be downloaded
from \url{https://github.com/timhunt/moodle-local_greet}.

Each plugin must contain a file called \code{version.php} which
defines some basic metadata about the plugin. This is used by the
Moodle's plugin installer system to install and upgrade the plugin.
For example, \code{local/greet/version.php} contains:

\begin{verbatim}
<?php
$plugin->component    = 'local_greet';
$plugin->version      = 2011102900;
$plugin->requires     = 2011102700;
$plugin->maturity     = MATURITY_STABLE;
\end{verbatim}

It may seem redundant to include the component name, since this can be
deduced from the path, but the installer uses this to verify that the
plugin has been installed in the right place. The version field is the
version of this plugin. Maturity is ALPHA, BETA, RC (release
candidate), or STABLE. Requires is the minimum version of Moodle that
this plugin is compatible with. If necessary, one can also document
other plugins that this one depends on.

Now here is the main script for this simple plugin (stored in
\code{local/greet/index.php}):

\begin{verbatim}
<?php
require_once(dirname(__FILE__) . '/../../config.php');        // 1

require_login();                                              // 2
$context = context_system::instance();                        // 3
require_capability('local/greet:begreeted', $context);        // 4

$name = optional_param('name', '', PARAM_TEXT);               // 5
if (!$name) {
    $name = fullname($USER);                                  // 6
}

add_to_log(SITEID, 'local_greet', 'begreeted',
        'local/greet/index.php?name=' . urlencode($name));    // 7

$PAGE->set_context($context);                                 // 8
$PAGE->set_url(new moodle_url('/local/greet/index.php'),
        array('name' => $name));                              // 9
$PAGE->set_title(get_string('welcome', 'local_greet'));       // 10

echo $OUTPUT->header();                                       // 11
echo $OUTPUT->box(get_string('greet', 'local_greet',
        format_string($name)));                               // 12
echo $OUTPUT->footer();                                       // 13
\end{verbatim}

\end{aosasect2}

\begin{aosasect2}{Bootstrapping Moodle}

\begin{verbatim}
require_once(dirname(__FILE__) . '/../../config.php');        // 1
\end{verbatim}

The single line of this script that does the most work is the first. I
said above that \code{config.php} contains the details Moodle needs to
connect to the database and find the moodledata folder. It ends,
however, with the line \code{require\_once('lib/setup.php')}. This:

\begin{aosaenumerate}

\item loads all the standard Moodle libraries using
  \code{require\_once};

\item starts the session handling;

\item connects to the database; and

\item sets up a number of global variables, which we shall meet later.

\end{aosaenumerate}

\end{aosasect2}

\begin{aosasect2}{Checking the User Is Logged In}

\begin{verbatim}
require_login();                                              // 2
\end{verbatim}

This line causes Moodle to check that the current user is logged in,
using whatever authentication plugin the administrator has
configured. If not, the user will be redirected to log in, and this
function will never return.

If we were writing a script that was more integrated into Moodle, we
would pass more arguments here, to say which course or activity this
page is part of, and then \code{require\_login} would also verify that
the user is enrolled in, or otherwise allowed to access this course,
and is allowed to see this activity. If not, an appropriate error
would be displayed.

\end{aosasect2}

\end{aosasect1}

\begin{aosasect1}{Moodle's Roles and Permissions System}

The next two lines of code show how to check that the user has
permission to do something. As you can see, from the developer's point
of view, the API is very simple. Behind the scenes, however, there is
a sophisticated access system which gives the administrator great
flexibility to control who can do what.

\begin{aosasect2}{Getting the Context}

\begin{verbatim}
$context = context_system::instance();                        // 3
\end{verbatim}

In Moodle, users can have different permissions in different
places. For example, a user might be a Teacher in one course, and a
Student in another, and so have different permissions in each
place. These places are called contexts. Contexts in Moodle form a
hierarchy rather like a folder hierarchy in a file-system. At the top
level is the system context (and, since our script is not very well
integrated into Moodle, we use that context).

Within the system context are a number of contexts for the different
categories that have been created to organise courses. These can be
nested, with one category containing other categories. Category
contexts can also contain Course contexts. Finally, each activity in a
course will have its own Module context.

\aosafigure{../images/moodle/contexts.png}{Contexts}{fig.moodle.contexts}

\end{aosasect2}

\begin{aosasect2}{Checking the User Has Permission to Use This Script}

\begin{verbatim}
require_capability('local/greet:begreeted', $context);        // 4
\end{verbatim}

Now we have got the context---the relevant area of Moodle to check the
permission in---we now need to decide which permission to check. Each
particular area of functionality that a user may or may not have is
called a Capability. Checking a capability provides more fine-grained
access control than the basic checks performed by
\code{require\_login}. For our simple example plugin, we have one
capability: \code{local/greet:begreeted}.

The check is done using the \code{require\_capability} function, which
takes the capability name and the context. Like other
\code{require\_{\ldots}} functions, it will not return if the user
does not have the capability. It will display an error instead. In
other places we would use the non-fatal \code{has\_capability}
function, which returns a Boolean. We might use that, for example, to
determine whether to display a link to this script from another page.

How does the administrator configure which user has which permission?
Here is the calculation that \code{has\_capability} performs (at least
conceptually):

\begin{aosaenumerate}

\item Start from the current Context.

\item Get a list of the Roles that the user has in this Context.

\item Then work out what the Permission is for each Role in this
  Context.

\item Aggregate those permissions to get a final answer.

\end{aosaenumerate}

\end{aosasect2}

\begin{aosasect2}{Defining Capabilities}

As the example shows, a plugin can define new capabilities, relating
to the particular functionality it provides. Inside each Moodle
plugin, there is a sub-folder of the code called \code{db}. This
contains all the information required to install or upgrade the
plugin. One of those bits of information is a file called
\code{access.php} that defines the capabilities. Here is the
\code{access.php} file for our plugin, which lives in
\code{local/greet/db/access.php}:

\begin{verbatim}
<?php
$capabilities = array('local/greet:begreeted' => array(
    'captype' => 'read',
    'contextlevel' => CONTEXT_SYSTEM,
    'archetypes' => array('guest' => CAP_ALLOW, 'user' => CAP_ALLOW)
));
\end{verbatim}

This gives some metadata about each capability which are used when
constructing the permissions management user interface. It also give
default permissions for common types of role.

\end{aosasect2}

\begin{aosasect2}{Roles}

The next part of the Moodle permissions system is roles. A role is
really just a named set of permissions. When you are logged into
Moodle, you will have the ``Authenticated user'' role in the system
context, and since the system context is the root of the hierarchy,
that role will apply everywhere.

Within a particular course, you may be a student, and that role
assignment will apply in the course context, and all the module
contexts within it. In another course, however, you may have a
different role. For example, Mr Gradgrind may be teacher in the
``Facts, Facts, Facts'' course, but a Student in the professional
development course ``Facts Aren't Everything''. Finally, a user might
be given the Moderator role in one particular forum (Module context).

\end{aosasect2}

\begin{aosasect2}{Permissions}

A role defines a permission for each capability. For example, we will
probably say that the teacher role ALLOWS \code{moodle/course:manage},
but the student role does not. However, both Student and Teacher will
allow \code{mod/forum:startdiscussion}.

The roles are normally defined globally, but they can be re-defined in
each context. For example, if we want to make one particular wiki
activity read-only to students of the course, then we would override
the permissions for the Student role in that wiki (Module) context, to
set the permission for \code{mod/wiki:edit} to PREVENT.

There are four Permissions:

\begin{aosaitemize}

\item NOT SET / INHERIT (default)

\item ALLOW

\item PREVENT

\item PROHIBIT (example coming up soon)

\end{aosaitemize}

In a given context, a role will have one of these four permissions for
each capability. One different between PROHIBIT and PREVENT is that a
PROHIBIT cannot be overridden in sub-contexts.

\end{aosasect2}

\begin{aosasect2}{Permission Aggregation}

Finally we aggregate the permissions for all the roles the user has in
this context.

\begin{aosaitemize}

\item If any role gives the permission PROHIBIT for this capability,
  return false.

\item Otherwise, if any role gives ALLOW for this capability, return
  true.

\item Otherwise return false.

\end{aosaitemize}

A use case for PROHIBIT is this. Suppose a user has been making
abusive posts in a number of forums, and we want to stop them
immediately. We can create a Naughty user role, which sets
\code{mod/forum:post} and other such capabilities to PROHIBIT. We can
then assign this role to the abusive user in the system context. That
way, we can be sure that the user will not be able to post any more in
any forum. (One would then talk to the student, and having reached a
satisfactory outcome, removing that role assignment so that they my
use the system again.)

So, Moodle's permissions system gives administrators a huge amount of
flexibility. They can define whichever roles they like with different
permissions for each capability; they can alter the role definitions
in sub-contexts; and then they can assign different roles to users in
different contexts.

\end{aosasect2}

\end{aosasect1}

\begin{aosasect1}{Back to Our Example Script}

Next in our script we cover some miscellaneous points:

\begin{aosasect2}{Get Data From the Request}

\begin{verbatim}
$name = optional_param('name', '', PARAM_TEXT);               // 5
\end{verbatim}

Something that every web application has to do is to get data from the
request (GET or POST variables) without being susceptible to SQL
injection or cross-site scripting attacks. Moodle provides two ways to
do this.

The simple method is the one shown here. To get a single variable, we
give the parameter name (here \code{name}) a default value, and the
expected type. The expected type is used to clean the input of all
unexpected characters. There are numerous types like
\code{PARAM\_INT}, \code{PARAM\_ALPHANUM}, \code{PARAM\_EMAIL}, and so
on.

There is also a similar \code{required\_param} function, which like
other \code{require\_{\ldots}} functions stops execution and displays
an error message to the user if the expected parameter is not found.

The other way Moodle has to get data from the request is a
fully-fledged forms library. This is a wrapper around the HTML
QuickForm library from PEAR\footnote{For non-PHP programmers, PEAR is
  PHP's equivalent of CPAN.}. This seemed like a good choice when we
selected it, but is now no longer maintained, so at some time in the
future we will have to tackle moving to a new forms library, which
many of us look forwards to, because QuickForm has several irritating
design issues. For now, however, it is adequate. Forms can be defined
as a collection of fields of various types (e.g. text box, select
drop-down, date-selector) with client- and server- side validation
(including use of the same \code{PARAM\_{\ldots}} types).

\end{aosasect2}

\begin{aosasect2}{Global Variables}

\begin{verbatim}
if (!$name) {
    $name = fullname($USER);                                  // 6
}
\end{verbatim}

Here we see the first of the global variables Moodle
provides. \code{\$USER} makes accessible the information about the
user accessing this script. Other globals include:

\begin{aosaitemize}

\item \code{\$CFG}: holds the commonly used configuration settings.

\item \code{\$DB}: the database connection.

\item \code{\$SESSION}: a wrapper around the PHP session.

\item \code{\$COURSE}: the course we are within.

\end{aosaitemize}

\noindent
and several others, some of which we will encounter later.

You may have read the words ``global variable'' with horror. Note,
however, that PHP processes a single request at a time. Therefore
these variables are not as global as all that. In fact, PHP global
variables can be seen as an implementation of the Thread-scoped
registry pattern (in the sense of Martin Fowler, Patterns of
Enterprise Application Architecture) and this is the way in which
Moodle uses them. It is very convenient in that it makes commonly used
objects available thought the request without requiring them to be
passed to every function and method. It is only infrequently abused.

\end{aosasect2}

\begin{aosasect2}{Nothing is Simple}

This line also serves to make a point about the problem
domain. Nothing is ever simple. Here, we just want to display a user's
name, but we can't simply concatenate \code{\$USER->firstname},
\code{'~'}, and \code{\$USER->lastname}. The school may have policies
about showing either of those parts, and different cultures have
different conventions for which order to show names. Therefore, there
are several configurations settings, and we have a function to
assemble the full name according to the rules.

Similarly with dates. Different users may be in different
time-zones. Moodle stores all dates as Unix time-stamps, which are
integers, and so work simply cross-DB. There is then a \code{userdate}
function to display the time-stamp to the user using the appropriate
timezone and locale settings.

\end{aosasect2}

\begin{aosasect2}{Logging}

\begin{verbatim}
add_to_log(SITEID, 'local_greet', 'begreeted',
        'local/greet/index.php?name=' . urlencode($name));    // 7
\end{verbatim}

All significant actions in Moodle are logged. Logs are written to a
table the database. This is a trade-off. It makes sophisticated
analysis quite easy, and indeed various reports base on the logs are
included with Moodle. On a large and busy site, however, it is a
performance problem. The log table gets huge, which makes backing up
the database more difficult, and makes queries on the log table
slow. There can also be write contention on the log table. These
problems can be mitigated in various ways, for example by batching
writes, or archiving or deleting old records to remove them from the
main database.

\end{aosasect2}

\end{aosasect1}

\begin{aosasect1}{Generating Output}

Output is mainly handled via two global objects.

\begin{aosasect2}{The \code{\$PAGE} Global}

\begin{verbatim}
$PAGE->set_context($context);                                 // 8
\end{verbatim}

\code{\$PAGE} stores the information about the page we are going to
output that will be needed when the time comes to generate the
HTML. In this script, we need to explicitly specify the current
context. (In other situations, this might have been set automatically
by \code{require\_login}.) We also need to set the URL for this
page. This may seem redundant. The rationale for requiring it is that
you might get to a particular page using any number of different URLs,
but the URL passed to \code{set\_url} should be the canonical URL for
this page---a good permalink, if you like. We also specify the page
title, which will end up in the head element of the HTML.

\end{aosasect2}

\begin{aosasect2}{Moodle URL}

\begin{verbatim}
$PAGE->set_url(new moodle_url('/local/greet/index.php'),
        array('name' => $name));                              // 9
\end{verbatim}

I just wanted to flag this nice little helper class we have. As an
aside, recall that the \code{add\_to\_log} function call above did not
use this helper class. Indeed, the log API cannot code with
\code{moodle\_url} objects. This sort of inconsistency is a typical
sign of a code-base as old as Moodle's.

\end{aosasect2}

\begin{aosasect2}{Internationalisation}

\begin{verbatim}
$PAGE->set_title(get_string('welcome', 'local_greet'));       // 10
\end{verbatim}

Moodle uses its own system to allow the interface to be translated
into any language. There may now be good PHP internationalisation
libraries, but in 2002 when it was first implemented there was not one
available that was adequate. The system is based around the
\code{get\_string} function. Strings are identified by a key and the
plugin Frankenstyle name. As can be seen on line 12, it is possible to
interpolate values into the string. (Multiple values are handled using
PHP arrays or objects.)

The strings are looked up in language files that are just plain PHP
arrays. Here is the language file
\code{local/greet/lang/en/local\_greet.php} for our plugin:

\begin{verbatim}
<?php
$string['greet:begreeted'] = 'Be greeted by the hello world example';
$string['welcome'] = 'Welcome';
$string['greet'] = 'Hello, {$a}!';
$string['pluginname'] = 'Hello world example';
\end{verbatim}

Note that, as well as the two string used in our script, we also have
to define a string to give a name to our capability, and we have to
define a string \code{pluginname} which is the name of our plugin as
it appears in the user interface.

The different languages are identified by the two-letter country code
(\code{en} here). Languages packs may derive from other language
packs. For example the \code{fr\_ca} (French Canadian) language pack
declares \code{fr} (French) as the parent language, and thus only has
to define those strings that differ from the French. Since Moodle
originated in Australia, \code{en} means British English, and
\code{en\_us} (American English) is derived from it.

Again, the simple \code{get\_string} API for plugin developers hides a
lot of complexity, including working out the current language (which
may depend on the current user's preferences, or the settings for the
particular course they are currently in), and then searching through
all the language packs and parent language packs to find the string.

Producing the language pack files, and co-ordinating the translation
effort is managed at \url{http://lang.moodle.org/}, which is moodle
with a custom plugin\footnote{\code{local\_amos},
  \url{http://docs.moodle.org/22/en/AMOS}.}. It uses both Git and the
database as a back-end to store the language files with full version
history.

\end{aosasect2}

\begin{aosasect2}{Starting Output}

\begin{verbatim}
echo $OUTPUT->header();                                       // 11
\end{verbatim}

This is another innocuous-looking line that does much more than it
seems. The point is that before any output can be done, the applicable
theme (skin) to use must be worked out. The theme may depend on a
combination which context we are in and the user's
preferences. \code{\$PAGE-{\textgreater}context} was, however, only
set on line 8. In order to solve the problem, some PHP magic is used
so that the first time a method on \code{\$OUTPUT} is called, it
re-initialises itself based on the information in \code{\$PAGE}, and
only then generates the output.

Another thing that needs to be done is that every page in Moodle may
contain \emph{blocks}. These are extra configurable bits of content
that are included in the page. (They are a type of plugin.) Again, the
exact collection of blocks to display depends in a flexible way (that
the administrator can control) on the page context and some other
aspects of the page identity. Therefore, another part of preparing for
output is a call to
\code{\$PAGE-{\textgreater}blocks-{\textgreater}load\_blocks()}.

Once all the necessary information has been worked out, the theme
plugin (that controls the overall look of the page) is called to
generate the overall page layout, including whatever standard header
and footer is desired. This call is also responsible for adding the
output from the blocks at the appropriate place in the HTML. In the
middle of the layout there will be a div where the specific content
for this page goes. The HTML of this layout is generated, and then
split in half after the start of the main content div. The first half
is returned, and the rest is stored to be returned by
\code{\$OUTPUT-{\textgreater}footer()}.

\end{aosasect2}

\begin{aosasect2}{Outputting the Body of the Page}

\begin{verbatim}
echo $OUTPUT->box(get_string('greet', 'local_greet',
        format_string($name)));                               // 12
\end{verbatim}

This line outputs the body of the page. Here we simply display the
greeting in a box. The greeting is again a localised string, this time
with a value substituted into a placeholder. The core renderer
\code{\$OUTPUT} provides many convenience methods like \code{box} to
describe the required output in quite high-level terms. Different
themes can control what HTML is actually output to make the box.

The content that originally came from the user (\code{\$name}) is
output though the \code{format\_string} function. This is the other
part of providing XSS protection. It also enables the user of text
filters (another plugin type). An example filter would be the LaTeX
filter, which replaces input like \code{\$\$x + 1\$\$} with an image
of the equation. I will mention, but not explain, that there are
actually three different functions (\code{s}, \code{format\_string},
and \code{format\_text}) depending on the particular type of content
being output.

\end{aosasect2}

\begin{aosasect2}{Finishing Output}

\begin{verbatim}
echo $OUTPUT->footer();                                       // 13
\end{verbatim}

Finally, we have to output the footer of the page. This example does
not show it, but Moodle tracks all the JavaScript that is required by
the page, and outputs all the necessary script tags in the
footer. This is standard good practice. It allows users to see the
page without waiting for all the JavaScript to load. A developer would
include JavaScript using API calls like
\code{\$PAGE-{\textgreater}requires-{\textgreater}js('/local/greet/cooleffect.js')}.

\end{aosasect2}

\begin{aosasect2}{Should We Be Mixing Logic and Output?}

Obviously, putting the output code directly in \code{index.php}, even
if at a high level of abstraction, limits the flexibility that themes
have to control the output. This is another sign of the age of the
Moodle code-base. The \code{\$OUTPUT} global was introduced in 2010 as
a stepping stone on the way from the old code, where the output and
controller code was in the same file, to a design where all the view
code was properly separated. This also explains the rather ugly way
that the entire page layout is generated, then split in half, so that
any output from the script itself can be placed between the header and
the footer. Once the view code has been separated out of the script,
into what Moodle calls a renderer, the theme can then choose to
completely (or partially) override the view code for a given script.

With a small refactoring, we can go the final step and move all the
output code out of our index.php and into a renderer. We could change
the end of index.php (lines 11 to 13) to:

\begin{verbatim}
$output = $PAGE->get_renderer('local_greet');
echo $output->greeting_page($name);
\end{verbatim}

and then make a new file \code{local/greet/renderer.php}:

\begin{verbatim}
<?php
class local_greet_renderer extends plugin_renderer_base {
    public function greeting_page($name) {
        $output = '';
        $output .= $this->header();
        $output .= $this->box(get_string('greet', 'local_greet', $name));
        $output .= $this->footer();
        return $output;
    }
}
\end{verbatim}

If the theme wished to completely change this output, it would define
a subclass of this renderer that overrides the \code{greeting\_page}
method. \code{\$PAGE-{\textgreater}get\_renderer()} determines the
appropriate renderer class to instantiate depending on the current
theme. Thus, the output (view) code is fully separated from the
controller code in \code{index.php}, and we have been able to refactor
from typical legacy Moodle code to a clean MVC architecture.

\end{aosasect2}

\end{aosasect1}

\begin{aosasect1}{Database Abstraction}

The hello world script was sufficiently simple that it did not need to
access the database, although several of the Moodle library calls used
did do database queries. Now I will quickly talk about the Moodle
database layer.

Moodle used to use the ADOdb library of as the basis of its database
abstraction layer, but there were issues for us, and the extra layer
of library code had a noticeable impact on performance, so in Moodle
2.0 we switched to our own abstraction layer, which is a thin wrapper
around the various PHP database libraries.

\begin{aosasect2}{The \code{moodle\_database} Class}

The heart of the library is the \code{moodle\_database} class. This
defines the interface provided by the \code{\$DB} global variable,
which gives access to the database connection. A typical usage might
be:

\begin{verbatim}
$course = $DB->get_record('course', array('id' => $courseid));
\end{verbatim}

That translates into the SQL:

\begin{verbatim}
SELECT * FROM mdl_course WHERE id = $courseid;
\end{verbatim}

\noindent
and returns the data as a plain PHP object with public fields, so you
could access \code{\$course-{\textgreater}id},
\code{\$course-{\textgreater}fullname}, etc.

Simple methods like this deal with basic queries, and simple updates
and inserts. Sometimes it is necessary to do more complex SQL, for
example to run reports. In that case, there are methods to execute
arbitrary SQL:

\begin{verbatim}
$courseswithactivitycounts = $DB->get_records_sql(
        'SELECT c.id, ' . $DB->sql_concat('shortname', "' '", 'fullname') . ' AS coursename,
                COUNT(1) AS activitycount
         FROM {course} c
         JOIN {course_modules} cm ON cm.course = c.id
         WHERE c.category = :categoryid
         GROUP BY c.id, c.shortname, c.fullname ORDER BY c.shortname, c.fullname',
        array('categoryid' => $category));
\end{verbatim}

\noindent
Some things to note there:

\begin{aosaenumerate}

\item The table names are wrapped in \code{\{\}} so that the library
  can find them and prepend the table name prefix.

\item The library uses placeholders to insert values into the SQL. In
  some cases this can use the facilities of the underlying database
  driver. In other cases the we have to insert the values
  ourselves. Our library can support either named placeholders (as
  above) or anonymous ones, using \code{?} as the placeholder.

\item In order that the SQL works on all our supported databases you
  have to use the right subset of standard SQL. For example, you can
  see that I have used the AS keyword for column aliases, but not for
  table aliases. Both of those things are necessary.

\item Even so, there are some things where no subset of standard SQL
  will work on all our supported databases, for example, every
  database has a different way to concatenate strings. In these cases,
  we have compatibility function to generate the correct SQL.

\end{aosaenumerate}

\end{aosasect2}

\begin{aosasect2}{Defining the Database Structure}

Another area where database management systems differ a lot is in the
SQL syntax required to define tables. To get round this problem, each
Moodle plugin (and Moodle core) defines the required database tables
in an XML format, and the Moodle install system parses that, and uses
the information to create the right tables and indexes in the
database. There is a developer tool called XMLDB built into Moodle to
help create and edit these install files.

If the database structure needs to change between two releases of
Moodle (or of a plugin), then the developer is responsible for writing
code (using an additional database object that provides DDL methods)
to update the database structure, while preserving all the users'
data. Thus, Moodle will always self-update from one release to the
next, simplifying the maintenance task for administrators.

One contentious point, stemming from the fact that Moodle started out
using MySQL 3, is that the Moodle database does not use foreign
keys. This allows some buggy behaviour to remain undetected even
though modern databases would be capable of detecting the problem. The
difficulty is that people have been running Moodle sites without
foreign keys for years, so there is almost certainly inconsistent data
present, and so adding the keys now would be impossible, without a
very difficult clean-up job. Even so, since the XMLDB system was added
to Moodle 1.7 (in 2006!), the install.xml files have contained the
definitions of the foreign keys that should exist, and we are still
hoping, one day, to do all the work necessary to allow us to create
them during the install process.

\end{aosasect2}

\end{aosasect1}

\begin{aosasect1}{What We Have Not Covered}

I hope I have given you a good overview of how Moodle works. Due to
lack of space I have had to omit several interesting topics, including
how authentication, enrolment and grade plugins allow Moodle to
interoperate with student information systems, and the interesting
content-addressed way that Moodle stores uploaded files. Details of
these, and other aspects of Moodle's design can be found in the
developer documentation\footnote{\url{http://docs.moodle.org/dev/.}}.

\end{aosasect1}

\begin{aosasect1}{Lessons Learned}

One interesting aspect of working on Moodle is that it came out of a
research project. Moodle enables (but does not enforce) a social
constructivist
pedagogy\footnote{\url{http://docs.moodle.org/en/Pedagogy}}. That is,
we learn best by actually creating something, and we learn from each
other as a community. Martin Dougiamas's PhD question did not ask
whether this was an effective model for education, but rather he was
asking whether that is an effective model for running an open source
project. That is, can we view the Moodle project as an attempt to
learn how to build and use a VLE, and an attempt to learn that by
actually building an using Moodle as a community where teacher,
developers, administrators and students all teach and learn from each
other. I find this a very good model for how to think about the
process of doing software development. The main place this learning
happens is in discussions in the Moodle project forums, and in the bug
database.

Perhaps the most important consequence of this learning approach is
that you should not be afraid to start by implementing the simplest
thing at first. To give an example, at first Moodle had a few
hard-coded roles like Teacher, Student, Administrator. That was enough
for many years. Eventually, it proved too inflexible, and so the Roles
system was introduced. When it came time to design the roles system
for Moodle 1.7, there was a lot of experience in the community about
how people were using Moodle, and a lot of little feature requests
that showed what people needed from a more flexible access control
system. This all helped design the Roles system to be as simple as
possible, but as complex as necessary. In fact, the first version of
the roles system ended up slightly too complex, and it was
subsequently simplified a little in Moodle 2.0.

Now, if you take the view that programming is a problem solving
exercise, then you might think that Moodle got the design wrong the
first time, and later had to waste time correcting it. I suggest that
is an unhelpful viewpoint. At the time Moodle started, no-one knew
enough to design the roles system we have now. If you take the
learning viewpoint, then the various stages Moodle went through were
necessary and inevitable.

For this to work, it must be possible to change almost any aspect of a
system's architecture. I think Moodle shows that this is possible. It
does take a certain amount of effort, but it can be done when
necessary, and from the user's point of view it is just a gradual
evolution.

\end{aosasect1}

\end{aosachapter}
