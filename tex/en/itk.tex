\begin{aosachapter}{ITK}{s:itk}{Luis Ib\'{a}\~{n}ez and Brad King}

% Your chapter goes here---please look at /volume1/tex/en/wesnoth.tex for
% formatting ideas.

\begin{aosasect1}{What Is ITK?}
The Insight Toolkit ITK\footnote{\url{http://www.itk.org}} is a library for
image analysis that was developed by the initiative, and mainly with the
funding, of the US National Library of
Medicine\footnote{\url{http://www.nlm.nih.gov}}. ITK can be thought of as a
usable encyclopedia of image analysis algorithms, in particular for image
filtering, image segmentation and image registration. The library was developed
by a consortium involving universities, commercial companies, and many
individual contributors from around the world.  Development of ITK started in
1999, and recently after its 10th anniversary the library underwent a
refactoring process intended to remove crusty code and to reshape it for the
next decade.
\end{aosasect1}

\begin{aosasect1}{Architectural Features}

Software toolkits have a very synergistic relationship with their communities.
They shape one another in a continuous iterative cycle.  The software gets to
be modified until it satisfies the needs of the community, while the community
behaviors themselves are adapted based on what the software empowers or
restricts them to do. In order to better understand the nature of ITK's
architecture, it is therefore very useful to get a sense of what kind of
problems the ITK community is usually addressing and how they tend to go about
them.

\begin{aosasect2}{The Nature of the Beast}

\begin{center}
\begin{quotation}
\emph{
``If you don't understand the nature of the beast,\\
it would be of little use to know the mechanics of their anatomy''.\\
}
\hfill Dee Hock - \emph{One From Many}
\end{quotation}
\end{center}

In a typical image analysis problem, a researcher or an engineer will take an
input image, will improve some characteristics of the image by let's say
reducing noise or increasing contrast, and then will proceed to identify some
features in the image, such as corners and strong edges. This type of
processing is naturally well suited for a data pipeline architecture, as
shown in Figure~\ref{fig.itk.pipeline}.

To illustrate this point, Figure~\ref{fig.itk.brim} below, shows an image of a
brain from a magnetic resonance image (MRI), and the result of processing it
with a median filter to reduce its level of noise, as well as the outcome of an
edge detection filter used to identify the borders of anatomical structures.

\aosafigure{../images/itk/ExampleImageProcessingPipeline.pdf}{Image Processing Pipeline}{fig.itk.pipeline}


%
% NOTE to the EDITOR: We meant for these images to be rather small, and to be
% arranged in a single row, from left to right, but didn't quite found how
% to resize them with the parameters of the aosafigure command.
%
% \aosafigure{../images/itk/BrainProtonDensitySlice.png}{MRI Brain Image}{fig.itk.brim}
% \aosafigure{../images/itk/BrainProtonDensitySliceMedian.png}{Median Filter}{fig.itk.brimmedian}
% \aosafigure{../images/itk/BrainProtonDensitySliceCanny.png}{Edge Detection Filter}{fig.itk.brimcanny}
%
\begin{figure}[h!]
\centering
\includegraphics[width=0.3\textwidth]{../images/itk/BrainProtonDensitySlice.png}
\includegraphics[width=0.3\textwidth]{../images/itk/BrainProtonDensitySliceMedian.png}
\includegraphics[width=0.3\textwidth]{../images/itk/BrainProtonDensitySliceCanny.png}
\caption{From left to right: MRI Brain Image, Median Filter, Edge Detection Filter}
\label{fig.itk.brim}
\end{figure}

For each one of these tasks, the image analysis community has developed a
variety of algorithms, and continue developing new ones. \emph{``Why do they
continue doing this?''} you may ask, and the answer is that image processing is
a combination of science, engineering, art, and ``cooking'' skills. Claiming
that there is an algorithmic combination that is the ``right'' answer to an
image processing task, is as misleading as claiming that there is such a thing as
the ``right'' type of chocolate dessert for a dinner. Instead of pursuing
perfection, the community strives to produce a rich set of tools that
ensures that there will be no shortage of options to try when it comes the
time to face a given image processing challenge. This state of
affairs, of course, comes at a price. The cost is that the image
analyst has the difficult task of choosing among dozens of different
tools that can be used in different combinations to achieve similar
results.

The image analysis community is also closely integrated with the research
community. It is common to find specific research groups that become attached
to the algorithmic families they have developed. This custom of ``branding'',
and up to some level ``marketing'', leads to a situation where the best that the
software toolkit can do for the community is to offer a very complete set of
algorithmic implementations that they can try, and then mix and match to create
a recipe that satisfies their needs.

These are some of the reasons why ITK was designed and implemented as a large
collection of somewhat independent tools, many of which can be used to solve
similar problems. In this context, a certain level of ``redundancy'',
for example offering three different implementation of the Gaussian
filter, is not
seen as a problem but as a valuable feature. The toolkit was also conceived as
a resource that will grow and renew itself continuously, as new algorithms and
better implementations become available superseding exiting ones, as well as
new tools are developed in response to the emerging needs of new medical
imaging technologies.

Armed with this quick overview of the nature of activities that make the daily
routine of the image analysts who compose the ITK community, we can now dive
into the main features of the architecture:

\begin{aosaitemize}
\item Modularity
\item Data Pipeline
\item Factories
\item IO Factories
\item Streaming
\item Maintainability
\item Reusability
\end{aosaitemize}

\end{aosasect2}

\begin{aosasect2}{Modularity}
Modularity is one of the main characteristics of ITK. This is a requirement
that emerges from the way people in the image analysis community work when
solving their problems. Most image analysis problems put one or more
input images through a combination of processing filters that enhance or
extract particular pieces of information from the images. Therefore,
there is no single large processing object, but rather a myriad of small ones.
This structural nature of the image processing problem maps to implement the
software as a large collection of image processing filters that can be combined
in many different ways.

It is also the case that certain types of processing filters are clustered into
families, inside which some of their implementation features can be factorized.
This leads to natural grouping of the image filters into modules and groups of
modules.

Modularity, therefore occurs at three natural levels in ITK:

\begin{aosaitemize}
\item Filter Level
\item Filter Family Level
\item Filter Family Group Level
\end{aosaitemize}

At the image filter level, ITK has a collection of about 700 filters. Given
that ITK is implemented in C++, this is a natural level at which every one of
those filters is implemented by a \textbf{C++ Class} following and Object
Oriented design.  At the filter family level, ITK groups filters together
according to the nature of the processing that they perform. For example, all
filters that are related to Fourier Transforms will be put together into a
\textbf{Module}.  At the C++ level, Modules maps to directories in the
source tree, and to libraries once the software is compiled to its
binary form. ITK has about 120 of these Modules. Each module contains:

\begin{aosaenumerate}

\item The source code of the image filters that belong to that family.

\item A set of configuration files that describe how to build the module and
list dependencies between this module and other modules.

\item The set of unit tests corresponding to each one of the filters.

\end{aosaenumerate}

The group level is mostly a conceptual division that has been drawn on
top of the software to help locate filters in the source tree. Groups
are associated to high level concepts such as Filtering, Segmentation,
Registration and IO.

This hierarchical structure is illustrated in
Figure~\ref{fig.itk.modulehierarchy}.

\aosafigure{../images/itk/IllustrationOfModularStructure.pdf}{Hierarchical Structure of Groups, Modules and Classes}{fig.itk.modulehierarchy}


ITK currently has 124 modules, that are in turn aggregated into 13 major groups.
The modules have a variety of different sizes. This size distribution, in
bytes, is presented in Figure~\ref{fig.itk.modulesize}.

\aosafigure{../images/itk/moduleSizePlot.pdf}{Distribution of Module Size in Bytes}{fig.itk.modulesize}
\aosafigure{../images/itk/moduleSizePlotNoThirdParty.pdf}{Distribution of Module Size in Bytes without Third Party Libraries}{fig.itk.modulesizenothirdparty}

The modularization in ITK applies as well to a set of third party libraries
that are not directly part of the toolkit, but that the toolkit depend upon, and
that are distributed along with the rest of the code for the convenience of
users. Particular examples of these third party libraries are the image file
format libraries: HDF5, PNG, TIFF, JPEG and openjpeg among others. When these
libraries are excluded from the analysis of module size, the distribution
become the one shown in Figure~\ref{fig.itk.modulesizenothirdparty}

The reason why Figure~\ref{fig.itk.modulesize} and
Figure~\ref{fig.itk.modulesizenothirdparty} present separate analysis, with and
without the third party libraries, is that these libraries make about 56
percent of the size of ITK. This reflects the usual nature of open source
applications that build upon existing platforms. The size distribution of the
third party libraries does not necessarily reflect the architectural
organization of ITK, since we have adopted these useful third party libraries
just as they have been developed upstream. However, they third party code is
distributed along with the toolkit, and partitioning it was one of the key
driving directives for the modularization process.

The modular architecture of ITK enables and facilitates:

\begin{aosaitemize}
\item Reduction and clarification of cross-dependencies
\item Adoption of code contributed by the community
\item Evaluation of quality metrics per module (for example: code coverage)
\item Building selected subsets of the toolkit
\item Packaging selected subsets of the toolkit for redistribution
\item Continued growth by progressive addition of new modules
\end{aosaitemize}
\end{aosasect2}

The modularization process made possible to very explicitly identify and
declare the dependencies between different portions of the toolkit as they were
put into modules. In many cases, this exercise revealed artificial or incorrect
depencencies that had been introduced in the toolkit over time, and that passed
unnoticed when most of the code was put together in a few large groups.

The usefulness of evaluating quality metrics per module is twofold. In one
hand, it makes easier to held developers accoutable for the modules that they
are maintaining. On the other hand, it also makes possible to engage in
clean-up innitiatives in which a few developers focus for a short period of
time in raising the quality of a specific module. When concentrating in a small
portion of the toolkit, it is easier to see the effect of the effort, and to
keep developers engaged and motivated to keep up with the effort.

Again, we can recognize in these items the fact that the structure of the
toolkit reflects the organization of the community and in some cases the
processes that have been adopted for the continuous growth and quality control
of the software.

\begin{aosasect2}{Data Pipeline}
The staged nature of most image analysis tasks led naturally to the selection
of a Data Pipeline architecture as the backbone infrastructure for data
processing. The Data Pipeline enables:

\begin{aosaitemize}

\item \textbf{Filter Concatenation:} A set of image filters can be concatenated
one after another, composing a processing chain in which a sequence of
operations are applied to the input images.

\item \textbf{Parameter Exploration:} Once a processing chain is put together,
it is easy to change the parameters of any filter in the chain, and to explore
the effects that such change will have on the final output image.

\item \textbf{Memory Streaming:} Large images can be managed by processing only
sub-blocks of the image extent at a time. In this way, it becomes possible to
process large images that otherwise would have not fitted in main memory.

\end{aosaitemize}

\aosafigure{../images/itk/DataPipelineIllustrationInRegionGrowingApplication.png}{GUI Illustration of a Data Pipeline for performing region growing segmentation}{fig.itk.regiongrowinggui}

Figures~\ref{fig.itk.pipeline} and~\ref{fig.itk.brim} have already presented a
simplified representation of a data pipeline from the image processing point of
view. A more detailed case can be seen in
Figure~\ref{fig.itk.regiongrowinggui}, for a simple application that exposes
the Data Pipeline connections in its GUI, along with the set of numeric
parameters that can be adjusted for each one of the intermediate filters. Every
time that one of the numeric parameters is modified, the data pipeline marks
its output as ``dirty'' and knows that this particular filter, and all the
downstream ones that use its output, should be executed again. This feature of
the pipeline facilitates the exploration of the parameter space while
maintaining to a minimum the amount of processing power that have to be
invested at every instance of an experiment.

The process of updating the pipeline may be driven in such a way that only
sub-pieces of the images are processed at a time. This is a mechanism necessary
to support the functionality of Streaming. In practice, the process is
controled by the internal passing of a \code{RequestedRegion} specification
from one filter downstream to its provider filter upstream. This communication
is done through an internal API and it is not exposed to the application
developers. For a more concrete example, if a Gaussian image filter is
expecting to use as input a 100x100 pixels image that is produced by a Median
image filter, the Blur filter can ask the Median filter to produce only a
quarter of the image, that is an image region of size 100x25 pixels.  Such
request can be further propagated upstream, with the caveat that every
intermediate filter may have to add an extra border to the image region size in
order to produce that requested output region size.
More on data streaming in section~\ref{Streaming}.

Both a changes in parameters in a given filter, or a change in the
specific requested region to be processed by that filter will have the
effect of marking the pipeline as ``dirty'' and indicate the need for
a refresh from that filter to the downstream filters in that pipeline.

Two main types of objects were designed to hold the basic structure of the
pipeline.  They are the \code{DataObject} and the \code{ProcessObject}. The
\code{DataObject} is the abstraction of classes that carry data. For example,
images and geometrical meshes. The \code{ProcessObject} provides an abstraction
for the image filters and mesh filters that process such data.
\code{ProcessObjects} take \code{DataObjects} as input and perform some type of
algorithmic transformation on them, such as the ones illustrated in
Figure~\ref{fig.itk.brim}.

\aosafigure{../images/itk/ProcessObjectDataObject.pdf}{Relationship between \code{ProcessObjects} and \code{DataObjects}}{fig.itk.processobjectdataobject}

\code{DataObjects} are generated by \code{ProcessObjects}. This chain typically starts by
reading a \code{DataObject} from disk, for example by using a \code{FileImageReader} which is
a type of \code{ProcessObject}. The \code{ProcessObject} that created a given \code{DataObject} is
the only one that should modify such \code{DataObject}. This output \code{DataObject} is
typically passed as input to another \code{ProcessObject} downstream in the pipeline.

This sequence is illustrated in Figure~\ref{fig.itk.processobjectdataobject}.
The same \code{DataObject} may be passed as input to multiple
\code{ProcessObjects}, as it is shown in the Figure for the \code{DataObject}
constructed by the File Reader at the beginning of the pipeline.
In this particular case, the File Reader is an instance of the
\code{ImageFileReader} class, and the \code{DataObject} that it produces as
output is an instance of the \code{Image} class.
It is also common for some filters to require two \code{DataObjects} as input,
as it is the case of the ``Subtract Filter'' illustrated in the right side of
the same figure.

The initial design and implementation of the Data Pipeline in ITK was derived
from the one in the Visualization Toolkit (VTK)\footnote{See \emph{Architecture
of Open Source Applications}, Volume 1}, that was a mature project at the time
that ITK development was in its initial stages.

Figure~\ref{fig.itk.processobjectdataobjecthierarchy} shows the Object Oriented
hierarchy of the pipeline Objects in ITK. In particular the relationship
between the basic \code{Object}, \code{ProcessObject}, \code{DataObject}, and
some of the classes in the filter family and the data family. In this
abstraction, any object that is expected to be passed as input to a filter, or
to be produced as output by a filter, must derive from the \code{DataObject}. All
filters that produce and consume data are expected to derive from the
\code{ProcessObject}. The data negotiations required to move data through the
pipeline are implemented, part in the \code{ProcessObject} and part in the
\code{DataObject}.

\aosafigure{../images/itk/ProcessObjectDataObjectHierarchy.pdf}{Hierarchy of \code{ProcessObjects} and \code{DataObjects}}{fig.itk.processobjectdataobjecthierarchy}

The \code{LightObject} and \code{Object} classes are above the
dichotomy of the \code{ProcessObject} and \code{DataObject}. The
\code{LightObject} and \code{Object} classes provide common
functionalities such as the API for communications of \code{Events},
and the support for multi-threading.


\aosafigure{../images/itk/ProcessObjectDataObjectInteractionUML.pdf}{UML
diagram of Data Pipeline interactions between \code{ProcessObjects} and
\code{DataObjects} for the case of a minimal pipeline involving an
\code{ImageFileReader}, \code{MedianImageFilter} and
\code{ImageFileWriter}.}{fig.itk.processobjectdataobjectinteractionuml}

\end{aosasect2}

\begin{aosasect2}{Factories}

One of the fundamental design requirements of ITK is to provide support for
multiple platforms. This requirement emerges from the desire to maximize the
impact of the toolkit by making it usable to a broad community regardless of
their platform of choice. ITK adopted the \emph{Factory} design pattern to
address the challenge of supporting fundamental differences among the many
hardware and software platforms, without sacrificing the good fitness of a
solution to each one of the individual platforms.

The Factory Pattern in ITK uses class names as keys to a registry of class
constructors. The registration of Factories happens at run time, and can be
done by simply placing dynamic libraries in specific directories that ITK
applications search at start up time. This last feature provides a natural
mechanism for implementing a Plugin Architecture in a clean and transparent
way. The outcome is to facilitate the development of extensible image analysis
applications, satisfying the requirements of providing an ever-growing set of
image analysis capabilities.

\end{aosasect2}

\begin{aosasect2}{IO Factories}

The image analysis community has developed a very large set of file formats to
store image data. Many of these file formats are designed and implemented with
specific uses in mind, and therefore are fine-tuned to specific types of
images. As a consequence, on a regular basis, new image file formats are
conceived and promoted across the community. Aware of this situation, the ITK
development team designed an IO Architecture suitable for ease of
extensibility, in which it is easy to add support for more and more file
formats on a regular basis.

\aosafigure{../images/itk/ImageIOFactoriesDesignPattern.pdf}{IO Factories Dependencies}{fig.itk.io.factoriesregistry}

These IO extensible architecture is built upon the Factories mechanism
described in the previous section. The main difference is that in the IO case,
the IO Factories are registered in a specialized registry that is managed by
the \code{ImageIOFactory} base class, shown on the upper left corner of
Figure~\ref{fig.itk.io.factoriesregistry}. The actual functionality of reading
and writing data from image file formats is implemented in a family of
\code{ImageIO} classes, shown on the right side of
Figure~\ref{fig.itk.io.factoriesregistry}. These service classes are intended
to be instantiated on demand when the user request to read or write an image.
The services classes are not exposed to the application code. Instead,
application are expected to interact with the facade classes:

\begin{aosaitemize}
\item \code{ImageFileReader}
\item \code{ImageFileWriter}
\end{aosaitemize}

These are the two classes with which the application will invoke code such as:

\begin{aosaitemize}
\item \code{reader->SetFileName(``image1.png'');}
\item \code{reader->Update();}
\end{aosaitemize}

or

\begin{aosaitemize}
\item \code{writer->SetFileName(``image2.jpg'');}
\item \code{writer->Update();}
\end{aosaitemize}

In both cases the call to \code{Update()} triggers the execution of the
upstream pipeline to which these \code{ProcessObjects} are connected. Both the
reader and the writer behave as one filter more in a pipeline. In the
particular case of the reader, the call to \code{Update()} triggers the reading
of the corresponding image file in to memory. In the case of the writer, the
call to \code{Update()} triggers the execution of the upstream pipeline that is
providing the input to the writer, and finally results in an image being
written out to disk into a particular file format.

The self-contained nature of every IO Factory and ImageIO service classes is
also reflected in the modularization. Typically, an ImageIO class depends on a
specialized library that is dedicated to managing a specific file format. That
is the case for PNG, JPEG, TIFF and DICOM, for example. In those cases, the
third party library is managed as a self-contained module, and the specialized
ImageIO code that interfaces ITK to that third party library is also put in a
Module by itself. In this way, specific applications may disable many
file formats that are not relevant for their domain, and can focus on offering
only those file formats that are useful for the anticipated scenarios of this
application.

Just as with standard Factories, the IO Factories can also be loaded at
run-time from dynamic libraries. This flexibility facilitates the use of
specialized and in-house developed file formats without requiring all such file
formats to be incorporated directly into the ITK toolkit. The loadable IO
factories has been one of the most successful features in the Architectural
design of ITK. It has made possible to easily manage a challenging situation
without placing a burden in the code nor obscuring its implementation. More
recently, the same IO architecture has been adapted to manage the process of
reading and writing files containing spatial transformations represented by the
\code{Transform} class family.

\end{aosasect2}

\begin{aosasect2}{Streaming}
ITK was conceived initially as a set of tools for processing the images
acquired by the Visible Human
Project\footnote{\url{http://www.nlm.nih.gov/research/visible/visible_human.html}}.
At the time, it was clear that such a large dataset would not fit in the RAM of
computers that were typically available to the medical imaging research
community. It is still the case that the dataset will not fit in the typical
desktop computers that we use today. Therefore, one of the requirements for
developing the Insight Toolkit was to enable the streaming of image data
through the Data Pipeline. More specifically, to be able to process large
images by pushing sub-blocks of the image throughout the data pipeline, and then
assembling the resulting blocks on the output side of the pipeline.

There are interesting similarities between the process of Streaming
and the process of paralellizing the execution of a given filter. Both
of them rely on the possibility of dividing the image processing work
in to image chunks that are processed separatedly. In the streaming
case, the image chunks are processed across time, one after another;
while in the parallelization case, the image chunks are assigned to
different threads, which in turn are assigned to separate processor
cores. At the end, it is the algorithmic nature of the filter, the
one that will determine whether it is possible to split the output
image into chunks that can be computed independently based on a
corresponding set of image chunks from the input image. In ITK,
streaming and parallelization are actually orthogonal, in the sense
that there is an API to take care of the streaming process, and a
separate API dedicated to support the implementation of parallel
computation base on multiple-threads and shared memory.

Streaming, unfortunately, can not be applied to all types of algorithms. Specific
cases that are not suitable for streaming are:

\begin{aosaitemize}
\item Iterative algorithms that, to compute a pixel value at every iteration,
require to use as input the pixel values of its neighbors. This is the case of
most PDE-solving-based algorithms, such as anisotropic diffusion, demons
deformable registration, and dense level sets, for example.
\item Algorithms that requires the full set of input pixel values in order to
compute the value of one of the output pixels. Fourier transform and Infinite
Impulse Response (IIR) filters, such as the Recursive Gaussian filter, are
examples of this class.
\item Region propagation or front propagation algorithms in which the
modification of pixels also happens in an iterative way but for which the
location of the regions or fronts can not be systematically partitioned in
blocks in a predictable way. Region growing segmentation, sparse level sets,
some implementations of mathematical morphology operations and some forms of
watersheds are typical examples here.
\item Image registration algorithms, given that they require to have access to
the full input image data for computing metric values at every iteration of
their optimization cycles.
\end{aosaitemize}

Fortunately, on the other hand, the data pipeline structure of ITK enables
support for streaming in a variety of transformation filters by taking
advantage of the fact that all filters create their own output, and therefore
they do not overwrite the memory of the input image. This comes at the price of
memory consumption, since the pipeline has to allocate both the input and
output images in memory simultaneously. Filters such as flipping, axes
permutation, and geometric resampling fall in this category. In these cases,
the data pipeline manages the matching of input regions to output regions by
requiring every filter to provide a method called
\code{GenerateInputRequestedRegion()} that takes as argument a rectangular
output region. This method computes the rectangular input region that will be
needed by this filter to compute that specific rectangular output region. This
continuous negotiation in the data pipeline makes possible to find for every
output block the corresponding section of input image that is required for
computation.

To be more precise here, we must say therefore that ITK supports
Streaming\ldots but only in the algorithms that are \emph{``streamable''} in
nature. That said, in the spirit of being progressive regarding the remaining
algorithms, we should qualify this statement by not claiming that ``it is
impossible to stream such algorithms'', but rather say that ``our typical
approach to streaming is not suitable for these algorithms'' at this point, and
that hopefully new techniques will be devised by the community in the future to
address these cases.
\end{aosasect2}

\end{aosasect1}

\begin{aosasect1}{Lessons Learned}

\begin{aosasect2}{Reusability}
The principle of reusability can also be read as ``avoidance of redundancy''.
In the case of ITK, this has been achieved by a three-prone approach.

\begin{aosaitemize}
\item First, the adoption of Object Oriented Programming, and in particular the
proper creation of class hierarchies where common functionalities are
factorized in base classes.
\item Second, the adoption of Generic Programming, implemented via the heavy
use of C++ templates, factorizing behaviors that are identified as patterns.
\item Third, a generous use of C++ macros has also permitted to reuse standard
snippets of code that are needed in a myriad of places across the toolkit.
\end{aosaitemize}

Many of these items may sound like platitudes and appear as obvious
choices today, but when ITK development started in 1990, some of them
where not that obvious. In particular, at the time, the support of
most C++ compilers offered for templates were not quite following a
consistent standard. Even today, decisions such as the adoption of
generic programming and the use of a widely templated implementation
continue to be controversial in the community. This is manifested in
the communities that prefer to use ITK via the wrapping layers to
Python, Tcl or Java.

There is also the general risk of doing ``too much of a good thing''.
Meaning, the risk of taking too far the use of templates, or taking to
far the use of macros. It is easy to get overboard and to end up
creating a new language on top of C++ that is essemtially based on the
use of templates and macros.

As a concrete example, the widespread use of explicitly naming types via the C++
``typedefs'', has proved to be particularly important. This practice plays two
roles. In one hand it provides a human-readable informative name describing the
nature of the type and its purpose. On the other hand, it ensures that the type
is used consistently across the toolkit. As an example, during the refactoring
of the toolkit for its 4.0 version, a massive effort was invested in collecting
the cases where C++ integer types such as \code{int}, \code{unsigned int},
\code{long} and \code{unsigned long}, were used all across the toolkit and to
replace them with types named after the proper concept that the associated
variables were representing. This was the most costly part of the process of
ensuring that the toolkit was able to take advantage of 64bits types for
managing images larger than 4Gigabytes in all platforms. A task that was of the
utmost importance for promoting the use of ITK in the fields of Microscopy and
Remote sensing, where image of tens of Gigabytes in size are of common
occurrence.
\end{aosasect2}

\begin{aosasect2}{Maintainability}
The architecture satisfies the constraints that minimize maintenance cost.
\begin{aosaitemize}
\item Modularity (at the class level)
\item Many small files
\item Code reuse
\item Repeated patterns
\end{aosaitemize}
\end{aosasect2}

These characteristics reduce maintenance cost in the following ways:

\begin{aosaitemize}
\item Modularity (at the class level): Makes possible to enforce test
driven development technics at the image filter level, or in general
the ITK class level. A stringent testing discipline applied to small
and modular pieces of code has the advantage fo reducing the pools of
code where bugs can hide, and by the natural decoupling that results
from modularization, it is a lot easier to locate defects and
eliminate them.
\item Many small files: Facilitate the assignment of portions of the
code to specific developers, and simplify the tracking of defect when
they are associated to specific commits in the revision control
system. The discipline of keeping small files also leads to the
enforcement of the golden rule of functions and classes ``do one
thing, do it right''.
\item Code reuse: When code is reused, instead of being copy pasted
and reimplemented, the code itself benefits from the higher level of
scrutiny that results from exercising this code in many different
circumstances. It leads that bring more eyes to look at the code, or
at least at the effects of the code, and benefit from the maximum:
``given enough eyeballs, all bugs are shallow''.
\item Repeated patterns: Simplify the work of the maintenance
developers, who in reality account for more than $75\%$ of the
software development cost, over the lifetime of a project. By using
coding patterns that are consistently repeated in different places in
the code, it is a lot easier for a developer to open a file, and
quickly get into the right mindset about what the code is doing, or
what it is intnded to do.
\end{aosaitemize}

As the developers got involved in regular maintenance activities, they
got exposed to the ``common failures'' of certain details, in particular:

\begin{aosaitemize}
\item Assumptions that some filters make regarding specific pixel types for
their input or output images, but that are not enforced via types nor concept
checking, and that may as well be missing to be specified in the documentation.
\item Missing to write for readability. This is one of the most common
challenges for any software whose new algorithm implementations originate in
the research community. It is common in that environment to write code that
``just works'', and to forget that the purpose of code is not just to be
executed at run time, but that it must be easily readable by the next
developer. Typical good rules of ``clean code'' writing tend to be ignored when
researchers are excited about getting their new shiny algorithm to work. For
example, small functions that do one thing and one thing only (the single
responsibility principle and the principle of least surprise), proper naming of
variables and functions.
\item Lack of attention to failure cases and error management. It is common to
focus on the ``nice cases'' of data processing and to fail to provide code for
managing all the cases that can go wrong. Helas, adopters of the toolkit
quickly run into such cases once they start developing and deploying
applications in the real world.
\item Insufficient testing. It requires a lot of discipline to follow the
practice of test driven development. Specially the notion of writing the tests
first, and only implement functionalities as you test them. It is almost always
the case that bugs in the code are hiding behind the cases that were skipped
while implementing the testing code.
\end{aosaitemize}

Thanks to the communication practices of Open Source communities, many of these
items end up being exposed through questions that are commonly asked in the
mailing lists, or that are directly reported as bugs by users  in the issue
tracker. After dealing with many of such issues, developers learn to write code
that is ``good for maintenance''. Some of these traits apply to both coding
style and the actual organization of the code. It is our view that a developer
only reaches mastery after passing some time (at least a year) doing maintenance
and getting exposed to ``all the things that can go wrong''.

\begin{aosasect2}{The Invisible Hand}
The software should look like written by a single person. The best developers
are the ones that write code that can be taken over by anybody else, should
they be taken down by the ``Proverbial Bus'' when crossing a street. We have
grown to recognize that any trace of ``personal touch'' is an indication of a
defect introduced in the software.

In order to enforce and promote code style uniformity, the following tools have
proved to be very effective:

\begin{aosaitemize}
\item KWStyle\footnote{\url{http://public.kitware.com/KWStyle}}: for automatic
style checking. This is a simplified C++ parser that checks coding style and
flags any violations.
\item Gerrit\footnote{\url{http://code.google.com/p/gerrit}}: for regular code
reviews. This tools serves two purposes: In one hand, it prevents immature
code from entering the code base, by distilling its errors, deficiencies and
imperfections during iterative review cycles where other developers contribute
to improve the code. On the other hand, the tool also provide a virtual
training camp in which new developers get to learn from more experienced
developers (read ``experienced'' as: \emph{those who have made all the
mistakes, and know where the bodies are buried\ldots}) on how to improve the
code and avoid known problems that have been observed during maintenance
cycles.
\item Git hooks: that enforce the use of the KWStyle and Gerrit and that also
perform some checks of their own. For example, ITK uses Git hooks that prevent
commits of code with tabs, and with trailing blank spaces.
\end{aosaitemize}

The team has also explored the use of
``Uncrustify''\footnote{\url{http://uncrustify.sourceforge.net}} as a tool for
enforcing a consistent style. It is worth emphasizing that uniformity of style
is not a simple matter of aesthetic appeal. It is really a matter of economics,
since it is well known that the subsequent cost of maintenance of the code is
about 13 times the cost of the initial development, and that about 80\% of that
maintenance cost is the time that developers dedicate to reading someone else's
code, trying to figure out what it was supposed to do. Uniform style does
wonders for reducing the time that it takes for developers to immerse
themselves into a newly open file and get to understand the code before they
introduce any modifications in it. By the same token, it also reduces the
chances that subsequent developers will misinterpret the code and that their
modifications end up introducing new bugs when they were honestly trying to fix
old bugs.

The key for making these tools effective is to make sure that they are:
\begin{aosaitemize}
\item Available to all developers, hence our preference for Open Source tools.
\item Run on a regular basis. In the case of ITK, these tools have been
integrated in the Nightly and Continuous Dashboard builds managed by
CDash\footnote{\url{http://www.cdash.org/CDash/index.php?project=Insight}}.
\item Run as close as possible to the point where the code is being written,
since in this way, deviations can be fixed immediately, and developers train
themselves by learning what kind of practices break style rules.
\end{aosaitemize}

\end{aosasect2}

\begin{aosasect2}{Refactoring}
ITK Started in the year 2000 and grew continuously until the year 2010. In
2011, thanks to an infusion of Federal funding investment, the development team
had the truly unique opportunity to embark in a refactoring effort. The funding
was provided by the National Library of Medicine as part of the initiative of
the American Recovery and Reinvestment Act (ARRA). This is not a minor feat.
Once you have been working on a piece of software for over a decade, and you
are offered the opportunity to clean it up: What would you change ?.

This opportunity for widespread refactoring is very rare. For the previous ten
years of the Toolkit, we relied on the daily effort of performing small local
refactorings by cleaning up specific corners of the toolkit as we run into
them.  This continuous process of clean up and improvement takes advantage of
the massive collaboration of open source communities, and it is safely enabled
by the testing infrastructure driven by CDash, that exercises on a regular
basis about 84\% of the code in the toolkit. Note that in contrast, the average
code coverage of the software industry is estimated to be only 50\%.

Among the many things that were changed in the refactoring effort, the ones
that are more relevant to the architecture are:

\begin{aosaitemize}
\item Modularization was introduced in the toolkit
\item Interger types were standardized
\item Typedefs where fixed to allow management of image larger than 4Gb in all platforms
\item The software process was revised
\begin{aosaitemize}
\item Migrated from CVS to Git
\item Introduced Code Review with Gerrit
\item Introduced testing on demand with CDash@home
\item Improved method for downloading data for unit testing
\end{aosaitemize}
\item Deprecated support for obsolete compilers
\item Improved support for many IO image file formats including
\begin{aosaitemize}
\item DICOM
\item JPEG2000
\item TIFF (BigTIFF)
\item HDF5
\end{aosaitemize}
\item Introduced a framework for supporting GPU computation
\item Introduced support for video processing
\begin{aosaitemize}
\item Added a bridge to OpenCV
\item Added a bridge to VXL
\end{aosaitemize}

\end{aosaitemize}

Maintenance based on incremental modifications works fine for the
local improvement of specific C++ classes. Tasks such as adding
features to an image filter, improving performance of a given
algorithm, addressing bug reports, and improving documentation of
specific image filters. However, a massive refactoring is needed for
infrastructure modifications that affect a large number of classes
across the board, such as the ones that we have listed above. To cite
specific examples, the modifications needed to support images larger
than 4Gb was probably one of the largest patches ever applied to ITK.
It required the modification of hundreds of classes and could not have
been done incrementally without incurring in a great deal of pain. The
modularization is another clear example of a task that could not have
been done incrementally. It trully affected the entire organization of
the toolkit, how its testing infrastructure works, how testing data
was managed, how the toolkit is packaged and distributed, and how new
contributions will be encapsulated to be added to the toolkit in the
future.

\end{aosasect2}

\begin{aosasect2}{Reproducibility}

One of the early lessons learned in ITK was that the many papers published in
the field were not as easy to implement as we were led to believe.  The
computational field tend to over-celebrate algorithms and to dismiss the
practical work of writing software as \emph{``just an implementation detail''}.

That dismissive attitude is quite damaging to the field, since it diminishes the
importance of the first-hand experience with the code and its proper use. The
outcome is that most published papers are simply not reproducible, and that
when researchers and students attempt to use such techniques, they end up
spending a lot of time in the process and deliver \emph{variations} of the
original work. It is actually quite difficult in practice to verify if an
implementation matches what was described in a paper.

ITK disrupted, for the good, that environment and restored a culture of DYI
(\emph{Do It Yourself}), in a field that has grown accustomed to theoretical
reasoning, and that had learned to dismiss experimental work. The new culture
brought by ITK is a practical and pragmatic one in which the virtues of the
software are judged by its practical results and not by the appearance of
complexity that is celebrated in some scientific publications. Helas, it turns
out that in practice, the most effective processing methods are those that
would appear to be too simple to be accepted for a scientific paper.

The culture of Reproducibility is a continuation of the philosophy of Test
Driven Development, and systematically results in better software. Higher
clarity, readability, robustness and focus.

In order to fill the gap of lack reproducible publications, the ITK community
created the Insight Journal\footnote{\url{http://www.insight-journal.org}}.
This is an Open Access, fully online publication, in which contributions are
required to include code, data, parameters, and tests in order to enable the
verification of reproducibility. Articles are published online in less than 24
hours after submission. Then, they are made available for peer-review by any
member of the community. Readers get full access to all the materials
accompanying the article, namely: source code, data, parameters, and testing
scripts. The Journal have provided a prolific space for sharing new code
contributions, and from there, make their way into the code base. The Journal
recently received its 500th Article and continues to be a used as the official
entry gate for new code to be added to the ITK toolkit.
\end{aosasect2}

\end{aosasect1}

\end{aosachapter}
