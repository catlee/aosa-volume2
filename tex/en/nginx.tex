\begin{aosachapter}{nginx}{s:nginx}{Andrey Alexeev}

NGINX (``engine x'') is a free open source web server written by Igor
Sysoev, a Russian software engineer. Since its public launch in 2004,
NGINX has been focusing on high performance, high concurrency and low
memory usage. Additional features on top of the web server
functionality, like load balancing, caching, access and bandwidth
control, ability to integrate efficiently with a variety of
applications---have helped to make NGINX a good choice for modern web
site architectures. Currently NGINX is the second most popular open
source web server on the Internet.

\begin{aosasect1}{Why Is Ensuring High Concurrency Important?}

These days the Internet is so wide-spread and ubiquitous it's hard to
imagine it wasn't exactly there a decade ago. It has greatly
evolved---literally from a simple HTML filled clickable text, based on
NCSA and then on Apache web servers, to an always-on communication
media used by more than 2 billion users worldwide. With the the
proliferation of permanently connected PCs, mobile devices and
recently tablets, Internet landscape is rapidly changing and the
entire economies become digitally wired. Online services have become
much more elaborate with a clear bias towards instantly available live
information and entertainment. Security aspects of running online
business have also significantly changed. Accordingly web sites are
now much more complex than before, and generally require a lot more
engineering efforts to be robust and scalable.

On of the biggest challenges for a web site architect has always been
concurrency. Since the beginning of web services, the levels of
concurrency have been continuously growing. It's not uncommon for a
popular web site to serve hundreds of thousand and even millions of
simultaneous users. A decade ago the major cause of concurrency were
slow clients---users with ADSL or dial-up connections. Nowadays,
concurrency is caused by a combination of the mobile clients and by
the newer application architectures which are typically based on
maintaining a persistent connection allowing the client to be quickly
updated with the news, tweets, friend feeds etc. Another important
factor contributing to increased concurrency is the changed behavior
of modern browsers, which would open 4-6 simultaneous connections to a
web site to improve page load speed.

To illustrate the problem with slow clients, imagine a simple
Apache-based web server which produces a relatively short 100KB
answer---a web page with a text or an image. It can be merely a
fraction of a second to generate or retrieve this page, but then it
takes 10 seconds to transmit it to the client with a bandwidth of 80
Kbps (10 KB/s). Essentially, the web server would relatively quickly
pull a 100 KB content, and then it would be busy for 10 seconds slowly
sending this content to the client before freeing client's
connection. Now imagine that you have 1,000 simultaneously connected
clients who have requested a similar content. If only 1 MB of
additional memory is allocated per client, it would result in 1000 MB
(approx. 1GB) of extra memory occupied by serving just 1,000 clients
with a 100 KB content. In reality, a typical web server based on
Apache commonly allocates more than 1 MB of additional memory per
connection, and regrettably tens of Kbps is still often the effective
speed of mobile communications. Although the situation with sending
content to a slow client might be to some extent improved by
increasing the size of operating system kernel socket buffers, it's
not a general solution to the problem and can have undesirable
side-effects.

With the persistent connections the problem of handling concurrency is
even more pronounced, because to avoid latency associated with
establishing new HTTP connections, clients would stay connected, and
for each connected client there's a certain amount of memory allocated
by the web server.

Consequently, to handle increasing workloads associated with the
growing audiences and hence higher levels of concurrency---and to be
able to continuously doing so, a web site should be based on a number
of very efficient building blocks. While the other parts of the
equation such as hardware (CPU, memory, disks), network capacity,
application and data storage architectures are obviously important, it
is the web server software where client connections are accepted and
processed accordingly. Thus, the web server should be able to scale
non-linearly with the growing number of simultaneous connections and
requests per second.

\begin{aosasect2}{Isn't Apache Suitable?}

Apache---the web server software that still largely dominates the
Internet today has its roots in the beginning of the
1990's. Originally its architecture matched the then existing
operating systems and hardware, but also the state of the prior
Internet, where a web site was typically a standalone physical server
running a single instance of Apache. By the beginning of the 2000's it
was obvious that a model of such standalone web server cannot be
easily replicated to satisfy the needs of the growing web
services. Although Apache provided a solid foundation for the future
development, its architecture of spawning a copy of itself per each
new connection wasn't really suitable to ensure non-linear scalability
of a web site. Eventually Apache became a general purpose web server
focusing on having many different features, a variety of 3rd party
extensions, and being universally applicable to practically any kind
of web application development. However, nothing comes without a price
and the downside of having such a rich and universal combination of
tools in a single piece of software is less scalability because of
increased CPU and memory usage per connection.

Thus, when the server hardware, operating systems and network
resources eventually ceased to be major constraints for a web site
growth, web developers worldwide started to look around for a more
efficient means of running web servers. Around 10 years ago Daniel
Kegel, a prominent software engineer, proclaimed that ``it's time for
web servers to handle ten thousands clients simultaneously'' (see
\url{http://www.kegel.com/c10k.html}) and made predictions for what we
currently see as Internet cloud services. Kegel's C10K manifest
produced a number of attempts to solve the problem of web server
optimization to handle a large number of clients at the same time, and
NGINX turned out to be one of the most successful ones.

Aimed at solving C10K problem with up to 10,000 simultaneous
connections NGINX was written with a different architecture in
mind---the one which is much more suitable for non-linear scalability
with both the number of simultaneous connections and the requests per
second. NGINX is event-based, so it doesn't follow Apache's style
of spawning new processes or threads for each web page request . The
end result is that even as the load increases, memory and CPU usage
remain predictable. NGINX can now deliver tens of thousands of
concurrent connections per a generic hardware server.

When the first version of NGINX appeared, it was meant to be deployed
alongside Apache with the static content like HTML, CSS, JavaScript
and images moved to NGINX to offload concurrency and latency
processing from Apache-based application servers. Over the course of
its development NGINX has added integration with the applications
through the use of FastCGI, uswgi or SCGI protocols, and with the
distributed memory object caching systems like memcached. Other useful
functionality like reverse proxy with load balancing and caching was
added as well. These additional features have shaped into a very
efficient combination of tools to build a scalable web infrastructure
upon.

\end{aosasect2}

\begin{aosasect2}{Are There More Advantages of Using NGINX?}

Handling high concurrency with high performance and efficiency has
always been one of the key benefits of deploying NGINX, however there
are now even more interesting things to observe.

In the last few years, web architects have embraced the idea of
decoupling and separating their application infrastructure and the web
servers. However, what would previously exist in the form of a LAMP
(Linux, Apache, MySQL, PHP, Python or Perl) based web site, might now
become not merely a LEMP based one (with the 'E' stemming from 'engine
x'), but more and more often it's about pushing a web server to the
edge of the infrastructure and integrating the same or revamped set of
applications and database tools around it in a different way.

NGINX is very well suited for this, as it provides the key features
necessary to conveniently offload concurrency, latency processing, SSL
(Secure Sockets Layer), static content, compression and caching,
connections and requests throttling, and even HTTP media streaming
from the application layer to a much more efficient edge web server
layer. It also allows to integrate directly with memcached/Redis or
other 'NoSQL' solution to boost the performance with serving large
number of concurrent users.

These days with the new flavors of development kits and programming
languages appearing, more and more companies are changing their habits
of developing and deploying new applications. NGINX has become one of
the most important components of these changing paradigms, and it
already helped many companies to start and develop their web services
in time and within their budgets.

The first lines of NGINX were written in 2002. In 2004 it was released
to the public under the 2-clause BSD license.  The audience of NGINX
users has been growing ever since, contributing ideas, submitting bug
reports, suggestions and observations that all have been immensely
helpful and ultimately beneficial for the entire community.

NGINX codebase is original and was written entirely from scratch in
the C programming language. NGINX has been ported to many
architectures and operating systems, including Linux, FreeBSD,
Solaris, Mac OS X, AIX and Microsoft Windows. NGINX has its own
function libraries and with its standard modules doesn't use much
beyond the system's C library, except for zlib, PCRE and OpenSSL which
can be optionally excluded from the build if not needed or because of
potential license conflicts.

A few words about the Windows version of NGINX. While NGINX works
under Windows environment, Windows version of NGINX is more like a
proof-of-concept rather than a fully functional Windows port. There
are certain limitations of both NGINX and Windows kernel architectures
that don't mix quite well at this time. Known issues of NGINX version
for Windows include the following: much less number of concurrent
connections, less performance, no caching and no bandwidth policing
functionality. Future versions of NGINX for Windows will be matching
the mainstream functionality more closely.

\end{aosasect2}

\end{aosasect1}

\begin{aosasect1}{Overview of NGINX architecture}

Traditional process-based or thread-based models of handling
concurrent connections involve handling each connection with a
separate process or thread, blocking on network and input/output
operations until completion. Depending on the application it can be
very inefficient in terms of memory and CPU consumption. Spawning a
separate process or thread requires preparation of a new runtime
environment, including new \emph{malloc} and stack memory areas, and
new execution context. Additional CPU time is also spent on creation
which can eventually lead to a poor performance due to thread
thrashing on excessive context switching. All of the complications
above manifest themselves in an older web server architecture like
Apache's. This is a trade off between offering a rich set of generally
applicable features and optimized usage of hardware server resources.

As from the very beginning NGINX was meant to be a specialized tool to
achieve more performance, density and more economic use of server
resources while enabling most dynamic growth of a web site, it has
followed a different model. It was actually inspired by an ongoing
development of advanced event-based mechanisms in a variety of
operating systems. What came as a result is a modular, event-driven,
asynchronous, single-threaded non-blocking architecture which became
the foundation of NGINX code.

NGINX is using multiplexing and event notifications heavily, and
dedicates specific tasks to separate processes. Connections are
processed in a highly efficient run-loop in a limited number of
single-threaded processes---called \emph{workers}. Within each worker
NGINX can handle many thousands of concurrent connections and requests
per second.

\begin{aosasect2}{Structure of the Code}

NGINX worker's code includes the core and the functional modules. The
core of NGINX is responsible for maintaining a tight run-loop and
executing appropriate sections of modules' code on each stage of
request processing. Modules constitute most of the presentation and
application layer functionality of NGINX. Modules read and write from
the network and storage, transform content, do outbound filtering,
apply server-side include actions and pass the requests to the
upstream servers when proxying is activated.

Modularity of NGINX architecture generally allows to extend the set of
server features without modifying the NGINX core. NGINX modules come
in slightly different incarnations, namely core modules, event
modules, phase handlers, protocols, handlers of variables, filters,
upstreams and load balancers. At this time NGINX doesn't support
dynamically loadable modules, that is---modules are compiled along
with the core at build stage. However, support for loadable modules
and ABI is planned for the future major releases. More detailed
information about the roles of different modules can be found in
``NGINX internals'' section of this document.

While handling a variety of consequent actions associated with
accepting, processing and managing network connections and content
retrieval, NGINX uses event notification mechanisms and a number of
disk I/O performance enablements in Linux, Solaris and BSD-based
operating systems, like \emph{kqueue}, \emph{epoll}, or \emph{event
ports}. The goal here is to provide as many hints to the operating
system as possible, in regards to obtaining a timely asynchronous
feedback for inbound or outbound traffic, disk operations, reading
from or writing to sockets, timeouts and so on. The usage of different
methods for multiplexing and advanced I/O operations is heavily
optimized for every UNIX-based operating system NGINX runs on.

High-level overview of NGINX architecture is presented in
\aosafigref{fig.nginx.arch}.

\aosafigure{../images/nginx/architecture.png}{NGINX Architecture Diagram}{fig.nginx.arch}

\end{aosasect2}

\begin{aosasect2}{Workers Model}

So, NGINX doesn't spawn a process or thread for every
connection. Instead worker processes accept new requests from a shared
listen socket and execute a highly efficient run-loop inside each
worker to process thousands of connections per worker. There's no
arbitration or distributing connections to the workers. Upon a start,
initial set of listening sockets is created. Workers then continuously
accept, read from and write to the sockets while processing HTTP
requests and responses.

The run-loop is the most complicated part of NGINX worker code where
most of the work of processing the requests is done. It includes
comprehensive inner calls and is relying heavily on the idea of
asynchronous handling of tasks. Asynchronous operations are
implemented through modularity, event notifications, extensive use of
callback functions and fine-tuned timers. Overall, the key principle
here is to be as much non-blocking as possible. The only situation
where NGINX can still block is when there's no enough disk storage
performance for a worker.

Because NGINX is free from forking a process or thread per connection,
the memory usage is very conservative and extremely efficient in the
vast majority of cases. NGINX conserves CPU cycles as well because
there's no ongoing create-destroy pattern for processes or
threads. What NGINX is doing is checking the state of network and
storage, initializing new connection, adding it to the run-loop,
processing asynchronously until completion and de-allocating
connection from the run-loop. Combined with the careful use of
\emph{syscall}s and accurate implementation of supporting interfaces
like pool and slab memory allocators, it typically leads to a
relatively moderate-to-low CPU usage even under extreme workloads.

With the model of running several workers to handle connections, NGINX
scales well across multiple cores. Overall, a separate worker per core
allows full utilization of multicore architectures, prevents thread
thrashing and lock-ups. There's no resource starvation and the
resource controlling mechanisms are isolated within single-threaded
worker processes. This model also allows more scalability across
physical storage devices, facilitates more disk utilization and
generally allows to avoid blocking on disk I/O. As a result, the
hardware server resources are also utilized more efficiently with the
workload shared across several workers.

Depending on the load patterns, disk and CPU utilization on the
server, the number of NGINX workers should be adjusted
accordingly. The rules are somewhat basic here, and the system
administrator shall try a couple of configurations against real-life
workloads. General recommendations might be the following. If the load
pattern is CPU intensive---for instance, handling a lot of TCP/IP,
doing SSL, or compression---the number of NGINX workers should match
the number of CPU cores. If the load is mostly disk I/O bound (serving
different sets of content from storage, heavy proxying etc.), the
number of workers might be a product of the number of cores and a
multiplier of 1.5�2, and some engineers are aligning it with the
number of individual storage units instead. Efficiency of the latter
approach depends on the type and configuration of disk storage,
though.

There are certain drawbacks of the existing implementation as
well. One major problem, the developers of NGINX will be solving in
the next versions of the product, is how to avoid most of the blocking
on disk I/O. At this time, if there's no enough storage performance to
serve disk operations generated by a particular worker, such worker
may still block on reading/writing from disk. A number of mechanisms
and configuration file directives exist to mitigate such disk I/O
blocking scenarios. Most notably, combinations of options like
\emph{sendfile} and \emph{AIO} typically produce a lot of headroom for
disk performance. Depending on the data set, amount of memory
available for NGINX, and the underlying storage architecture, a
particular NGINX installation should be planned accordingly.

Another problem of the existing worker model is related to a somewhat
limited support for embedded scripting. For one, with the standard
NGINX distribution, only Perl embedded scripting support is
implemented. There's a simple explanation for that---key problem here
is a possibility of an embedded script to potentially block on any
operation or exit unexpectedly. Both types of behavior would
immediately lead to a situation where the worker is hung, affecting
many thousands of connections at once. More work is planned in regards
to make embedded scripting with NGINX simpler, more reliable and
suitable for a broader range of applications like hosting.

\end{aosasect2}

\begin{aosasect2}{Roles of NGINX Processes}

NGINX runs the following processes in memory. There is a single master
process and several worker processes. There are also a couple of
special purpose processes, namely---cache loader and cache
manager. All processes are single-threaded in version 1.x of
NGINX. All processes primarily use shared memory mechanisms for
IPC. Master process is run as root. Cache loader, cache manager and
the workers run as unprivileged user.

The master process is responsible for the following tasks:

\begin{aosaitemize}

\item reading and validating configuration;

\item creating, binding and closing sockets;

\item starting, terminating and maintaining configured number of
worker processes;

\item reconfiguring without service interruption;

\item controlling non-stop binary upgrades (starting new binary and
rolling back if necessary);

\item re-opening log files;

\item compiling embedded Perl scripts.

\end{aosaitemize}

The worker processes accept, handle and process connections from the
clients, provide reverse proxying and filtering functionality and do
almost everything else that NGINX is capable of. In regards to
monitoring the behavior of NGINX instance, a system administrator
should keep an eye on workers as they are the processes reflecting the
actual day-to-day operations of a web server.

The cache loader process is responsible for checking the on-disk cache
items and populating NGINX in-memory database with the cache
metadata. Essentially, cache loader is preparing NGINX instance to
work with the files already stored on disk in a specially allocated
directory structure. It traverses the directories, checks cache
content metadata, updates the relevant entries in shared memory and
then exits when everything is clean and ready for operations.

The cache manager is mostly responsible for cache expiration and
invalidation. It stays in memory during normal NGINX operations and it
is restarted by the master process in the case of failure.

\end{aosasect2}

\begin{aosasect2}{Brief Overview of NGINX Cache}

Cache in NGINX is implemented in the form of a hierarchical data
storage on a filesystem. Cache keys are configurable, and different
request-specific parameters can be used to control what gets into
cache. Cache keys and cache metadata are stored in the shared memory
segments---which cache loader, cache manager and the workers may
access. Currently there isn't any in-memory caching of files, other
than optimizations implied by the operating system's VFS
mechanisms. Each cached response is placed in a different file on the
filesystem. The hierarchy (levels and naming details) are controlled
through NGINX configuration directives. When a response is written to
the cache directory structure, the path and the name of the file are
derived from MD5 hash of the proxy URL.

What happens when NGINX is going to place content in the cache is the
following. When NGINX reads the response from an upstream server, the
content is first being written to a temporary file outside of cache
directory structure. When NGINX finishes request processing it renames
the temporary file into the cache directory file. If the temporary
files directory for proxying is on another file system, the file will
be copied. Thus it's recommended to keep both temporary and cache
directories on the same file system. It is also quite safe to delete
files from the cache directory structure when they need to be
explicitly purged. There are 3rd party extensions for NGINX available
which make it possible to control cached content remotely, and more
work is planned to integrate such functionality in the main
distribution.

\end{aosasect2}

\end{aosasect1}

\begin{aosasect1}{NGINX Configuration}

Ideas that served as the foundation for NGINX configuration design
were inspired by the Apache experience of the author of NGINX. What
has been primarily learned from the work duties is that a much more
scalable configuration is essential for a web server. The main problem
was seen in maintaining large complicated configs with lots of virtual
servers, directories, locations and datasets. In a relatively big web
setup it can turn into a nightmare if not done properly both at the
application level and by the system engineer himself.

So, from the very beginning NGINX configuration was designed to
simplify day-to-day operations and to provide easy means for further
expansion of web server configuration.

NGINX configuration is kept in a number of plain text files which
would typically reside in \emph{/usr/local/etc/nginx} or
\emph{/etc/nginx} directory. Main configuration file is usually called
\emph{nginx.conf}. To keep it uncluttered, parts of the configuration
can be put in separate files which can be automatically included in
the main one. However, it should be noted here that NGINX doesn't
currently support Apache-style distributed configurations (.htaccess
files). All of the configuration relevant to NGINX web server behavior
should reside in a centralized set of configuration files.

The configuration files are initially read and verified by the master
process. Compiled read-only form of NGINX configuration is available
for use by the worker processes as the latter are forked from the
master process. Configuration structures are automatically shared by
usual virtual memory management mechanisms.

NGINX configuration has several different contexts for \emph{main},
\emph{http}, s\emph{erver}, \emph{upstream}, \emph{location} (and also
\emph{mail} for mail proxy) blocks of directives. Contexts never
overlap. For instance, there's no such thing as putting a location
block in the main block of directives. Also, to avoid unnecessary
ambiguity there isn't anything like a ``global web server''
configuration. NGINX configuration is meant to be clean and logical,
allowing to maintain complicated config files that comprise thousands
of directives. To quote the author, ``Locations, Directories, and
other blocks in the global server configuration are the features I
never liked in Apache, so this is the reason why they were never
implemented in NGINX''.

Configuration syntax, formatting and definitions follow a so-called
C-style convention. Back then this particular approach for making
configuration files had been already a proven recipe for a variety of
open source and commercial software. By its design, C-style
configuration is well suitable for nested descriptions, being logical
and easy to create, read and maintain---and many engineers liked
it. Besides, C-style configuration of NGINX can be easily automated as
well.

While some of the NGINX directives resemble certain parts of Apache
configuration, setting up an NGINX instance is quite a different
experience. For instance, rewrite rules are supported by NGINX, though
it would require an administrator to manually adapt a legacy Apache
rewrite configuration to match NGINX style. The implementation of
rewrite engine differs too.

In general, NGINX settings also provide support for several original
mechanisms that can be very useful as part of a lean web server
configuration. It makes sense to briefly mention variables and
\emph{try\_files} directive here which are somewhat unique to
NGINX. Variables in NGINX were developed to provide additional, even
more powerful mechanism to control run-time configuration of a web
server. Variables are optimized for quick evaluation and are
internally pre-compiled to indices. Evaluation is done on-demand, that
is---value of a variable is typically calculated only once and cached
for the lifetime of a particular request. Variables can be used with
different configuration directives, providing additional flexibility
for describing conditional request processing behavior. The
\emph{try\_files} directive was initially meant to be gradually
replacing conditional \emph{if} configuration statements in a more
proper way, and it was designed to quickly and efficiently try/match
against different URI-to-content mapping. Overall \emph{try\_files}
definition works great and can be extremely efficient and useful. It's
recommended for a reader to thoroughly check the \emph{try\_files}
directive and adopt its use whenever
applicable\footnote{\url{http://nginx.org/en/docs/http/ngx\_http\_core\_module.html#try\_files}}.

\end{aosasect1}

\begin{aosasect1}{NGINX Internals}

In this section an attempt is made to describe some of the internals
of NGINX. It can be important for a system administrator to understand
a general picture of how NGINX works inside, and what is the
relationship between different parts of NGINX code.

As it was mentioned before, NGINX code consists of a core code and a
number of modules. The core of NGINX is responsible for providing the
foundation of the web server, web- and mail reverse proxies
functionality---it enables the use of underlying network protocols,
builds the necessary run-time environment, and ensures seamless
interaction between different modules. However, most of the protocol
and application specific features are done by NGINX modules---not the
core.

Internally NGINX is processing connections through a pipeline---or
chain of modules. In other words for every operation there's a module
which is doing the relevant work like compression, modifying the
content, executing server-side includes, communicating to the upstream
application servers through FastCGI or uwsgi protocols, or talking to
a memcached.

There are a couple of NGINX modules that sit somewhere between the
core and the real ``functional modules''. These modules are ``http''
and ``mail''. These two modules provide an additional level of
abstraction between the core and lower level components. What is
implemented inside these in-between modules, is primarily the handling
of the sequence of events associated with a respective application
layer protocol like HTTP, SMTP or IMAP. In combination with NGINX
core, these upper level modules are responsible for maintaining the
right order of calls to the respective ``functional'' modules. While
HTTP protocol is currently implemented as part of ``http'' module,
there are plans to separate it into a functional module in the future,
due to a necessity to support other protocols like
SPDY\footnote{\url{http://www.chromium.org/spdy}}.

The functional modules can be divided into event modules, phase
handlers, output filters, handlers of variables, protocols, upstreams
and load balancers. Most of these modules complement HTTP
functionality of NGINX, though event modules and protocols are also
used for ``mail''. Event modules provide a particular OS-dependent
event notification mechanism like \emph{kqueue} or \emph{epoll.}
Depending on the operating system capabilities and build
configuration, NGINX uses either this or that event module. Overall,
protocol modules implement HTTPS, TLS/SSL, SMTP, POP3 and IMAP.

A typical HTTP request processing cycle looks like the following:

\begin{aosaitemize}

\item Client sends HTTP request

\item NGINX chooses the appropriate phase handler based on the
configured location matching the request

\item (If applicable because of configuration) load balancer picks an
upstream server for proxying

\item Handler does its job and passes each output buffer to the first
filter

\item First filter passes the output to the second filter

\item Second filter passes the output to third and so on

\item Final response is sent to the client

\end{aosaitemize}

NGINX module invocation is extremely customizable---it is performed
through a series of callbacks using pointers to the executable
functions. Caveat emptor, it may place a big burden on the programmers
who'd like to write their own modules, because it has to be defined
exactly how and when the module should run. There are certain ongoing
activities to make both NGINX API and developers' documentation better
and generally more available too.

Examples of where a module can attach include:

\begin{aosaitemize}

\item Before the configuration file is read and processed

\item For every configuration directive for the location and the
server where it appears

\item When the main configuration is initialized

\item When the server (i.e., host/port) is initialized

\item When the server configuration is merged with the main
configuration

\item When the location configuration is initialized or merged with
its parent server configuration

\item When the master process starts or exits

\item When a new worker process starts or exits

\item When handling a request

\item When filtering the response header and the body

\item When picking, initiating and re-initiating a request to an
upstream server

\item When processing the response from an upstream server

\item When finishing an interaction with an upstream server

\end{aosaitemize}

Inside a worker, a simplified overview of a sequence of actions
leading to the run-loop where the response is generated, looks like
the following:

\begin{aosaenumerate}

\item Begin ngx\_worker\_process\_cycle()

\item Process events with OS specific mechanisms (such as \emph{epoll}
or \emph{kqueue})

\item Accept events and dispatch the relevant actions

\item Process/proxy request header, body

\item Generate response content (header, body), stream it to the
client

\item Finalize request

\item Re-initialize timers and events

\end{aosaenumerate}

The run-loop itself (steps 5 and 6) ensures incremental generation of
a response and streaming it to the client.

A more detailed view of an HTTP request processing might look like
this:

\begin{aosaitemize}

\item Init request processing

\item Process header

\item Process body

\item Call the associated handler

\item Run through the processing phases

\end{aosaitemize}

Which brings us to the phases. When NGINX is handling an HTTP request,
it passes it through a number of processing phases. At each phase
there are handlers to call. In general, phase handlers process a
request and produce the relevant output. Phase handlers are attached
to the locations defined in the configuration file.

Phase handlers typically do four things: get the location
configuration, generate an appropriate response, send the header, and
send the body. A handler has one argument, a specific structure
describing the request. A request structure has a lot of useful
information about the client request, such as the request method, URI,
and the header.

When the HTTP request header is read, NGINX does a lookup of the
associated virtual server configuration. If the virtual server is
found, the request goes through:

\begin{aosaitemize}

\item Server rewrite phase

\item Location phase

\item Location rewrite phase (which can bring the request back to the
previous step)

\item Access control phase

\item Try\_files phase

\item Log phase

\end{aosaitemize}

In an attempt to generate the necessary content as a response to the
request, NGINX is passing the request for processing to a suitable
content handler. Depending on the exact location configuration NGINX
may try so-called unconditional handlers first, like \emph{perl},
\emph{proxy\_pass}, \emph{flv,} \emph{mp4 etc}. If the request doesn't
match the above content handlers, it goes further to be picked by one
of the following handlers---in this exact order of
appearance---\emph{random index}, \emph{index}, \emph{autoindex},
\emph{gzip\_static}, \emph{static}.

The details about indexing modules can be found in the reference
documentation\footnote{\url{http://nginx.org/en/docs/}}, but these are
the modules which handle the requests with the trailing slash. If
there isn't a location to pass the request to any specialized module
like \emph{mp4} or \emph{autoindex}, the content is considered to be
just a file or directory on disk (that is, static) and is served by
\emph{static} content handler. For a directory it would automatically
rewrite the URI so that the trailing slash is always there (and then
issue an HTTP redirect).

After the content handlers content is passed to the filters. Filters
are also attached to the locations, and there can be several filters
configured for a location. Filters do the task of manipulating the
output produced by a handler. The order of filter execution is
determined at the compile time. For the out-of-the-box filters it's
predefined, and for a 3rd party development it can be configured at
build stage. In the existing NGINX implementation filters can only do
outbound changes and there's no current mechanism to write and attach
filters to do input content transformation. Input filtering is
something to appear in the future versions of NGINX.

Filters follow the design pattern which looks like the following---a
filter gets called, starts working, calls the next filter until the
final filter in the chain. After that NGINX finalizes the
response. Filters don't have to wait for the previous filter to
finish. Next filter in chain can start its own work as soon as the
input from the previous one is available (functionally much like in
the Unix pipeline). In turn, the output response being generated can
be passed to the client before the entire response from the upstream
server is received.

There are header filters and the body filters. NGINX feeds the header
and the body of the response to the associated filters separately.

A header filter consists of three basic steps:

\begin{aosaitemize}

\item Decide whether to operate on this response

\item Operate on the response

\item Call the next filter

\end{aosaitemize}

Body filters transform the generated content. Examples of the body
filters include:

\begin{aosaitemize}

\item Server-side includes

\item XSLT filtering

\item Image filtering (for instance, resizing images on-the-fly)

\item Charset modification

\item Gzip

\item Chunked encoding

\end{aosaitemize}

After the filter chain the response is passed to the writer. Along
with the writer there're a couple of additional special purpose
filters, namely---the \emph{copy} filter, and the \emph{postpone}
filter. The \emph{copy} filter is responsible to fill in the memory
buffers with the relevant response content which might be stored in a
proxy temp directory. \emph{Postpone} filter is used with subrequests.

Subrequests are a very important mechanism of request/response
processing. Subrequests are also one of the most powerful aspects of
NGINX. With subrequests NGINX can return the results from a different
URL, than the client has originally requested. Some web frameworks
call this an internal redirect. But NGINX goes further---not only can
filters perform multiple subrequests and combine the outputs into a
single response, but subrequests can be also nested and
hierarchical. A subrequest can perform their own sub-subrequest, and
sub-subrequest can initiate sub-sub-subrequests. Subrequests can map
to files on the hard disk, other handlers, or upstream
servers. Subrequests are most useful for inserting additional content
based on data from the original response. For example, the SSI
(server-side include) module uses a filter to scan the contents of the
returned document, and then replaces \emph{include} directives with
the contents of the specified URLs. Or it can be an example of making
a filter that treats the entire contents of a document as a URL to be
retrieved, and then appends the new document to the URL itself.

Upstream and load balancers are also worth to describe briefly
here. Upstreams are used to implement what can be identified as a
content handler which-is-a-reverse-proxy (\emph{proxy\_pass}
handler). Upstream modules mostly do preparation of the request to be
sent to an upstream server (or ``backend'') and gather the response
from upstream. There're no calls to output filters here, for
instance. What an upstream module exactly does is setting callbacks to
be invoked when the upstream server is ready to be written to and read
from. The following callbacks exist for that:

\begin{aosaitemize}

\item Crafting a request buffer (or a chain of them) to be sent to the
upstream server

\item Re-initializing/Resetting the connection to the upstream server
(which happens right before creating the request again)

\item Processing the first bits of an upstream response, saving
pointers to the payload received from the upstream server

\item Aborting the requests (which happens when the client terminates
prematurely)

\item Finalizing the request when NGINX finishes to read from the
upstream server

\item Trimming the response body (e.g. removing a trailer)

\end{aosaitemize}

Load balancer modules attach to the \emph{proxy\_pass} handler to
provide ability to choose an upstream server, when more than one
upstream server is eligible. Load balancer registers an enabling
configuration file directive, provides additional upstream
initialization functions (to resolve upstream names in DNS etc.),
initializes the connection structures, decides where to route the
requests, updates stats information. Currently NGINX supports two
standard disciplines for load balancing to upstream
servers---round-robin and ip-hash.

Upstream and load balancing handling mechanisms include algorithms to
detect failed upstream servers and to re-route the new requests to the
remaining ones, albeit a lot of additional work is planned to enhance
this functionality. In general, more work on load balancers is
planned, and in the next versions of NGINX the mechanisms of
distributing the load across different upstream servers as well as the
health checks will be greatly improved.

There are also a couple of other interesting modules which provide
additional set of variables for use in the configuration file. While
the variables in NGINX are created and updated across different
modules, there are two modules that are entirely dedicated to
variables---namely, \emph{geo} and \emph{map}. The \emph{geo} module
is used to facilitate tracking of clients based on their ip
addresses. This module can create arbitrary variables that depend on
the client IP addresses. The other module which is \emph{map} allows
to create variables from variables, essentially providing an ability
to do a flexible mapping of hostnames and other run-time
variables. This kind of modules may be called the handlers of
variables.

Memory allocation mechanisms implemented inside a single NGINX worker
to some extent were inspired by Apache. A high-level description of
NGINX memory management would be the following. For each connection
the necessary memory buffers are dynamically allocated, linked, used
for storing and manipulating header and body of the request and the
response, and then freed upon connection release. It is very important
to note that NGINX tries to avoid copying data in memory as much as
possible and most of the data is passed along by pointer values, not
by \emph{memcpy}'ing.

Going a bit deeper, when the response is generated by a module, the
retrieved content is put to a memory buffer which is then added to a
buffer chain link. Subsequent processing works with this buffer chain
link as well. Buffer chains are quite complicated in NGINX because
there are several processing scenarios depending on the module
type. For instance, it can be quite tricky to manage the buffers
precisely while implementing a body filter module. Such module can
only operate on one buffer (chain link) at a time and it must decide
whether to overwrite the input buffer, replace the buffer with a newly
allocated buffer, or insert a new buffer before or after the buffer in
question. To complicate things, sometimes a module will receive
several buffers so that it has an incomplete buffer chain that it must
operate on. However, at this time NGINX provides only a low-level API
for manipulating the buffer chains, so before doing an actual
implementation a 3rd party module developer should become really
fluent with this arcane part of NGINX.

One drawback of the above approach is that for the entire life of a
connection there are memory buffers allocated. So for long-lived
connections some memory is wasted. At the same time for a keepalive
connection NGINX currently maintains only 550 bytes of allocated
memory. An optimization that would be possible in the next releases of
NGINX is to re-use memory buffers for the long-lived connections.

The task of managing memory allocations is done by NGINX pool
allocator. Shared memory areas are used for accept mutex, cache
metadata, SSL session cache and the information associated with
bandwidth policing and management (limits). There is a slab allocator
implemented in NGINX to manage shared memory allocations. To allow
simultaneous safe use of shared memory a number of locking mechanisms
are available (mutexes and semaphores).  In order to organize complex
data structures NGINX also provides a red-black tree
implementation. RB-trees are used to keep cache metadata in shared
memory, track non-regex location definitions and in couple of other
places.

Unfortunately all of the above wasn't really ever described in a
consistent and simple manner, making the job of developing 3rd party
extensions for NGINX quite complicated. Although some good documents
on NGINX internals exist---for instance, those produced by Evan
Miller\footnote{\url{http://www.evanmiller.org/}}, such documents are
based on a huge effort in reverse engineering of NGINX code, and the
implementation of NGINX modules is still more of a black art for many.

Despite certain difficulties associated with 3rd party module
development, NGINX user community recently saw a lot of useful 3rd
party modules. There's for instance, an embedded Lua interpreter
module for NGINX, produced by a group of Chinese
authors\footnote{\url{http://openresty.org}}. There are additional
modules for load balancing, full WebDAV support, advanced cache
control and other interesting 3rd party work that the authors of this
book encourage and will support in the future.

\end{aosasect1}

\begin{aosasect1}{Lessons Learned}

When Igor Sysoev started to write NGINX most of the software enabling
the Internet already existed, and the architecture of such software
typically followed certain definitions of the previous server and
network hardware, operating systems, and past Internet architecture in
general. However this didn't prevent Igor from thinking he might be
able to improve things in the web servers area. So while the first
lesson might seem to be obvious, it's actually in that there's always
a room for improvement.

With the realization in mind of a better web software, Igor had
actually spent a lot of time developing initial code structure and
studying different ways of optimizing the code for a variety of
operating systems. With the ongoing development of a prototype of
NGINX version 2.0 ten years after Igor began his work on NGINX, it is
clear that the initial prototype of a new architecture, and the
initial code structure are vitally important for the future life of a
software product.

Another point worth mentioning is the development should be
focused. The Windows version of NGINX is probably a good example on
how it's worth it to avoid diluting the development efforts on
something that is not either a developer's core competence or the
target application. It is equally applicable to the rewrite engine,
that appeared during several attempts to enhance NGINX with more
features for backward compatibility with the existing legacy setups.

Last but not least, it's worth mentioning that despite NGINX
developer's community is not very large, 3rd party modules and
extensions for NGINX have always been a very important part of its
popularity. The work done by Evan Miller, Piotr Sikora, Valery
Kholodkov, Zhang Yichun (agentzh) and other talented software
engineers has been much appreciated by NGINX user community and the
original developers.

\end{aosasect1}

\end{aosachapter}
