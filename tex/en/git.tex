\begin{aosachapter}{Git}{s:git}{Susan Potter}

Work in progress\ldots

Git is an open source distributed Version Control System (dVCS) that was
born out of the needs and frustrations of the Linux Kernel development
community in 2005.

Git enables the management of a digital body of work (often,
but not limited to, code) across a peer-to-peer network of
collaborating repositories. It supports distributed workflows to
manage a body of work that is either eventually converging or
diverging in nature. A feat not easily accomplished
using a centralized Version Control System (VCS) such as CVS or
Subversion, the two most popular open source centralized VCS
projects as well as similar commercial products.

\begin{aosasect1}{Git's Origin}
To understand Git better it is helpful to understand the circumstances
from which the Git project was started in the Linux Kernel community.

The Linux Kernel was first released in October 1991. It rapidly grew a
community of core developers and contributors by the mid-to-late 1990s. Along
side individuals adopting Linux there were organisations or teams of
developers creating Linux distributions based on top of the Linux Kernel
project by the close of the decade. These distributions would sometimes
temporarily fork the official version of the kernel if patches relevant to
their user base were not yet included in the Kernel project when releasing
new versions of their distribution.

By the late 1990s Torvalds and other core developers voiced concerns
about managing patches from a large number of contributors for the
Kernel codebase using any of the open source VCSes available to them at
that time.

Between late 1999 and early 2005 many of the core developers of the Linux
Kernel opted to use BitKeeper, a proprietary distributed VCS made by
BitMover Inc. This was a contraversial move by arguably the best known open
source project at that time, a decision questioned by Richard Stallman, the
GNU project founder. Several key Linux Kernel developers (including Alan Cox)
refused to use BitKeeper, due to concerns over the proprietary Bitmover
license. By 2002 BitMover provided a way for the Linux BitKeeper servers to
interoperate with the Linux CVS server in a few ways. It offered a
public release and free use of its servers to a shortlist of free and
open source projects including the Linux Kernel. For a few years the
Linux Kernel would be maintained across two VCS systems in this way.

This ended during the first half of 2005, when BitMover retracted its
"Free Use" version for a number of key Linux Kernel developers and Git
was born to fill the void.

In April 2005, days after the BitMover announcement, Linus Torvalds began
development in haste of what was to become Git as we know it today. He began
by writing a collection of scripts to help him manage email patches
to apply one after the other. The aim of this initial collection of scripts
was to fail merges quickly so the maintainer could modify the codebase mid-
patch stream and continue merging contributed patches once cleanly able to.

Torvald's had one philosphical goal for Git - to embody the anti-CVS - plus
three usability goals from the outset:
\begin{aosaitemize}
  \item support distributed workflows like those enabled by BitKeeper
  \item offer safeguards against content corruption
  \item offer high performance
\end{aosaitemize}

Despite BitKeeper influencing the original design of Git, it is implemented
in fundamentally different ways and allows even more distributed and even
local-only workflows, which weren't possible with BitKeeper.

Around this time (c2005) three other open source distributed VCS projects
were initiated, including Mercurial, which was a project covered in volume 1
of this title series.

\end{aosasect1}

\begin{aosasect1}{Version Control System (VCS) Landscape}

Now is a good time to take a step back and look at the alternative VCS
solutions to Git. Understanding these differences will allow us to explore
the architectural choices made while developing Git and appreciating its
design philosphy.

As mentioned previously in this chapter, Git is a distributed VCS. What does
this mean?

On a fundamental level it means that a \emph{work} does not \emph{need} to
have a centralized, all-knowing repository that will be the only source of
truth for the \emph{work}. However, that doesn't prevent us for using it in
this way if we so wish to do so. Though it still may not behave exactly like
some of the centralized VCS solutions you might be familiar with already.

At the heart of every VCS tool is tracking the history of an evolving
\emph{work}. The Revision Control System (RCS) was one of the first popular
VCSes. It was based on tracking revisions of individual files taking care of
storing, retrieving, logging, identifying, and merging revisions. It offered
a number of improvements over the more primitive VCS it was based on,
Source Code Control System (SCCS). It did this by providing an easier
interface and improving performance of retrieving versions for its primary
use case. The Concurrent Versions System (CVS) was built on top of RCS adding
a client/server model, which made sharing changes of the work on a team more
tenable than its predecessor. Subversion was CVS's successor, which added
better support for renaming files/folders and also dropped first class
support of trunking and tags, opting instead for an unenforced repository
directory structure convention. Subversion continued CVS's centralized
client/server approach.

This VCS product family only support linear histories. Where later
versions supercede earlier versions. Git is not part of this family; there is
no easy conceptual evolution from Subversion to Git in terms of tracking
histories or content.

Instead Git enables full branching capability using directed acyclic
graphs (DAG). The history of a file is linked all the way
up its directory structure (via nodes representing directories) to the root
directory, which is then linked to a commit node. This commit node in turn
can have one or more parents (more on this later). This affords us two
properties that allow us to reason about our history and content in
more definite ways than the RCS family. Namely:
\begin{aosaitemize}
  \item When a content (file or directory) node in the graph has the same
  reference identity (the SHA in Git) as that in a different commit, the two
  nodes are guaranteed to contain the same content. Allowing Git to
  short-circuit content diffing efficiently.
  \item When merging two branches we are merging the content of two nodes
  in a DAG. The DAG allows Git to "efficiently" (as compared to the
  RCS family of VCS approach) determine common ancestors.
\end{aosaitemize}

As you can see most DAGs represented in Git are the special case of rooted
trees.

This diagram shows two dimensions of VCS variety. Namely distribution mode
and storage mode.

DIAGRAM HERE!!!

There are three primary distribution modes for VCSes: local, client/server
and distributed (or peer-to-peer). Some VCSes can support all of these, some
just one.

On the other axis we can see two primary categories for storage modes:
delta based and directed acyclic graph based.

As we can see Git is a VCS that can support local, client/server and
peer-to-peer distribution modes and uses DAG based content storage.

\end{aosasect1}


\begin{aosasect1}{The Toolkit}

Today the Git ecosystem posesses many GUIs on a number of operating systems.
These are mostly built on top of the Git core toolkit.

Due to the way Git was originally written by Linus and its inception within
the Linux community it was written with a toolkit design philosphy very much
in the Unix tradition of command line tools.

The Git toolkit is divided into two parts: the plumbing and
the porcelain. The plumbing consists of low-level commands that enable
the manipulation of directed acyclic graphs (DAG) and basic content
tracking. The porcelain is the smaller subset of git commands that most
end-users of Git are likely to need to use for maintaining repositories and
communicating between repositories for collaboration.

While the toolkit design has provided enough commands to offer fine grained
access to functionality for many scripters, application developers
complained about the lack of a linkable library for Git. Since the Git binary
calls die(), it was not reentrant and GUIs, web interfaces or longer running
services would have to fork/exec a call to the Git binary, which can be slow.

Shawn Pearce spearheaded an effort to create a linkable Git library with
more permissive licensing that didn't inhibit use of the library. This was
called libgit2. It didn't find much traction until a student named, Vincent
Marti chose it for his Google Summer of Code project last year. Since then
Vincent and GitHub have continued contributing to the libgit2 project and
created Ruby bindings for it in a project called Rugged. More recently
Python bindings around libgit2 have emerged in an open source project
called pygit2. These three open source projects are maintained independently
of the Git core project.

As you can see today there is a wide array of ways to integrate with Git.
From the plumbing portion of the toolkit, procelain layer and now the
linkable library, libgit2 and its offshoots.
\end{aosasect1}

\begin{aosasect1}{The Repository, Index and Working Areas}

Let us get our hands dirty and dive into using Git locally, if only to
understand a few fundamental concepts.

First to create a new initialized Git repository on our local filesystem
(using a Unix inspired operating system) we can do:
\begin{aosaitemize}
  \item \code{mkdir testgit}
  \item \code{cd testgit}
  \item \code{git init}
\end{aosaitemize}

Now we have an empty, but initialized Git repository sitting in our testgit
directory. We can branch, commit, tag and even communicate with other local
and remote Git repositories. Even communication with other types of VCS
repositories is possible with just a handful of \code{git} commands.

The \code{git init} command creates a .git subdirectory inside of testgit.
Let us have a peak inside of it:
\begin{aosaitemize}
  \item \code{tree .git/}\newline
  \code{
.git/\newline
|-- HEAD\newline
|-- config\newline
|-- description\newline
|-- hooks\newline
|   |-- applypatch-msg.sample\newline
|   |-- commit-msg.sample\newline
|   |-- post-commit.sample\newline
|   |-- post-receive.sample\newline
|   |-- post-update.sample\newline
|   |-- pre-applypatch.sample\newline
|   |-- pre-commit.sample\newline
|   |-- pre-rebase.sample\newline
|   |-- prepare-commit-msg.sample\newline
|   |-- update.sample\newline
|-- info\newline
|   |-- exclude\newline
|-- objects\newline
|   |-- info\newline
|   |-- pack\newline
|-- refs\newline
    |-- heads\newline
    |-- tags\newline
}
\end{aosaitemize}

The .git directory above is by default a subdirectory of the root working
directory, testgit. It contains a few different types of files and
directories:

\begin{aosaitemize}
  \item \emph{Configuration}: the .git/config, .git/description and
  .git/info/exclude files essentially help configure the local repository.
  \item \emph{Hooks}: the .git/hooks directory contains scripts that can
  be run on certain lifecycle events of the repository.
  \item \emph{Staging Area}: the .git/index file (which is not yet
  present in our tree listing above) will provide a staging area for our
  working directory.
  \item \emph{Object Database}: the .git/objects directory is the default
  Git object database, which contains all content or pointers to local
  content.
  \item \emph{References}: the .git/refs directory is the default location
  for storing reference pointers for both local and remote branches, tags and
  heads.
\end{aosaitemize}

This is the actual repository. The directory that contains the working set
of files is the \emph{working directory}, which is typically the parent of
the .git directory (or \emph{repository}). If you were creating a Git
remote repository that wouldn't have a working directory you could
initialize it using the \code{git init --bare} command. This would create
just the pared down repository files at the root, instead of creating it
as a subdirectory under the working tree.

Another file of great importance is the \emph{Git index}. It provides the
staging area between the local working directory and the local repository.
The index is used to stage specific changes within a file (or more) to
be committed all together. Even if you make changes related to various types
of features, the commits can be made with like changes together to more
logically describe them in the commit message. To selectively stage
specific changes in a file or set of files you can using \code{git add -p}.

The \emph{Git index}, by default, is stored as a single file inside the
repository directory. The paths to these three areas can be customized
using the following environment variables:
\begin{aosaitemize}
  \item \code{GIT\_DIR}: sets the repository directory or the \emph{.git}
  directory.
  \item \code{GIT\_INDEX}: sets the path to the \emph{index}. This is often
  referred to as the staging area as well.
  \item \code{GIT\_WORK\_DIR}: sets the path to the \emph{working directory}
  for the local Git repository. The work directory represents the "live"
  state of the current head (or HEAD) of the Git repository. This is usually
  a reference to the current branch.
\end{aosaitemize}

It is helpful to understand the interactions that take place between these
three areas (the repository, index and working areas) during the execution
of a few core Git commands:

\begin{aosaitemize}
  \item \code{git checkout [branch]} \newline
  \small{This will move the HEAD reference of the local repository to branch
  reference path (e.g. \code{refs/heads/master}), populate the index with
  this head data and refresh the working directory to represent the tree
  at that head.}
  \item \code{git add [files]} \newline
  \small{This will cross reference the checksums of the \emph{files}
  specified with the corresponding entries in the Git index to see if the
  index for staged files needs updating with the working directory's
  version. Nothing changes in the Git directory (or repository).}
\end{aosaitemize}

Let us explore what this means more concretely by inspecting the contents of
files under the .git directory (or repository).

\begin{aosaitemize}
  \item \code{GIT\_DIR=\${PWD}/.git}
  \item \code{cat \${GIT\_DIR}/HEAD}
\begin{verbatim}
ref: refs/heads/master
\end{verbatim}
  \item \code{MY\_CURRENT\_BRANCH=\$(cat .git/HEAD | sed 's/ref: //g')}
  \item \code{cat \${GIT\_DIR}/\${MY\_CURRENT\_BRANCH}}
\begin{verbatim}
cat: .git/refs/heads/master: No such file or directory
\end{verbatim}
\end{aosaitemize}

We get an error because, before making any commits to a Git repository at
all, no branches exist, except the default branch in Git is \code{master},
whether it exists yet or not.

We can verify that no branches exist yet by attempting to list the branches
in our repository with \code{git branch}.

Now if we make a new commit then the master branch is created by default for
this commit. Let us do this (continuing in the same shell, retaining
history and context):

\begin{aosaitemize}
  \item \code{git commit -m "Initial empty commit" --allow-empty}
  \item \code{git branch}
\begin{verbatim}
* master
\end{verbatim}
  \item \code{cat \${GIT\_DIR}/\${MY\_CURRENT\_BRANCH}}
\begin{verbatim}
3bce5b130b17b7ce2f98d17b2998e32b1bc29d68
\end{verbatim}
  \item \code{git cat-file -p \$(cat \${GIT\_DIR}/\${MY\_CURRENT\_BRANCH})}
\end{aosaitemize}

What we are starting to see here is the content representation inside Git's
object database.

\end{aosasect1}

\begin{aosasect1}{The Object Database}

\aosafigure{../images/git/object-hierarchy.png}{Git Objects}{fig.git.objects}

Git has four basic primitive objects that every type of content in the
local repository is built around. Each object type has the following
attributes: \emph{type}, \emph{size} and \emph{content}. The primitive object
types are:
\begin{aosaitemize}
  \item \emph{Tree}: elements in a tree can be another tree or a blob when
  representing a content directory.
  \item \emph{Blob}: a blob represents a file stored in the repository.
  \item \emph{Commit}: a commit points to a tree representing the top level
  directory for that commit as well as parent commits and standard
  attributes.
  \item \emph{Tag}: a tag has a name and points to a commit at the point in
  the repository history that the tag represents
\end{aosaitemize}

All object primitives are referenced by a SHA, a 40-digit object identity,
which has the following properties:
\begin{aosaitemize}
  \item if two objects are identical they will have the same SHA
  \item if two objects are different they will have different SHAs
  \item if an object was only copied partially or another form of data
        corruption occurred, recalculating the SHA of the current object
        will identify such corruption
\end{aosaitemize}

The first two properties of the SHA relating to identity of the objects is
most useful to enable Git's distributed model (the second goal of Git).
The latter property enables some safegaurds against corruption (the third
goal of Git above).

It should be noted that despite the desirable properties of using DAG based
storage for content tracking it is likely for many repositories that delta
storage will be more space efficient. Git tackles this by "packing" objects
in a compressed format using an pack index file to point to offsets to
specific objects in the corresponding pack file.

TODO: More to discuss here. Use plumbing commands like 'git show' and 'git
cat-file' to demonstrate how the primitives work together and build on top of
each other.

\end{aosasect1}

\begin{aosasect1}{Merging and Rebasing}

TODO: Step through a merge in diagrams and then contrast this to a rebase.

\end{aosasect1}

\begin{aosasect1}{Explaining Resets}

TODO: Explain resets in terms of the repository, index and working areas.

\end{aosasect1}

\begin{aosasect1}{Distributed Workflows}

TODO: Describe some distributed workflows enabled by Git.

\end{aosasect1}

\begin{aosasect1}{Future Work}

TODO/NOTES:
Discuss the work being done now on using different object database backends
for storage, from Redis, Memcached, SQLite and others.

\end{aosasect1}

\end{aosachapter}
