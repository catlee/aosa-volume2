\begin{aosachapter}{Git}{s:git}{Susan Potter}

\begin{aosasect1}{Git in a Nutshell}

Git enables the evolution of a digital body of work (often,
but not limited to, code) with many collaborators
using a peer-to-peer network of collaborating repositories. It
supports distributed workflows to manage a body of work that is
either eventually converging or temporarily diverging.

This chapter will show how various aspects of Git work under the covers
to enable this and how this differs from other VCSes.

\end{aosasect1}

\begin{aosasect1}{Git's Origin}
To understand Git's design philosophy better it is helpful to understand the
circumstances from which the Git project was started in the Linux Kernel
community, which had initially been maintained via tarballs and patches for
years initially.

The Linux kernel was unusual to most commercial software projects at that
time due to the large number of committers and the high variance of
contributors with respect to involvement and knowledge in the existing
codebase. The core development community struggled to find a VCS that
satisfied most of their needs. Initially the version control systems
reviewed were open source projects themselves and they were from the linear
history lineage such as SCCS, RCS, and CVS.

Git is an open source project that was born out of the needs and
frustrations of the Linux Kernel development community in 2005. At that time
the Linux kernel codebase had been managed across two VCS systems (BitKeeper
and CVS) by different core developers. BitKeeper offered a different
view of VCS history lineage to the popular open source VCS projects at this
time.

BitMover, the company that built BitKeeper, provided free licenses and VCS
hosting services to some core developers on the Linux kernel until early 2005
when it announced it would no longer provide free licenses or VCS hosting
services as of mid-2005.

In April 2005, days after this BitMover announcement, Linus Torvalds began
development in haste of what was to become Git as we know it today. He began
by writing a collection of scripts to help him manage email patches to apply
one after the other. The aim of this initial collection of scripts was to
fail merges quickly so the maintainer could modify the codebase mid-patch
stream and continue merging contributed patches once cleanly able to.

Torvald's had one philosphical goal for Git - to embody the anti-CVS - plus
three usability design goals from the outset:
\begin{aosaitemize}
  \item support distributed workflows like those enabled by BitKeeper
  \item offer safeguards against content corruption
  \item offer high performance
\end{aosaitemize}

These design goals have been accomplished and maintained to a degree as I
will attempt to show by dissecting Git's use of DAGs for content storage,
reference pointers for heads, object model representatation, remote protocol
and how it tracks the merging of trees.

Despite BitKeeper influencing the original design of Git, it is implemented
in fundamentally different ways and allows even more distributed plus
local-only workflows, which were not possible with BitKeeper, which requires
a server.

Around this time (circa 2005) three other open source distributed VCS projects
were initiated, including Mercurial, which is a project covered in volume 1
of this title series. All of these dCVS tools offer slightly different ways
to enable distributed workflows, which centralized VCS systems before them
were not capable of handling directly. \emph{Note: Subversion has an
extension maintained by different developers to support server-to-server
synchronization named SVK.}

Open source DVCS projects today include Bazaar, Darcs, Fossil, Git,
Mercurial, and Veracity.

\end{aosasect1}

\begin{aosasect1}{Version Control System (VCS) Design}

Now is a good time to take a step back and look at the alternative VCS
solutions to Git. Understanding these differences will allow us to explore
the architectural choices faced while developing Git.

A version control system usually has three core functional
requirements, namely:
\begin{aosaitemize}
  \item storing content
  \item tracking changes to the content (history including merge metadata)
  \item distributing the content and history with collaborators
\end{aosaitemize}

\emph{Note: The third requirement above is not a functional requirement for
all VCSs.}

\begin{aosasect2}{Content Storage}

The most common design choices for storing content in the VCS world are:
\begin{aosaitemize}
  \item delta based
  \item directed acyclic graph (DAG)
\end{aosaitemize}

Git stores the content as a directed acyclic graph (DAG). Where a hierarchy
mimicking the filesystem's directory structure is created with objects of
different types (see 'Object Database' below).

\end{aosasect2}
\begin{aosasect2}{Commit and Merge Histories}

On the history and change tracking front most VCS software uses one of
the following approaches to solve this:
\begin{aosaitemize}
  \item linear history
  \item directed acyclic graph (DAG) for history
\end{aosaitemize}

Again Git uses a DAG, this time to store its history. Each commit contains
metadata about its ancestors. A commit in Git can have zero or many
(theoretically unlimited) parent commits. For example, the first commit
in a Git repository would have zero parents. The result of a three way merge
would have three parents.

Another primary difference between Git, Subversion and its linear history
ancenstors, is its ability to directly support branching that will record
most merge history cases.

\aosafigure{../images/git/dag-example.png}{Example of a DAG representation
  in Git}{fig.git.dag}

Git enables full branching capability using directed acyclic
graphs (DAG) to store content. The history of a file is linked all the way
up its directory structure (via nodes representing directories) to the root
directory, which is then linked to a commit node. This commit node in turn
can have one or more parents. This affords Git two
properties that allow us to reason about our history and content in
more definite ways than the RCS family. Namely:
\begin{aosaitemize}
  \item When a content (file or directory) node in the graph has the same
  reference identity (the SHA in Git) as that in a different commit, the two
  nodes are guaranteed to contain the same content. Allowing Git to
  short-circuit content diffing efficiently.
  \item When merging two branches we are merging the content of two nodes
  in a DAG. The DAG allows Git to "efficiently" (as compared to the
  RCS family of VCS approach) determine common ancestors.
\end{aosaitemize}

\end{aosasect2}
\begin{aosasect2}{Distribution}

To tackle content distribution of a working copy to collaborators on a
project VCS solutions have handled this in one of three ways:
\begin{aosaitemize}
  \item local-only: this would be for VCS solutions that do not make the
    third functional requirement above
  \item central server: where all changes to the repository must transact
    via one specific repository for it to be recorded in history at all.
  \item distributed model: where there will often be publically accessible
    repositories for collaborators to "push" to but commits can be made
    locally and pushed to these public nodes later. Allowing offline work.
\end{aosaitemize}

\end{aosasect2}

To demonstrate the benefits and limitations of each major design choice
for VCS functional needs, we will consider a Subversion and Git repository
(on a server), content wise they are equivalent (the HEAD of the default
branch in the Git repository has the equivalent content as the Subversion
repository's latest revision on trunk). A developer, named Alex,
has a local checkout of the Subversion repository and a local clone of the
Git repository locally.

Let us say Alex makes a change to a 1MB file in the local Subversion
checkout then commits the change. Locally the checkout of the file mimicks
the latest change and local meta data is updated. In the centralized
Subversion repository, during Alex's commit, a diff is generated between the
previous snapshot of the files and the new changes and this delta is stored
in the repository.

Contrast this with the way Git works.
When Alex makes the same modification to the equivalent file in the local
Git clone, the change will be recorded locally first then Alex can "push"
the local pending commits to a public repository so the work can be shared
more easily with other collaborators on the project. The content changes are
stored identically for each Git repository that the commit exists in. Upon
the local commit (the simplest case), the local Git repository will create a
new object representing a file for the changed file (with all its content
inside). For each directory above the changed file (plus the repository
root directory) a new tree object is created with a new identifier. A DAG
is created starting from the newly created root tree object pointing to
blobs (reusing existing blob references where the files content has not
changed in this commit) and referencing the newly created blob in place
of that file's previous blob object in the previous tree hierarchy. At this
point the commit is still just local to the current Git clone on Alex's
local device. When Alex "pushes" the commit to a publically accessible
Git repository this commit gets sent to that repository. After the public
repository verifies the commit can apply to the branch the same objects
are stored in the public repository as were originally created in the
local Git repository.

There are a lot more moving parts in the Git scenario, both under the
covers and two user interactions with the tool as opposed to just one
to share the commit with others. However, there are benefits to this
model such as:
\begin{aosaitemize}
  \item providing the ability for the collaborators to work offline and
    commit incrementally as they work offline
  \item allowing the collaborator to determine when his/her work is
    ready to share
  \item offer the collaborator history access to the repository when
    offline
\end{aosaitemize}

In the Subversion scenario the collaborator did not have to remember
to push to the public remote repository when ready for others to
view the changes made. When a small modification to a larger file is sent
to the central repository in Subversion the delta stored is much more
efficient than storing the complete file contents for each version.
However, as we will see later there is a work around for this that Git
uses in certain scenarios.

\end{aosasect1}

\begin{aosasect1}{The Toolkit}

Today the Git ecosystem posesses many command-line and UI tools on a number
of operating systems (including Windows, which was originally barely
supported). Most of these tools are mostly built on top of the Git core
toolkit.

Due to the way Git was originally written by Linus and its inception within
the Linux community it was written with a toolkit design philosphy very much
in the Unix tradition of command line tools.

The Git toolkit is divided into two parts: the plumbing and
the porcelain. The plumbing consists of low-level commands that enable
the manipulation of directed acyclic graphs (DAG) and basic content
tracking. The porcelain is the smaller subset of git commands that most
end-users of Git are likely to need to use for maintaining repositories and
communicating between repositories for collaboration.

While the toolkit design has provided enough commands to offer fine grained
access to functionality for many scripters, application developers
complained about the lack of a linkable library for Git. Since the Git binary
calls \code{die()}, it is not reentrant and GUIs, web interfaces or longer
running services would have to fork/exec a call to the Git binary, which can
be slow.

Work is being done to improve the situation for application developers, see
\emph{Current And Future Work} section for more information.
\end{aosasect1}

\begin{aosasect1}{The Repository, Index and Working Areas}

Let us get our hands dirty and dive into using Git locally, if only to
understand a few fundamental concepts.

First to create a new initialized Git repository on our local filesystem
(using a Unix inspired operating system) we can do:
\begin{aosaitemize}
  \item \code{mkdir testgit}
  \item \code{cd testgit}
  \item \code{git init}
\end{aosaitemize}

Now we have an empty, but initialized Git repository sitting in our testgit
directory. We can branch, commit, tag and even communicate with other local
and remote Git repositories. Even communication with other types of VCS
repositories is possible with just a handful of \code{git} commands.

The \code{git init} command creates a .git subdirectory inside of testgit.
Let us have a peak inside of it:
\begin{aosaitemize}
  \item \code{tree .git/}\newline
  \code{
.git/\newline
|-- HEAD\newline
|-- config\newline
|-- description\newline
|-- hooks\newline
|   |-- applypatch-msg.sample\newline
|   |-- commit-msg.sample\newline
|   |-- post-commit.sample\newline
|   |-- post-receive.sample\newline
|   |-- post-update.sample\newline
|   |-- pre-applypatch.sample\newline
|   |-- pre-commit.sample\newline
|   |-- pre-rebase.sample\newline
|   |-- prepare-commit-msg.sample\newline
|   |-- update.sample\newline
|-- info\newline
|   |-- exclude\newline
|-- objects\newline
|   |-- info\newline
|   |-- pack\newline
|-- refs\newline
    |-- heads\newline
    |-- tags\newline
}
\end{aosaitemize}

The .git directory above is by default a subdirectory of the root working
directory, testgit. It contains a few different types of files and
directories:

\begin{aosaitemize}
  \item \emph{Configuration}: the .git/config, .git/description and
  .git/info/exclude files essentially help configure the local repository.
  \item \emph{Hooks}: the .git/hooks directory contains scripts that can
  be run on certain lifecycle events of the repository.
  \item \emph{Staging Area}: the .git/index file (which is not yet
  present in our tree listing above) will provide a staging area for our
  working directory.
  \item \emph{Object Database}: the .git/objects directory is the default
  Git object database, which contains all content or pointers to local
  content. All objects are immutable once created.
  \item \emph{References}: the .git/refs directory is the default location
  for storing reference pointers for both local and remote branches, tags and
  heads. A reference is a pointer to an object, usually of type \emph{tag} or
  \emph{commit}. References are managed outside of the Object Database to
  allow the references to change where they point to as the repository
  evolves. Special cases of references may point to other references, e.g.
  \code{HEAD}.
\end{aosaitemize}

This is the actual repository. The directory that contains the working set
of files is the \emph{working directory}, which is typically the parent of
the .git directory (or \emph{repository}). If you were creating a Git
remote repository that would not have a working directory you could
initialize it using the \code{git init --bare} command. This would create
just the pared down repository files at the root, instead of creating it
as a subdirectory under the working tree.

Another file of great importance is the \emph{Git index}. It provides the
staging area between the local working directory and the local repository.
The index is used to stage specific changes within a file (or more) to
be committed all together. Even if you make changes related to various types
of features, the commits can be made with like changes together to more
logically describe them in the commit message. To selectively stage
specific changes in a file or set of files you can using \code{git add -p}.

The \emph{Git index}, by default, is stored as a single file inside the
repository directory. The paths to these three areas can be customized
using environment variables.

It is helpful to understand the interactions that take place between these
three areas (the repository, index and working areas) during the execution
of a few core Git commands:

\begin{aosaitemize}
  \item \code{git checkout [branch]} \newline
  \small{This will move the HEAD reference of the local repository to branch
  reference path (e.g. \code{refs/heads/master}), populate the index with
  this head data and refresh the working directory to represent the tree
  at that head.}
  \item \code{git add [files]} \newline
  \small{This will cross reference the checksums of the \emph{files}
  specified with the corresponding entries in the Git index to see if the
  index for staged files needs updating with the working directory's
  version. Nothing changes in the Git directory (or repository).}
\end{aosaitemize}

Let us explore what this means more concretely by inspecting the contents of
files under the .git directory (or repository).

\begin{aosaitemize}
  \item \code{GIT\_DIR=\${PWD}/.git}
  \item \code{cat \${GIT\_DIR}/HEAD}
\begin{verbatim}
ref: refs/heads/master
\end{verbatim}
  \item \code{MY\_CURRENT\_BRANCH=\$(cat .git/HEAD | sed 's/ref: //g')}
  \item \code{cat \${GIT\_DIR}/\${MY\_CURRENT\_BRANCH}}
\begin{verbatim}
cat: .git/refs/heads/master: No such file or directory
\end{verbatim}
\end{aosaitemize}

We get an error because, before making any commits to a Git repository at
all, no branches exist, except the default branch in Git is \code{master},
whether it exists yet or not.

Now if we make a new commit then the master branch is created by default for
this commit. Let us do this (continuing in the same shell, retaining
history and context):

\begin{aosaitemize}
  \item \code{git commit -m "Initial empty commit" --allow-empty}
  \item \code{git branch}
\begin{verbatim}
* master
\end{verbatim}
  \item \code{cat \${GIT\_DIR}/\${MY\_CURRENT\_BRANCH}}
\begin{verbatim}
3bce5b130b17b7ce2f98d17b2998e32b1bc29d68
\end{verbatim}
  \item \code{git cat-file -p \$(cat \${GIT\_DIR}/\${MY\_CURRENT\_BRANCH})}
\end{aosaitemize}

What we are starting to see here is the content representation inside Git's
object database.

Let us explore what this means more concretely by inspecting the contents of
files under the .git directory (or repository).

\begin{aosaitemize}
  \item \code{GIT\_DIR=\${PWD}/.git}
  \item \code{cat \${GIT\_DIR}/HEAD}
\begin{verbatim}
ref: refs/heads/master
\end{verbatim}
  \item \code{MY\_CURRENT\_BRANCH=\$(cat .git/HEAD | sed 's/ref: //g')}
  \item \code{cat \${GIT\_DIR}/\${MY\_CURRENT\_BRANCH}}
\begin{verbatim}
cat: .git/refs/heads/master: No such file or directory
\end{verbatim}
\end{aosaitemize}

We get an error because, before making any commits to a Git repository at
all, no branches exist, except the default branch in Git is \code{master},
whether it exists yet or not.

We can verify that no branches exist yet by attempting to list the branches
in our repository with \code{git branch}.

Now if we make a new commit then the master branch is created by default for
this commit. Let us do this (continuing in the same shell, retaining
history and context):

\begin{aosaitemize}
  \item \code{git commit -m "Initial empty commit" --allow-empty}
  \item \code{git branch}
\begin{verbatim}
* master
\end{verbatim}
  \item \code{cat \${GIT\_DIR}/\${MY\_CURRENT\_BRANCH}}
\begin{verbatim}
3bce5b130b17b7ce2f98d17b2998e32b1bc29d68
\end{verbatim}
  \item \code{git cat-file -p \$(cat \${GIT\_DIR}/\${MY\_CURRENT\_BRANCH})}
\end{aosaitemize}

What we are starting to see here is the content representation inside Git's
object database.

\end{aosasect1}

\begin{aosasect1}{The Object Database}

\aosafigure{../images/git/object-hierarchy.png}{Git Objects}{fig.git.objects}

Git has four basic primitive objects that every type of content in the
local repository is built around. Each object type has the following
attributes: \emph{type}, \emph{size} and \emph{content}. The primitive object
types are:
\begin{aosaitemize}
  \item \emph{Tree}: elements in a tree can be another tree or a blob when
  representing a content directory.
  \item \emph{Blob}: a blob represents a file stored in the repository.
  \item \emph{Commit}: a commit points to a tree representing the top level
  directory for that commit as well as parent commits and standard
  attributes.
  \item \emph{Tag}: a tag has a name and points to a commit at the point in
  the repository history that the tag represents
\end{aosaitemize}

All object primitives are referenced by a SHA, a 40-digit object identity,
which has the following properties:
\begin{aosaitemize}
  \item if two objects are identical they will have the same SHA
  \item if two objects are different they will have different SHAs
  \item if an object was only copied partially or another form of data
        corruption occurred, recalculating the SHA of the current object
        will identify such corruption
\end{aosaitemize}

The first two properties of the SHA relating to identity of the objects is
most useful to enable Git's distributed model (the second goal of Git).
The latter property enables some safegaurds against corruption (the third
goal of Git above).

Despite the desirable properties of using DAG based storage for content
storage and merge histories, for many repositories delta storage will be
more space efficient than using \emph{loose} DAG objects.

\end{aosasect1}

\begin{aosasect1}{Storage and Compression Techniques}

Git tackles the storage space problem by "packing" objects in a compressed
format using an index file to point to offsets to specific objects in the
corresponding \emph{packed} file.

\aosafigure{../images/git/packed-format.png}{Diagram of a pack file with
  corresponding index file}{fig.git.pack}

We can count the number of \emph{loose} (or unpacked) objects in the local
Git repository using \code{git count-objects}. Now we can have Git pack
\emph{loose} objects in the object database, remove loose objects already
packed, and find redundant pack files with Git plumbing commands if desired.

There are two versions of pack files as of this writing: version 1 and
version 2. Version 1 being used in versions of Git before version 1.6 of Git
and version 2 of the pack file format being used by default in version 1.6
and above of Git.

In version 1 of the pack file format in Git, only CRC checksums for
the pack file and index file are stored in the index file. This means there
is the possibility of undetectable corruption in the compressed data during
the repacking phase without any further checks. Version 2 of the pack file
format overcomes this problem by including the CRC checksums of each
compressed object in the pack index file. Version 2 also allows packfiles
larger than 4 Gb, which version 1 does not support. As a way to quickly
detect pack file corruption the end of the pack file contains a 20-byte SHA1
sum of the ordered list of all the SHAs in that file.

The emphasis of the newer pack file format is to help fulfill Git second
usability design goal of offering safegaurds against data corruption.

For remote communication Git calculates the commits and content that needs
to be sent over the wire to syncrhonize repositories (or just a branch) and
generates the pack file format on the fly to send back using the desired
protocol of the client (assuming the server supports it).

\end{aosasect1}

\begin{aosasect1}{Merge Histories}

As mentioned previously Git differs fundamentally in merge history approach
than the RCS family of VCSes. Subversion, for example, represents
file or tree history has a linear progression. Whatever has a higher revision
number will supercede anything before it. Branching is not supported directly,
only through an unenforced directory structure within the repository.

Let us first use an example to show how this can be problematic when
maintaining multiple branches of a work. Then we will look at a scenario to
show its limitations.

When working on a "branch" in Subversion at the typical root
\code{branches/branch-name}, we are working on directory subtree adjacent to
the \code{trunk} (typically where the live or \emph{master} equivalent code
resides within). Let us say this branch is to represent parallel development
of the \code{trunk} tree.

As an example,
we might be rewriting a codebase to use a different database. Part of the
way through our rewrite we wish to merge in upstream changes from another
branch subtree (not trunk). We merge in these changes, manually if necessary,
and proceed with our rewrite. Later that day we finish our database vendor
migration code changes on our \code{branches/branch-name} branch and merge
our changes into \code{trunk}. The problem with the way linear history VCSes
like Subversion handle this is that there is no way to know that the
changesets from the other branch are now contained within the trunk.

DAG based merge history VCSes, like Git, handle this case reasonably well.
Assuming the other branch does not contain commits that have not been merged
into our database vendor migration branch (say \code{db-migration} in our
Git repository), we can determine - from the commit object parent
relationships - that a commit on the \code{db-migration} branch contained
the \emph{tip} (or HEAD) of the other upstream branch. Note that a commit
object can have zero or more (bounded by only the abilities of the merger)
parents. Therefore the merge commit on the \code{db-migration} branch
\emph{knows} it merged in the current HEAD of the current branch and the
HEAD of the other upstream branch through the SHA hashes of the parents.
The same is true of the merge commit in the \code{master} (the \code{trunk}
equivalent in Git).

\aosafigure{../images/git/merge-history.png}{Diagram showing merge history
 lineage}{fig.git.merge}

A question that is hard to answer definitively using DAG (and linear) based
merge histories is which commits are contained within each branch. For
example, in the above scenario we assumed we merged in to each branch all
the changes from both branches. This may not be the case.

For simpler cases Git has the ability to \emph{cherry pick} commits from
other branches in to the current branch, assuming the commit can cleanly
be applied to the branch.

\end{aosasect1}

\begin{aosasect1}{What's Next?}

As mentioned previously, Git core as we know it today is based on a toolkit
design philosophy from the Unix world, which is very handy for scripting,
but less useful for embedding inside or linking with longer running
applications or services.

To combat this Shawn Pearce spearheaded an effort to create a linkable Git
library with more permissive licensing that did not inhibit use of the
library. This was called libgit2. It did not find much traction until a
student named, Vincent Marti chose it for his Google Summer of Code project
last year. Since then Vincent and GitHub have continued contributing to the
libgit2 project and created Ruby bindings for it in a project called Rugged.
More recently Python bindings around libgit2 have emerged in an open source
project called pygit2. These three open source projects are maintained
independently of the Git core project.

From this effort another related project that allows Git's object database
to be persisted in a number of different key-value database backends has
emerged such as Memcached, Redis, SQLite and MySQL (at time of writing).

As you can see today there is a wide array of ways to use the Git format.
The face of Git is no longer just the toolkit command line interface of
the Git Core project, rather it is the repository format and protocol to
share between repositories.

As of writing, most of these projects have not reached a stable release
according to their developers, so work in the area still needs to be done,
but the future of Git appears bright.

\end{aosasect1}

\begin{aosasect1}{Lessons Learned}

In software, every design decision is ultimately a trade-off. As a power
user of Git for version control and someone who has developed software
around the Git object database model I have a deep fondness of Git in its
present form. Therefore, these lessons learned are more of a reflection
of common recurring complaints about Git that due to design decisions and
focus of the Git core developers.

One of the most common complaints by developers and managers that evaluate
Git for their project's version control needs has been the lack of IDE
integration on par with other VCS tools. The toolkit design of Git has made
this more challenging than integrating other modern VCS tools into IDE
and related tools.

Earlier in Git's history some of the commands were implemented as shell
scripts. These shell script command
implementations made Git less portable, especially to Windows. I am
sure the Git core developers did not lose sleep over this fact, but it
has impacted adoption of Git in larger organizations negatively due to
portability issues that were prevalent in the early days of Git's
development. Today another project named Git for Windows has been started
by volunteers to ensure new versions of Git are ported to Windows in a
timely manner.

An indirect consequence of designing Git around a toolkit design
with a lot of plumbing commands is that new users get lost quickly from
getting lost in all the available subcommands to not being able to
understand error messages because a low level plumbing task failed. This
has made adopting Git harder for some developer teams.

Even with these complaints about Git, I am excited about the possibilities
of future development on the Git Core project plus all the related open
source projects that have been launched from it.

\end{aosasect1}

\end{aosachapter}
