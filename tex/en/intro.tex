\begin{aosachapter}{Introduction}{s:intro}{Amy Brown and Greg Wilson}

In the introduction to Volume 1 of this series, we wrote:

\begin{quotation}
Building architecture and software architecture have a lot in common,
but there is one crucial difference.  While architects study thousands
of buildings in their training and during their careers, most software
developers only ever get to know a handful of large programs
well{\ldots} As a result, they repeat one another's mistakes rather
than building on one another's successes{\ldots} This book is our
attempt to change that.
\end{quotation}

\noindent
In the year since that book appeared, over two dozen people have
worked hard to create the sequel you have in your hands.  They have
done so because they believe, as we do, that software design can and
should be taught by example---that the best way to learn how think
like an expert is to study how experts think.  From web servers and
compilers through health record management systems to the
infrastructure that Mozilla uses to get Firefox out the door, there
are lessons all around us.  We hope that by collecting some of them
together in this book, we can help you become a better developer.

--- Amy Brown and Greg Wilson

\section*{Contributors}

\emph{Andrey Alexeev (nginx)}: FIXME

\emph{Chris AtLee (Firefox Release Engineering)}: FIXME

\emph{Michael Bayer (SQLAlchemy)}: FIXME

\emph{Lukas Blakk (Firefox Release Engineering)}: FIXME

\emph{Amy Brown (editorial)}: Amy worked in the software industry for
ten years before decamping to create a freelance editing and book production
business. She has an underused degree in Math, two small children, a
husband and a cat. She can be found online at \url{http://www.arbrown.ca/}

\emph{Michael Droettboom (matplotlib)}: FIXME

\emph{Elizabeth Flanagan (Yocto)}: FIXME

\emph{Jeff Hardy (Iron Languages)}: FIXME

\emph{Sumana Harihareswara (MediaWiki)}: Sumana is the community manager for
MediaWiki as the volunteer development coordinator for the Wikimedia
Foundation. She previously worked with the GNOME, Empathy, Telepathy,
Miro, and AltLaw projects. Sumana is an advisory board member for the
Ada Initiative, which supports women in open technology and culture.
She lives in New York City. Her personal site is at
\url{http://www.harihareswara.net/}.

\emph{Tim Hunt (Moodle)}: FIXME

\emph{John Hunter (matplotlib)}: John Hunter is a Quantitative Analyst
at TradeLink Securities.  He received his doctorate in neurobiology at
the University of Chicago for experimental and numerical modeling work
on synchronization, and continued his work on synchronization
processes as a postdoc in Neurology working on epilepsy. He left
academia for quantitative finance in 2005.  An avid python programmer
and lecturer in scientific computing in python, he is original author
and lead developer of the scientific visualization package matplotlib.

\emph{Luis Ibanez (ITK)}: Luis has worked for 11 years on the development of
the Insight Toolkit (ITK), an open source library for medical imaging analysis.
Luis is a strong supporter of open access and the revival of reproducibiliy
verification in scientific publishing. Luis has been teaching a course on Open
Source Software Practices at Rensselaer since 2007.

\emph{Mike Kamermans (Processing.js)}: FIXME

\emph{Luke Kanies (Puppet)}: FIXME

\emph{Brad King (ITK)}: FIXME

\emph{Simon Marlow (The Glasgow Haskell Compiler)}: FIXME

\emph{Kate Matsudaira (Scalable Web Architecture and Distributed Systems)}: FIXME

\emph{Jessica McKellar (Twisted)}: FIXME

\emph{John O'Duinn (Firefox Release Engineering)}: FIXME

\emph{Guillaume Paumier (MediaWiki)}: Guillaume is Technical Communications
Manager at the Wikimedia Foundation, the nonprofit  behind Wikipedia
and MediaWiki. A Wikipedia photographer and editor since 2005, Guillaume
is the author of a Wikipedia handbook in French. He also holds an engineering
degree in physics and a Ph.D in microsystems for life sciences. His home online
is at \url{http://guillaumepaumier.com}.

\emph{Benjamin Peterson (PyPy)}: FIXME

\emph{Simon Peyton-Jones (The Glasgow Haskell Compiler)}: FIXME

\emph{Susan Potter (Git)}: Susan is a polyglot software developer with a
penchant for skepticism. She has been designing, developing and deploying
distributed trading services and applications since 1996, recently switching
to building multi-tenant systems for software firms. Susan is a passionate
power user of Git, Linux, and Vim. You can find her tweeting random thoughts
on Erlang, Haskell, Scala, and (of course) Git @SusanPotter\footnote{\url{https://twitter.com/SusanPotter}}.

\emph{Eric Raymond (GPSD)}: FIXME

\emph{Jennifer Ruttan (OSCAR)}: FIXME

\emph{Stan Shebs (GDB)}: Stan has had open source as his day job since
1989, when a colleague at Apple needed a compiler to generate code for
an experimental VM, and GCC 1.31 was conveniently at hand.  After
following up with the oft-disbelieved Mac System 7 port of GCC (it was
the experiment's control case), Stan went to Cygnus Support, where he
maintained GDB for the FSF and helped on many embedded tools projects.
Returning to Apple in 2000, he worked on GCC and GDB for Mac OS X.  A
short time at Mozilla preceded a jump to CodeSourcery, now part of
Mentor Graphics, where he continues to develop new features for GDB.
Stan's professorial tone is explained by his Ph.D. in Computer Science
from the University of Utah.

\emph{Michael Snoyman (Yesod)}: FIXME

\emph{Jeffrey M.\ Squyres (Open MPI)}: Jeff works in the rack server
division at Cisco; he is Cisco's representative to the MPI Forum
standards body and is a chapter author of the MPI-2 standard.  Jeff is
Cisco's core software developer in the open source Open MPI project.
He has worked in the High Performance Computing (HPC) field since his
early graduate-student days in the mid-1990's.  After some active duty
tours in the military, Jeff received his doctorate in Computer Science
and Engineering from the University Notre Dame in 2004.

\emph{Martin Sustrik (ZeroMQ)}: FIXME

\emph{Christopher Svec (FreeRTOS)}: Chris is an embedded software
engineer who currently develops firmware for low power wireless chips.
In a previous life he designed x86 processors, which comes in handy
more often than you'd think when working on non-x86 processors.
Chris has a Bachelor’s and Master’s Degree in Electrical
and Computer Engineering, both from Purdue University.
He lives in Boston with his wife and golden retriever. You can find
him on the web at \url{http://saidsvec.com}.

\emph{Barry Warsaw (Mailman)}: FIXME

\emph{Greg Wilson (editorial)}: Greg has worked over the past 25 years
in high-performance scientific computing, data visualization, and
computer security, and is the author or editor of several computing
books (including the 2008 Jolt Award winner \emph{Beautiful Code}) and
two books for children.  Greg received a Ph.D.\ in Computer Science
from the University of Edinburgh in 1993.

\emph{Armen Zambrano Gasparnian (Firefox Release Engineering)}: FIXME

\section*{Acknowledgments}

We would like to thank Google for supporting Amy Brown's work on this
project, and Cat Allman for arranging it.  We would also like to thank
all of our technical reviewers:

FIXME

\noindent
especially Tavish Armstrong and Trevor Bekolay, without whose
above-and-beyond this book would have taken a lot longer to produce.

\section*{Contributing}

Dozens of volunteers worked hard to create this book, but there is
still a lot to do.  You can help by reporting errors, by helping to
translate the content into other languages, or by describing the
architecture of other open source projects.  Please contact us at
\code{aosa@aosabook.org} if you would like to get involved.

\end{aosachapter}
