\begin{aosachapter}{Introduction}{s:intro}{Amy Brown and Greg Wilson}

Mention \cite{bib:aosa1}.

And be sure to thank Google for their support!

\section*{Contributors}

\emph{Andrey Alexeev (nginx)}: FIXME

\emph{Chris AtLee (Firefox Release Engineering)}: FIXME

\emph{Michael Bayer (SQLAlchemy)}: FIXME

\emph{Lukas Blakk (Firefox Release Engineering)}: FIXME

\emph{Amy Brown (editorial)}: Amy worked in the software industry for
ten years before decamping to create a freelance editing and book production
business. She has an underused degree in Math, two small children, a
husband and a cat. She can be found online at \url{http://www.arbrown.ca/}

\emph{Jon Cruz (Inkscape)}: FIXME

\emph{Selena Deckelmann (PostgreSQL)}: FIXME

\emph{Michael Droettboom (matplotlib)}: FIXME

\emph{Elizabeth Flanagan (Yocto)}: FIXME

\emph{Jeff Hardy (Iron Languages)}: FIXME

\emph{Sumana Harihareswara (MediaWiki)}: FIXME

\emph{Elise Huard (Erlang VM)}: FIXME

\emph{Tim Hunt (Moodle)}: FIXME

\emph{John Hunter (matplotlib)}: John Hunter is a Quantitative Analyst
at TradeLink Securities.  He received his doctorate in neurobiology at
the University of Chicago for experimental and numerical modeling work
on synchronization, and continued his work on synchronization
processes as a postdoc in Neurology working on epilepsy. He left
academia for quantitative finance in 2005.  An avid python programmer
and lecturer in scientific computing in python, he is original author
and lead developer of the scientific visualization package matplotlib.

\emph{Luis Ibanez (ITK)}: Luis has worked for 11 years on the development of
the Insight Toolkit (ITK), an open source library for medical imaging analysis.
Luis is a strong supporter of open access and the revival of reproducibiliy
verification in scientific publishing. Luis has been teaching a course on Open
Source Software Practices at Rensselaer since 2007.

\emph{Mike Kamermans (Processing.js)}: FIXME

\emph{Luke Kanies (Puppet)}: FIXME

\emph{Nigel Kersten (Puppet)}: FIXME

\emph{Brad King (ITK)}: FIXME

\emph{Simon Marlow (The Glasgow Haskell Compiler)}: FIXME

\emph{Kate Matsudaira ()}: FIXME

\emph{Jessica McKellar (Twisted)}: FIXME

\emph{Sarah Mei (Diaspora)}: FIXME

\emph{John O'Duinn (Firefox Release Engineering)}: FIXME

\emph{Guillaume Paumier (MediaWiki)}: FIXME

\emph{Benjamin Peterson (PyPy)}: FIXME

\emph{Simon Peyton-Jones (The Glasgow Haskell Compiler)}: FIXME

\emph{Susan Potter (Git)}: Susan is a polyglot software developer with
over 14 years of professional experience coding, building, designing,
maintaining and supporting production distributed systems in risk management,
market data and some front office application development for investment
banking, hedge fund and proprietary trading clients. Susan current defines the
system and application architecture, design, coding and testing standards for
a Platform as a Service (PaaS) firm providing hosted front and middle office
applications, web services and APIs.

\emph{Eric Raymond (GPSD)}: FIXME

\emph{Jennifer Ruttan (OSCAR)}: FIXME

\emph{Eugene Sandulenko (scummvm)}: FIXME

\emph{Stan Shebs (GDB)}: Stan has had open source as his day job since
1989, when a colleague at Apple needed a compiler to generate code for
an experimental VM, and GCC 1.31 was conveniently at hand.  After
following up with the oft-disbelieved Mac System 7 port of GCC (it was
the experiment's control case), Stan went to Cygnus Support, where he
maintained GDB for the FSF and helped on many embedded tools projects.
Returning to Apple in 2000, he worked on GCC and GDB for Mac OS X.  A
short time at Mozilla preceded a jump to CodeSourcery, now part of
Mentor Graphics, where he continues to develop new features for GDB.
Stan's professorial tone is explained by his Ph.D. in Computer Science
from the University of Utah.

\emph{Michael Snoyman (Yesod)}: FIXME

\emph{Jeffrey M.\ Squyres (Open MPI)}: Jeff works in the rack server
division at Cisco; he is Cisco's representative to the MPI Forum
standards body and is a chapter author of the MPI-2 standard.  Jeff is
Cisco's core software developer in the open source Open MPI project.
He has worked in the High Performance Computing (HPC) field since his
early graduate-student days in the mid-1990's.  After some active duty
tours in the military, Jeff received his doctorate in Computer Science
and Engineering from the University Notre Dame in 2004.

\emph{Martin Sustrik (ZeroMQ)}: FIXME

\emph{Christopher Svec (FreeRTOS)}: FIXME

\emph{James Turnbull (Puppet)}: FIXME

\emph{Barry Warsaw (Mailman)}: FIXME

\emph{Greg Wilson (editorial)}: Greg has worked over the past 25 years
in high-performance scientific computing, data visualization, and
computer security, and is the author or editor of several computing
books (including the 2008 Jolt Award winner \emph{Beautiful Code}) and
two books for children.  Greg received a Ph.D.\ in Computer Science
from the University of Edinburgh in 1993.

\emph{Harry Wood (OpenStreetMap)}: FIXME

\emph{Armen Zambrano Gasparnian (Firefox Release Engineering)}: FIXME

\section*{Acknowledgments}

FIXME

\section*{Contributing}

Dozens of volunteers worked hard to create this book, but there is
still a lot to do.  You can help by reporting errors, by helping to
translate the content into other languages, or by describing the
architecture of other open source projects.  Please contact us at
\code{aosa@aosabook.org} if you would like to get involved.

\end{aosachapter}
