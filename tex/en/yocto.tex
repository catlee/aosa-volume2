\begin{aosachapter}{Yocto}{s:yocto}{Elizabeth Flanagan}

The Yocto Project\textsuperscript{\textregistered} is an open source
collaborative project that provides a common starting point for
developers of embedded Linux systems to create customized
distributions for embedded products in a hardware agnostic
setting. Sponsored by the Linux Foundation™, Yocto is more than a
build system. It provides tools, processes, templates and methods for
developers to rapidly create and deploy products for the embedded
market. One of the core components of Yocto, is the Poky Build
system. As Poky is a large and complex system, we will be focusing on
one of its core components, BitBake.  BitBake is a Gentoo Portage
inspired build tool, used by both the Yocto Project and OpenEmbedded
communities to utilize metadata in order to create Linux images from
source.

In 2001, Sharp Corporation introduced the SL-5000 PDA, named Zaurus,
which ran an embedded Linux distribution, Lineo. Not long after the
Zaurus's introduction, Chris Larson founded the OpenZaurus Project, a
replacement Linux distribution for the SharpROM, based on a build
system called buildroot. With the founding of the project, people
began contributing many more software packages, as well as targets for
other devices, and it wasn't long that the build system for OpenZaurus
began to show fragility. In January 2003, the community began
discussing a new build system to incorporate the community usage model
of a generic build system for embedded Linux distributions. This would
eventually become OpenEmbedded. Chris Larson, Michael Lauer, and
Holger Schurig began work on OpenEmbedded by porting hundreds of
OpenZaurus packages over to the new build system.

The Yocto Project springs from this work. At the project's core is the
Poky build system, created by Richard Purdie. It began as a stabilized
branch of OpenEmbedded using a core subset of the thousands of
OpenEmbedded recipes, across a limited set of architectures. Over
time, it slowly coalesced into more than just an embedded build
system, but into a complete software development platform, with an
Eclipse plugin, a fakeroot replacement and QEMU based images. Around
November 2010, the Linux Foundation announced that this work would all
continue under the heading of the Yocto Project as a Linux Foundation
sponsored project. It was then established that Yocto and OpenEmbedded
would coordinate on a core set of package metadata called OE-Core
combining the best of both Poky and OpenEmbedded with an increased use
of layering for additional components.

\begin{aosasect1}{Introduction to the Poky Build System}

The Poky build system is the core of the Yocto Project. In Poky's
default configuration, it can provide a starting image footprint that
ranges from a shell accessible minimal image all the way up to a Linux
Standard Base compliant image with a GNOME Mobile and Embedded (GMAE)
based reference user interface called Sato. From these base image
types, metadata layers can be added to extend functionality; layers
can provide an additional software stack for an image type, add a
board support package (BSP) for additional hardware or even represent
a new image type.

Using the 1.1 release of Poky, named
``edison''\footnote{\url{http://downloads.yoctoproject.org/releases/yocto/yocto-1.1/poky-edison-6.0.tar.bz2}},
we will show how BitBake utilizes these recipes and configuration
files to generate an embedded image. From a very high level, the build
process starts out by setting up the shell environment for the build
run. This is done by sourcing a file, oe-init-build-env, that exists
in the root of the poky source tree. This sets up the shell
environment, creates an initial customizable set of configuration
files and wraps the BitBake runtime with a shell script that Poky uses
to determine if the minimal system requirements have been met.

For example, one of the things it will look for is the existence of
Pseudo, a fakeroot replacement contributed to the Yocto Project by
Wind River Systems. At this point, 'bitbake core-image-minimal', for
example, should be able to create a fully functional cross compilation
environment and then create a Linux image based on the image
definition for ``core-image-minimal'' from source as defined
in the yocto metadata layer.

During the creation of our image, BitBake will parse its
configuration, include any additional layers, classes, tasks or
recipes defined and begin by creating a weighted dependency
chain. This process provides an ordered and weighted task priority
map. BitBake then uses this map to determine what packages must be
built in which order in order to most efficiently fulfill compilation
dependencies. Tasks needed by the most other tasks are weighted higher
and thus, run earlier during the build process. The task execution
queue for our build is created. BitBake also stores the parsed
metadata summaries and if on subsequent runs, it determines that the
metadata has changed, it can re-parse only what has changed. The
BitBake scheduler and parser are some of the more interesting
architectural designs of BitBake and some of the decisions surrounding
them and their implementation by BitBake contributors are ones we will
be discussing later.

\aosafigure{../images/yocto/aosa1.jpg}{Poky Build Process}{fig.yocto.execution}

BitBake then runs through its weighted task queue, spawning threads
(up to the number defined by \code{BB\_NUMBER\_THREADS} in
\code{conf/local.conf}) that begin executing those tasks in the
predetermined order. The tasks executed during a package's build may
be modified, prepended or appended to through its recipe. The basic,
default package task order of execution starts by fetching and
unpacking package source and then configuring and cross-compiling the
unpacked source. The compiled source is then split up into packages
and various calculations are made on the compilation result such as
the creation of debug package information.  The split up packages are
then packaged into a supported package format; rpm, ipk or deb package
formats being supported. BitBake will then use these packages to build
the root file system.

\begin{aosasect2}{Poky Build System Concepts}

One of the most powerful aspects of the
Poky build system is that every aspect of a build is controlled by
metadata. Metadata can be loosely grouped into configuration files or
package recipes. A recipe is a
collection of non-executable
metadata used by BitBake to set variables or define additional build
time tasks. A recipe contains fields such as the recipe description,
the recipe version, the license of the package and the upstream
source repository are set. It may also indicate
that the build process utilizes autotools, make, distutils or any
other build process, in which case, the basic functionality can be
defined by classes it inherits from
the OE-Core layers class definitions in \code{./meta/classes}.
Additional tasks can also be
defined, as well as task prerequisites. BitBake
also supports both \_prepend
and \_append as a method of extending task functionality by injecting
code indicated by using prepend or append suffix into the beginning
or end of a task.

\begin{figure}
\begin{verbatim}
DESCRIPTION = "GNU grep utility"
HOMEPAGE = "http://savannah.gnu.org/projects/grep/"
BUGTRACKER = "http://savannah.gnu.org/bugs/?group=grep"
SECTION = "console/utils"
LICENSE = "GPLv3"
LIC_FILES_CHKSUM = "file://COPYING;md5=8006d9c814277c1bfc4ca22af94b59ee"
PR = "r0"
SRC_URI = "${GNU_MIRROR}/grep/grep-${PV}.tar.gz"
SRC_URI[md5sum] = "03e3451a38b0d615cb113cbeaf252dc0"
SRC_URI[sha256sum] = "e9118eac72ecc71191725a7566361ab7643edfd3364869a47b78dc934a357970"

inherit autotools gettext

EXTRA_OECONF = "--disable-perl-regexp"

do_configure_prepend
() {

rm
-f ${S}/m4/init.m4

}
</FONT></FONT>

do_install () {

  autotools_do_install

  install -d ${D}${base_bindir}

  mv ${D}${bindir}/grep ${D}${base_bindir}/grep.${PN}

  mv ${D}${bindir}/egrep ${D}${base_bindir}/egrep.${PN}

  mv ${D}${bindir}/fgrep ${D}${base_bindir}/fgrep.${PN}
}

pkg_postinst_${PN}() {

  update-alternatives --install ${base_bindir}/grep grep grep.${PN} 100

  update-alternatives --install ${base_bindir}/egrep egrep egrep.${PN} 100

  update-alternatives --install ${base_bindir}/fgrep fgrep fgrep.${PN} 100
}

pkg_prerm_${PN}() {

  update-alternatives --remove grep grep.${PN}

  update-alternatives --remove egrep egrep.${PN}

  update-alternatives --remove fgrep fgrep.${PN}
}
\end{verbatim}
\caption{BitBake Recipe for \code{grep}}
\label{fig.yocto.recipe}
\end{figure}

Configuration files can be broken down into two types. Those that
configure BitBake and the overall build run, and those that configure
the various layers Poky uses to create different configurations of a
target image. A layer is any grouping of metadata that provides some
sort of additional functionality. These can be BSP for new devices,
additional image types or additional software outside of the core
layers. In fact, the core Yocto metadata, meta-yocto, is itself a
layer applied on top of the OE-Core metadata layer, meta which adds
additional software and image types to the OE-Core layer.

An example of how one would use layering is by creating a NAS device
for the Intel n660 (Crownbay), utilizing x32, the new 32-bit native
ABI for x86-64, with a custom software layer that adds a user
interface.

Given the task at hand, we could split this functionality out into
layers. At the lowest level we would utilize a BSP layer for Crownbay
that would enable Crownbay-specific hardware functionality, such as
video drivers. As we want x32, we would use the experimental meta-x32
layer. The NAS functionality would be layered on top of this by adding
the Yocto Project's example NAS layer, meta-baryon. And lastly, we'll
use an imaginary layer called meta-myproject, to provide the software
and configuration to create a graphical user interface to enable
configuration of the NAS.

\aosafigure{../images/yocto/aosa2.jpg}{Example of BitBake Layering}{fig.yocto.layering}

During the setup of the BitBake environment, by sourcing
oe-build-init-env, some initial configuration files are generated.
These configuration files allow us quite a bit of control over how and
what Poky generates. The first of these configuration files is
bblayers.conf. This file is what we will use to add additional layers
in order to build our example project.

\begin{figure}
\begin{verbatim}
# LAYER_CONF_VERSION is increased each time build/conf/bblayers.conf
# changes incompatibly
LCONF_VERSION = "4"
BBFILES ?= ""
<BR>BBLAYERS = " \
/home/eflanagan/poky/meta \
/home/eflanagan/poky/meta-yocto \
/home/eflanagan/poky/meta-intel/crownbay \
/home/eflanagan/poky/meta-x32 \<BR>/home/eflanagan/poky/meta-baryon\
/home/eflanagan/poky/meta-myproject \
"
\end{verbatim}
\caption{Example \code{bblayers.conf}}
\label{fig.yocto.bblayers}
\end{figure}

The BitBake layers file, bblayers, defines a variable, BBLAYERS that
BitBake uses to look for BitBake layers. In order to fully understand
this, we should also look at how our layers are actually
constructed. Using
meta-baryon\footnote{\url{git://git.yoctoproject.org/meta-baryon}} as
our example layer, we want to examine the layer configuration
file. This file, conf/layer.conf, is what BitBake parses after its
initial parsing of bblayers.conf. From here it adds additional
recipes, classes and configuration to the build.

\begin{figure}
\begin{verbatim}
# Layer configuration for meta-baryon layer
# Copyright 2011 Intel Corporation
# We have a conf directory, prepend to BBPATH to prefer our versions
BBPATH := "${LAYERDIR}:${BBPATH}"

# We have recipes-* directories, add to BBFILES
BBFILES := "${BBFILES} ${LAYERDIR}/recipes-*/*/*.bb ${LAYERDIR}/recipes-*/*/*.bbappend"

BBFILE_COLLECTIONS += "meta-baryon"
BBFILE_PATTERN_meta-baryon := "^${LAYERDIR}/"
BBFILE_PRIORITY_meta-baryon = "7"
\end{verbatim}
\caption{\code{meta-baryon}'s \code{layer.conf}}
\label{fig.yocto.metabaryon}
\end{figure}

All of the BitBake configuration files help generate BitBake's
datastore which is utilized during the creation of the task execution
queue. During the beginning of a build, BitBake's BBCooker class is
started. The cooker manages the build task execution by 'baking' the
'recipes'. One of the first things the cooker does is attempt to load
and parse configuration data. Remember, though, that BitBake is
looking for two types of configuration data. In order to tell the
build system where it should find this configuration data (and in turn
where to find recipe metadata), the cooker's parseConfigurationFiles
method is called.  With few exceptions, the first configuration files
that the cooker looks for is bblayers.conf, whose main purpose is to
set up global build time variables, such as directory structure naming
for various rootfs directories, the initial LDFLAGS to be used during
compile time. Most end users will never touch this file as most
anything needed to be changed here would be within a recipe context as
opposed to build wide. As this file is parsed, BitBake also includes
configuration files that are relative to each layer in BBLAYERS and
adds the variables found in those files to its data store.

\begin{figure}
\begin{verbatim}
include conf/site.conf
include conf/auto.conf
include conf/local.conf
include conf/build/${BUILD_SYS}.conf
include conf/target/${TARGET_SYS}.conf
include conf/machine/${MACHINE}.conf
\end{verbatim}
\caption{Portion of BitBake.conf Showing Included Configuration Files}
\label{fig.yocto.included}
\end{figure}

\end{aosasect2}

\end{aosasect1}

\begin{aosasect1}{BitBake Architecture}

Before we delve into some of BitBake's current architectural design,
it would help to understand how BitBake once worked. For this, in
order to fully appreciate how far BitBake has come, we will use the
initial version, BitBake 1.0. In that first release of BitBake, a
build's dependency chain was determined based upon recipe
dependencies. If something failed during the build of an image,
BitBake would move on to the next task and try to run it again later.
What this means, obviously, is that builds took a very long time. One
of the things BitBake also did is keep each and every variable that a
recipe used in one very large dictionary. Given the number of recipes
and the number of variables and tasks needed to accomplish a build,
BitBake 1.0 was a memory hog. At a time when memory was expensive and
systems did not have much, builds could be painful affairs. It was not
unheard of for a system to run out of memory (writing to swap!)  as it
slugged through a long running build. In its first incarnation, while
it did the job (sometimes), it did it slowly while consuming enormous
resources. Worse, as BitBake 1.0 had no concept of a data persistence
cache or shared state, it also had no ability to do incremental
builds. So if a build failed, one would have to restart it from
scratch.

A quick diff between the current BitBake version used in Poky
``edison'' 1.13.3 and 1.0 shows the implementation of BitBake's
client-server architecture, the data persistence cache, its datastore,
a copy-on-write improvement for the datastore, shared state
implementation and drastic improvements over how it determines task
and package dependency chains. This evolution has made it more
reliable, more efficient and more dynamic. Much of this functionality
came out of necessity for quicker more reliable builds that used less
resources. Three improvements to BitBake that we will examine is the
implementation of a client-server architecture, optimizations around
BitBake's data storage and work done around how BitBake determines its
build and task dependency chain.

\begin{aosasect2}{BitBake IPC}

Since we now know a good deal about how the Poky build system utilizes
configurations, recipes and layers to create embedded images, we're
prepared to begin to look under the hood of BitBake and examine how
this is all combined. Starting with the core BitBake executable,
bitbake/bin/bake, we can begin to see the process BitBake takes as it
begins to set up the infrastructure needed to begin a build. The first
item of interest is BitBake's Interprocess Communications
(IPC). Initially, BitBake had no actual concept of a
client-server. This functionality was factored into the BitBake design
over a period of time in order to allow BitBake to run multiple
processes during a build, as it was initially single threaded, and to
allow different user experiences.

\aosafigure{../images/yocto/aosa3.jpg}{Overview of BitBake IPC}{fig.yocto.ipc}

All Poky builds are begun by starting a user interface instance. The
user interface provides a mechanism for logging of build output, build
status and build progress as well as receiving events from build tasks
through the BitBake event module.  The default user interface used is
knotty, BitBake's command line interface. Called knotty, or ``(no)
tty'', since it handles both ttys and non-ttys, it is one of a few
interfaces that are supported.  One of these additional user
interfaces is Hob. Hob is the graphical interface to BitBake, a kind
of ``BitBake commander''. In addition to the typical functions you
would see in the knotty user interface, hob, written by Joshua Lock,
brings the ability to modify configuration files, add additional
layers and packages, and fully customize a build.

BitBake user interfaces have the ability to send commands to the next
module brought up by the BitBake executable, the BitBake server. Like
the user interface, BitBake also supports multiple different server
types, amongst them XMLRPC. The default server that most users use
when executing BitBake from the knotty user interface is BitBake's
process server. After bringing up the server, the BitBake executable
brings up the cooker.

The cooker is a core portion of BitBake and is where most of the
particularly interesting things that occur during a Poky build are
called from. The cooker is what manages the parsing of metadata,
initiates the generation of the dependency and task trees, and manages
the build. One of the functions of BitBake's server architecture is
allowing multiple ways of exposing the command API, indirectly, to the
user interface. The command module is the worker of BitBake, running
build command and triggering events that get passed up to the user
interface through BitBakes event handler.  Once the cooker is brought
up from the BitBake executable, it initializes the BitBake datastore
and then begins to parse all of Poky's configuration files. It then
creates the runqueue object, and triggers off the build.

\end{aosasect2}

\begin{aosasect2}{BitBake DataSmart Copy-on-Write Data Storage}

In BitBake 1.0, all BitBake variables were parsed and stored in one
very large dictionary during the initialization of that version's data
class. As previously mentioned, this was problematic, in that very
large python dictionaries are slow on writes and member access and if
the build host runs out of physical memory during the build, we end up
hitting swap. While this is less likely in most systems in late 2011,
when OpenEmbedded and BitBake were first starting up, the average
computer's specification did not generally have more than one or two
gigabytes of memory, and usually, quite less.

This was one of the major pain points in early BitBake. Two major
issues needed to be worked out in order to help increase performance.
One was precomputation of the build dependency chain. The other was to
reduce the size of data being stored in memory. As much of the data
being stored for a recipe doesn't change from recipe to recipe, for
example, TMPDIR, BB\_NUMBER\_THREADS and other global BitBake
variables, having a copy of the entire data environment per recipe
stored in memory was inefficient. The solution was Tom Ansell's
copy-on-write dictionary that ``abuses classes to be nice and fast''.
BitBake's COW module is both an especially fearless and clever hack.
Running 'python BitBake/lib/bb/COW.py' and examining the module will
give you an idea of how this copy-on-write implementation works and
how BitBake uses it to store data efficiently

The DataSmart module, which utilizes the COW dictionary, stores the
data from the initial poky configuration, data from .conf files and
.bbclass files, in a dict as a data object. Each of these objects can
contain another data object of just the diff of the data. So if a
recipe changes something from the initial data configuration, instead
of copying the entire configuration in order to localize it, a diff of
the parent data object is stored at the next layer down in the in the
COW stack. When an attempt is made to access a variable, the data
module will use DataSmart to look into the top level of the stack. If
the variable is not found it will defer to a lower level of the stack
until it does find the variable or errors out.

One of the other interesting things about the DataSmart module centers
around variable expansion. As BitBake variables can contain executable
python, one of the things that needs to be done is the variable needs
to be run through BitBakes bb.codeparser to ensure that it's valid
python and that it contains no circular references.  An example of a
variable containing python is this example taken from
./meta/conf/distro/include/tclibc-eglibc.inc:

\begin{verbatim}
LIBCEXTENSION = "${@['', '-gnu'][(d.getVar('ABIEXTENSION', True) or '') != '']}"
\end{verbatim}

This variable is included from one of the OE-Core configuration files,
\code{./meta/conf/distro/include/defaultsetup.conf} and is used to
provide a set of default options across different distro
configurations that one would want to lay on top of Poky or
OpenEmbedded. This file imports some eglibc specific variables that
are set dependent on the value of another BitBake variable
ABIEXTENSION. During the creation of the datastore, the python within
this variable needs to be parsed and validated to ensure tasks that
utilize this variable will not fail.

\end{aosasect2}

\begin{aosasect2}{BitBake Scheduler}

Once BitBake has parsed the configuration and created its datastore,
it needs to parse the recipes required for the image and produce a
build chain. This is one of the more substantial improvements to
BitBake. Originally, BitBake took its build priorities from a
recipe. If a recipe had a DEPENDS, it would try to figure out what to
build in order to satisfy that dependency. If a task failed because it
lacked a prerequisite needed for its buildout, it was simply put to
the side and attempted later. This had obvious drawbacks, both in
efficiency and reliability.

As no precomputed dependency chain was established, task execution
order was figured out during the build run. This limited BitBake to
being single threaded as at no time. To give an idea of how painful
single threaded BitBake builds can be, the smallest image
``core-image-minimal'' on a standard developer machine in 2011
(Core\textsuperscript{\textregistered} i7, 16 gigabytes of DDR3
memory) takes about three or four hours to build a complete cross
compilation toolchain and use it to produce packages that are then
used to create an image. For reference, a build on the same machine
with BB\_NUMBER\_THREADS at 14 and PARALLEL\_MAKE set to ``-j 12''
takes about 30 to 40 minutes. As one could imagine running single
threaded with no precomputed order of task execution on slower
hardware that had less memory with a large portion wasted by duplicate
copies of the entire datastore took much longer.

\begin{aosasect3}{Dependencies}

When we talk of build dependencies, we need to make a distinction
between the various types. A build dependency, or DEPENDS, is
something we require as a prerequisite so that Poky can build the
required package, whereas a runtime dependency, RDEPENDS, requires
that the image the package is to be installed on also contain the
package listed as an RDEPENDS. Take, for example, the package
task-core-boot. If we look at the recipe for it in
meta/recipes-core/tasks/task-core-boot.bb we will see two BitBake
variables set. RDEPENDS and DEPENDS. BitBake uses these two fields
during the creation of its dependency chain. 

\begin{figure}
\begin{verbatim}
DEPENDS = "virtual/kernel"
 ...

RDEPENDS_task-core-boot = "\
base-files \
base-passwd \
busybox \
initscripts \
...
\end{verbatim}
\caption{Portion of task-core-boot.bb showing DEPENDS and RDEPENDS}
\label{fig.yocto.depends}
\end{figure}

Packages aren't the only thing in BitBake with dependencies. Tasks
also have their own dependencies.  Within the scope of BitBake's
runqueue, we recognize four types: Internally dependent, DEPENDS
dependent, RDEPENDS dependent and inter-task dependent.

Internally dependent tasks are set within a recipe and add a task
before and/or after another task. For example, in a recipe, we could
add a task called ``do\_deploy'' by adding the line 'addtask deploy
before do\_build after do\_compile'.  This would add a dependency for
running the do\_deploy task prior to do\_build being started, but
after do\_compile is completed. DEPENDS and RDEPENDS dependent tasks
are tasks that run after a denoted task.  For example, if we wanted to
run do\_deploy of a package after the do\_install of its DEPENDS or
RDEPENDS, our recipe would include ``do\_deploy[deptask] =
'do\_install' `` or ``do\_deploy[rdeptask] = 'do\_install' ``. For
inter-task dependencies, if we wanted a taks to be dependant on a
different package's task we would add, using the above example of
do\_deploy, do\_deploy[depends] = ``{\textless}target's
name{\textgreater}:do\_install''.

\end{aosasect3}

\begin{aosasect3}{RunQueue}

As an image build out can have hundreds of recipes, each with multiple
packages and task, each of with its own dependency, BitBake is now
tasked with trying to sort this out into something it can use as an
order of execution. A</SPAN>fter the cooker has gotten the entire list
of packages needed to be built from the initialization of the bb.data
object, it will begin to create a weighted task map from this data in
ordered to produce an ordered list of tasks it needs to run, called
the runqueue. Once the runqueue is created, BitBake can begin
executing it in priority order, tasking out each portion to a
different thread.

Within the provider module, BitBake will first look to see if there is
a PREFERRED\_PROVIDER for a given package or image. As more than one
recipe can provide a given package and as tasks are defined in
recipes, BitBake needs to decide which provider of a package it will
use. It will sort all the providers of the package, weighting each
provider by various criteria. For example, preferred versions of
software will get a higher priority than others, however, BitBake also
takes into account package version as well as the dependencies of
other packages. Once it has selected the recipe from which it will
derive its package, BitBake will iterate over the DEPENDS and RDEPENDS
of that recipe and proceed to compute the providers for those
packages. This chain reaction will produce a list of packages need to
for image generation as well as providers for those packages.

Runqueue now has a full list of all packages needed to be built and a
dependency chain. In order to begin execution of the build, the
runqueue module now needs to create the TaskData object in order to
begin to sort out a weighted task map. It begins by taking each
buildable package it has found, spliting out the tasks needed to
generate that package and weighing each of those tasks based upon the
number of packages that require it. Tasks with a higher weight have
more dependants, and therefore are generally run earlier in the
build. Once this is complete, the runqueue module then prepares to
convert the TaskData object into a runqueue.

The creation of the runqueue is somewhat complex. BitBake first
iterates through the list of task names within the TaskData object in
order to determine task dependencies. As it iterates through TaskData,
it begins to build a weighted task map. When it is complete, if it has
found no circular dependencies, unbuildable tasks or any such
problems, it will then order the task map by weight and return a
complete runqueue object to the cooker. The cooker will begin to
attempt to execute the runqueue, task by task. Depending upon image
size and computing resources, Poky may take from a half hour to hours
in order to generate a cross compilation toolchain, a package feed and
the embedded Linux image specified. It is worth noting that from the
time of executing 'bitbake
{\textless}image\_name{\textgreater}' from the commandline,
the entire process up to right before the execution of the task
execution queue has taken less than a few seconds.

\end{aosasect3}

\end{aosasect2}

\end{aosasect1}

\begin{aosasect1}{Conclusion}

In my discussions with community members and my own personal
observations, I've identified a few areas where things should have
perhaps been done differently as well as a few valuable lessons. It is
important to note, that ``arm chair quarterbacking'' a decade long
development effort is not meant as a criticism of those who've poured
their time and effort into a wholly remarkable collection of
software. As developers, the most difficult part of our job is
predicting what we will need years down the road and how we can set up
a frame work to enable that work now. Few can achieve that without
some road bumps.

The first lesson I would point out is to make sure to develop a
written, agreed upon standards document exists and that is well
understood by the community. It should be designed for maximum
flexibility and growth. One place where I've personally run into this
issue is with my work in OE-Core's license manifest creation class,
especially with my experiences working with the LICENSE variable.

As no clearly documented standard existed for what LICENSE should
contain, a review of the many recipes available showed many
variations. The various LICENSE strings contained everything from
python abstract syntaxt tree parsable values to values that one would
have little hope of gaining meaningful data from. There was a
convention that was commonly used within the community; however the
convention had many variations, some less correct than others. This
wasn't the problem of the developer who wrote the recipe. It was a
community failure to define a standard.

As little prior work was actually done with the LICENSE variable
outside of checking for its existence, there was no particular concern
about a standard for that variable. Much trouble could have been
avoided had a project wide agreed upon standard been developed early
on.

The next lesson is a bit more general and speaks to an issue seen not
only within the Yocto Project but can also be applied to other large
scale projects that are systems design specific. It is the one of the
most important things developers can do to limit the amount of effort
duplication, refactoring and churn their project encounters.

If you think you've spent enough time on architectural design, you
probably haven't. If you think you haven't spent enough time on
architectural design, you definitely haven't. Spending more time on
front end planning won't stop you from later having to rip apart code
or even do major architectural changes, but it will certainly reduce
the amount of effort duplication needed in the long run. Designing
your software to be as modular as possible knowing that you will end
up revisiting areas for anything from minor tweaks to major rewrites
will make it so that when you do run into these issues, code rewrites
are less hair raising.

Some obvious places where this would have helped is identifying the
needs of end users with low memory systems. Had more thought been put
into bitbake's datastore earlier, perhaps we could have predicted the
problems associated with the datestore taking up too much memory and
dealt with it earlier.

The lesson here is that while it is nearly impossible to identify
every pain point your project will run into during its lifetime,
taking the time to do serious front end planning will help reduce the
effort needed later. BitBake, OE-Core and Yocto are all fortunate in
this regard as there was a fair amount of architectural planning done
early. This enabled us to be able to do major changes to the
architecture without too much pain and suffering.

\end{aosasect1}

\begin{aosasect1}{Acknowledgements}

First, thank you goes to Chris Larson, Michael Lauer, and Holger
Schurig and the many many people who have contributed to BitBake,
OpenEmbedded, OE-Core and Yocto over the years. Thank you also goes to
Richard Purdie for his letting me pick his brain, both on historical
and technical aspects of OE and his constant encouragement and
guidance, especially with some of the dark magic of BitBake.

\end{aosasect1}

\end{aosachapter}
